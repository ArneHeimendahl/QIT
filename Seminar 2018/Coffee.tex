\documentclass[11pt]{amsart}

\usepackage{amssymb}
\usepackage{dutchcal}        %nice Hilbertspace symbols
\usepackage{braket}          %for braket notation
\usepackage[utf8]{inputenc}
\usepackage{bm}              %bold writing
\usepackage{tikz}            %for drawing some nice graphics
\usepackage{tikzsymbols}
\usepackage[version=3]{mhchem}%chemical formulas nicely written
\usepackage{graphicx}        %some graphics stuff, like subfigures
\usepackage{subcaption}
\usepackage{wrapfig}         %figures in text float
\usepackage{cite}            %to cite
\usepackage{dsfont}           %for unity matrix
\usepackage{nicefrac}
\usepackage{xcolor}
\usepackage{xspace}
\usepackage{setspace}
\usepackage{tikz}
\usepackage{mathtools}
\usepackage{enumitem}
\usepackage[top=2.5cm, bottom=2.5cm, left=3.5cm, right=3.5cm]{geometry}
\usepackage{lmodern}		% nice schriftart
\usepackage{etoolbox}        %reformat sections etc.
\usepackage{float}									% forcieren von H bei Grafiken
\usepackage[unicode=true,							% Hyperlinks
bookmarks=true,bookmarksnumbered=false,bookmarksopen=false,
breaklinks=false,pdfborder={0 0 0},backref=false,colorlinks=false]
{hyperref}
\DeclareMathAlphabet{\mathpzc}{OT1}{pzc}{m}{it}

\makeatletter
\let\@afterindenttrue\@afterindentfalse
\makeatother
\pagestyle{plain}   %no title at every page
\graphicspath{ {images/} }





%--------------------------only for tikz---------------------------------------------------------
\newcommand{\xdownarrow}[1]{%
	{\left\downarrow\vbox to #1{}\right.\kern-\nulldelimiterspace}
}
\newcommand{\xuparrow}[1]{%
	{\left\uparrow\vbox to #1{}\right.\kern-\nulldelimiterspace}
}

\newcommand{\xswarrow}[1]{%
	{\left\swarrow\vbox to #1{}\right.\kern-\nulldelimiterspace}
}

\usetikzlibrary{shapes.callouts,calc,arrows,decorations.pathmorphing,intersections,
	decorations.pathreplacing
}
\tikzset{
	fatlevel/.style   = { ultra thick, black },
	level/.style   = {thick, black },
	vertex/.style = {very thick, black,opacity=1, shorten <= 0.15cm, shorten >= 0.15cm},
	virtual/.style={thick,densely dashed},
	connect/.style = { dashed, black },
	redtransition/.style = {thick,->,>=stealth',shorten >=1pt, red},
	transition/.style = {thick, black,->,>=stealth',shorten >=1pt},
	trans/.style={thick,<->,>=stealth},
	notice/.style  = { draw, rectangle callout, callout relative pointer={#1} },
	label/.style   = { text width=3cm },
	llabel/.style   = { text width=9.5cm },
	interface/.style={
		% The border decoration is a path replacing decorator. 
		% For the interface style we want to draw the original path.
		% The postaction option is therefore used to ensure that the
		% border decoration is drawn *after* the original path.
		postaction={draw,decorate,decoration={border,angle=135,
				amplitude=0.3cm,segment length=2mm}}},
}
%-----------------------------------------------------------

\definecolor{colororange}{HTML}{E65100} % orange
\definecolor{colordgray}{HTML}{795548} % dark gray for note
\definecolor{colorhgray}{HTML}{212121} % heavy dark gray for normal text
\definecolor{colorgreen}{HTML}{009688} % green
\definecolor{colorlgray}{HTML}{FAFAFA} % background light gray
\definecolor{colorblue}{HTML}{0277BB} % blue

%-----------------------------------------------------------








%------------------------- Theorem enviroments -----------------------------------------
\newtheoremstyle{mytheorem}% ⟨name⟩
{12pt}% ⟨Space above⟩
{12pt}% ⟨Space below⟩
{\itshape}% ⟨Body font⟩
{}% ⟨Indent amount⟩
{\bf}% ⟨Theorem head font⟩
{.}% ⟨Punctuation after theorem head⟩
{.5em}% ⟨Space after theorem head⟩
{}% ⟨Theorem head spec (can be left empty, meaning ‘normal’)⟩
\theoremstyle{mytheorem}
\newtheorem{theo}{Theorem}

\newtheoremstyle{mylemma}% ⟨name⟩
{12pt}% ⟨Space above⟩
{12pt}% ⟨Space below⟩
{\itshape}% ⟨Body font⟩
{}% ⟨Indent amount⟩
{\bf}% ⟨Theorem head font⟩
{.}% ⟨Punctuation after theorem head⟩
{.5em}% ⟨Space after theorem head⟩
{}% ⟨Theorem head spec (can be left empty, meaning ‘normal’)⟩
\theoremstyle{mytheorem}
\newtheorem{lemma}[subsubsection]{Lemma}

\newtheoremstyle{kor}% ⟨name⟩
{12pt}% ⟨Space above⟩
{12pt}% ⟨Space below⟩
{\itshape}% ⟨Body font⟩
{}% ⟨Indent amount⟩
{\bf}% ⟨Theorem head font⟩
{.}% ⟨Punctuation after theorem head⟩
{.5em}% ⟨Space after theorem head⟩
{}% ⟨Theorem head spec (can be left empty, meaning ‘normal’)⟩
\theoremstyle{mytheorem}
\newtheorem{kor}{Corollary}

\newtheoremstyle{prop}% ⟨name⟩
{12pt}% ⟨Space above⟩
{12pt}% ⟨Space below⟩
{\itshape}% ⟨Body font⟩
{}% ⟨Indent amount⟩
{\bf}% ⟨Theorem head font⟩
{.}% ⟨Punctuation after theorem head⟩
{.5em}% ⟨Space after theorem head⟩
{}% ⟨Theorem head spec (can be left empty, meaning ‘normal’)⟩
\theoremstyle{mytheorem}
\newtheorem{prop}{Proposition}

\newtheoremstyle{conj}% ⟨name⟩
{12pt}% ⟨Space above⟩
{12pt}% ⟨Space below⟩
{\itshape}% ⟨Body font⟩
{}% ⟨Indent amount⟩
{\bf}% ⟨Theorem head font⟩
{.}% ⟨Punctuation after theorem head⟩
{.5em}% ⟨Space after theorem head⟩
{}% ⟨Theorem head spec (can be left empty, meaning ‘normal’)⟩
\theoremstyle{mytheorem}
\newtheorem{conj}{Conjecture}

\newtheoremstyle{mydef}% ⟨name⟩
{12pt}% ⟨Space above⟩
{16pt}% ⟨Space below⟩
{}% ⟨Body font⟩
{}% ⟨Indent amount⟩
{\bf}% ⟨Theorem head font⟩
{.}% ⟨Punctuation after theorem head⟩
{.5em}% ⟨Space after theorem head⟩
{}% ⟨Theorem head spec (can be left empty, meaning ‘normal’)⟩
\theoremstyle{mydef}
\newtheorem{dfn}[subsubsection]{Definition}


\newtheoremstyle{myremark}% ⟨name⟩
{12pt}% ⟨Space above⟩
{16pt}% ⟨Space below⟩
{\itshape}% ⟨Body font⟩
{}% ⟨Indent amount⟩
{\bf}% ⟨Theorem head font⟩
{}% ⟨Punctuation after theorem head⟩
{.5em}% ⟨Space after theorem head⟩
{}% ⟨Theorem head spec (can be left empty, meaning ‘normal’)⟩
\theoremstyle{myremark}
\newtheorem{rmk}{Remark}
\newtheorem*{rmk*}{Remark}
\newtheorem*{qst*}{Question}


\newtheoremstyle{myex}% ⟨name⟩
{0pt}% ⟨Space above⟩
{16pt}% ⟨Space below⟩
{\itshape}% ⟨Body font⟩
{}% ⟨Indent amount⟩
{\bf}% ⟨Theorem head font⟩
{}% ⟨Punctuation after theorem head⟩
{.5em}% ⟨Space after theorem head⟩
{}% ⟨Theorem head spec (can be left empty, meaning ‘normal’)⟩
\theoremstyle{myex}
\newtheorem{ex}{Example}
\newtheorem*{ex*}{Example}





%-------------------------- reformat sections and subsections
\patchcmd{\subsection}{\bfseries}{\itshape\bfseries\centering}{}{}
\patchcmd{\subsection}{-.5em}{.5em}{}{}
\patchcmd{\section}{\normalfont}{\normalfont\bfseries\Large}{}{}
\patchcmd{\section}{-.5em}{.5em}{}{}
\patchcmd{\subsubsection}{\normalfont}{\itshape\bfseries\centering}{}{}
\patchcmd{\subsubsection}{-.5em}{.5em}{}{}


%---------------- space between page numbering
	\addtolength{\textheight}{-\baselineskip}
	\addtolength{\footskip}{\baselineskip}


%---------------- section spacing and table of contents formatting
	\setcounter{tocdepth}{3}% to get subsubsections in toc
	\let\oldtocsection=\tocsection
	\let\oldtocsubsection=\tocsubsection
	\let\oldtocsubsubsection=\tocsubsubsection
	\renewcommand{\tocsection}[2]{\hspace{0em}\oldtocsection{#1}{#2}}
	\renewcommand{\tocsubsection}[2]{\hspace{2em}\oldtocsubsection{#1}{#2}}
	\renewcommand{\tocsubsubsection}[2]{\hspace{4em}\oldtocsubsubsection{#1}{#2}}





%---------------------------- definitions
\newcommand{\el}{ {\scriptstyle{\in}}\; }
\newcommand{\sumel}{ {\scriptscriptstyle{\in}} }
\newcommand{\Hilb}{ {\mathcal{H}} }
\newcommand{\dualHilb}{ {\mathcal{H}^*} }
\newcommand{\Haml}{ {\hat{H}} }
\newcommand{\modul}[1]{{\left|{#1}\right|}}
\newcommand{\norm}[1]{\ensuremath{ \Vert {#1} \Vert}}
\newcommand{\trace}[1]{\ensuremath{\text{Tr} \, {#1} }}
\newcommand{\QC}{\ensuremath{\textup{QC}}}
\newcommand{\LC}{\ensuremath{\textup{LC}}}



\DeclareMathOperator*{\esssup}{ess\,sup} % essentiellen Supremums
\DeclareMathOperator{\spn}{span} % Span
\DeclareMathOperator{\supp}{supp} % Träger
\DeclareMathOperator{\proj}{proj} % Träger
\DeclareMathOperator{\tr}{Tr} 
\DeclareMathOperator{\sgn}{sign}
 %----- put in Preamble


\title{Seminar paper \\ Nonlocal games and the Grothendieck-Tsirelson inequality}
\author{Maxi Brandstetter, Arne Heimendahl, Felix Kirschner}

\begin{document}
\pagenumbering{roman}	

%adjust equation numberung 
\numberwithin{equation}{section}


\begin{abstract}
This paper is an elaboration and review of nonlocal games and the Grothendieck-Tsirelson inequality, our topic for the seminar \emph{Introduction to quantum information theory and quantum computing}, which took place on the 20th and 21st of September 2018 at the University of Cologne. We introduce some of the mathematical groundwork that helps us describe the quantum mechanical world. We define nonlocal games, which are mathematical concepts that enable us to explore the difference of classical and quantum mechanics. An example is given, that highlights the phenomenon of entanglement, which gives correlated measurement outcomes, where these are not expected. The mathematical structures underlying nonlocal games are discussed in detail and again it is shown that the quantum mechanical world provides more powerful tools than the of classical mechanics. Last but not least, we come across one of the many instances where the celebrated Grothendieck inequality comes in handy. The beautiful proof of the Grothendieck inequality by Krivine is given and the Grothendieck-Tsirelson theorem is elaborated. 
\end{abstract}


\maketitle	
\tableofcontents
	
	
\newpage	
\pagenumbering{arabic}



%\section{Einleitung}    %%Einleitung erstmal raus 
	\vspace{8pt}
%	Einleitung. Es geht um Grothendieck Ungleichung und so weiter. 


%---------------------------------------------------------------------------
\let \oldsection \section
\renewcommand{\section}{\vspace{15pt plus 8pt}\oldsection}
\let \oldsubsection \subsection
\renewcommand{\subsection}{\vspace{10pt plus 3pt}\oldsubsection}
%---------------------------------------------------------------------------



\newpage
\section{Introduction} 
	\vspace{8pt}
	In order to make sure everyone is on the same page we will have the following introduction where the necessary groundwork is done. The aim is to call a few basic definitions back to our memory so one can fluently read through this paper. \\
A complex matrix $A \in \mathbb{C}^{n \times n}$ is called Hermitian if $A^*=A$, where $A^*$ denotes the conjugate transpose of $A$. A complex Hermitian matrix $A$ is called positve semidefinite (abbreviated psd.) if one of the following holds: 
\begin{enumerate}
\item[i,] The matrix has only real non-negative eigenvalues 
\item[ii,] There exist complex $n-$dimensional vectors $z_1, \dots, z_n$ s.t. $A_{i,j} = \langle z_i, z_j \rangle = \sum_{k=1}^n \overline{z_{i_k}}z_{j_k}$
\item[iii,] For every $z \in \mathbb{C}^n$ we have $z^*Az\ge 0$
\item[iv,] There exists a complex matrix $B$ s.t. $A=B^*B$
\end{enumerate}
It can be shown that $i, - iv,$ are in fact equivalent. The set of positive semidefinite matrices is a cone, meaning for two psd. $n \times n$ matrices $A,B$ and $\alpha, \beta \in \mathbb{R}_+$ we have that $\alpha A + \beta B $ is also positive semidefinite. The set of real $n \times n$ matrices is denoted by $\mathcal{S}_n^+$.\\
In case someone is not familiar with the \textit{tensor product} a short introduction with an example or two is given here: 
Let $\mathcal{X} = \mathbb{C}^{n_1 \times m_1}$ and $\mathcal{Y}= \mathbb{C}^{n_2 \times m_2}$. Then the tensor product of the vector spaces $\mathcal{X}$ and $\mathcal{Y}$ is defined as $\mathcal{X} \otimes \mathcal{Y} = \mathbb{C}^{n_1n_2 \times m_1m_2}$. The tensor product of complex matrices can be obtained as follows: Index the rows and columns of a matrix by $\mathcal{R}$ and $\mathcal{C}$ and think of the matrix as a map from $\mathcal{R} \times \mathcal{C} \rightarrow \mathbb{C}$. For two complex matrices $A: \mathcal{R_1} \times \mathcal{C_1} \rightarrow \mathbb{C}$ and $B: \mathcal{R_2} \times \mathcal{C_2} \rightarrow \mathbb{C}$ their tensor product is the matrix $A \otimes B : (\mathcal{R_1} \times \mathcal{R_2}) \times (\mathcal{C_1} \times \mathcal{C_2})\rightarrow \mathbb{C}$ defined by $(A \otimes B)( (r_1,r_2),(c_1,c_2))= A(r_1,c_1)B(r_2,c_2)$. Considering a lexicographic order understand tensor products in the following way: $A \otimes B = \begin{pmatrix}
a_{11}B & \dots & a_{1n}B \\
\vdots && \vdots \\
a_{n1}B & \dots & a_{nn}B
\end{pmatrix}$, which is the \textit{Kronecker product} and coherent with the definition. For two complex vectors $v_1,v_2$ when we talk about their tensor products we will mean $v_1 \otimes v_2 = ( v_{1_1}v_2, v_{1_2}v_2, \dots , v_{1_n}v_2)^\top$ from which we can deduce that $\langle x_1 \otimes x_2 , y_1 \otimes y_2 \rangle = \langle x_1, x_2 \rangle \langle y_1 , y_2 \rangle$. Also for any matrices $A,B,C,D$ (assuming fitting dimensions) we have the following identities: 
\begin{enumerate}
\item[i,] $(A \otimes B ) \otimes C = A \otimes (B \otimes C) $
\item[ii,] $A \otimes (B + C) = A \otimes B + A \otimes C $
\item[iii,] $(A \otimes B)(C \otimes D) = (AC) \otimes (BD)$
\end{enumerate}
Also throughout this paper we will stick to the \textit{Dirac notation}, which is the standard notation for describing quantum states. In Dirac notation $\vert \psi \rangle$ refers to a vector in $\mathbb{C}^n$. The conjugate transpose of this vector is written $ \langle \psi \vert$. The non-negative integers, by convention, represent the canonical basis vectors, i.e. 
\begin{equation}
\vert 0 \rangle = \begin{pmatrix}
1 \\
0 \\
0 \\
\vdots \\
0
\end{pmatrix}, \vert 1 \rangle = \begin{pmatrix}
0 \\
1 \\
0 \\
\vdots \\
0 
\end{pmatrix} , \dots , \vert n-1 \rangle = \begin{pmatrix}
0 \\
0 \\
\vdots \\
0 \\
1 
\end{pmatrix}
\end{equation}
Usually the tensor product symbol is omitted when taking the tensor product of two vectors in Dirac notation. This means we write $ \vert \psi \rangle \vert \phi \rangle$ instead of $ \vert \psi \rangle \otimes \vert \phi \rangle$. We also would like to quickly remind ourselves what a Hilbert space is. Let $\mathcal{H}$ be an inner product space. Endow $\mathcal{H}$ with a norm $\vert \vert x \vert \vert = \sqrt{\langle x,x \rangle}$ and a metric $d(x,y) = \vert \vert x-y \vert \vert$. If every Cauchy sequence in $\mathcal{H}$ converges to an element in $\mathcal{H}$, i.e. $\mathcal{H}$ is complete, then $\mathcal{H}$ is a Hilbert space. Now we can define a \emph{state}.\\
A state is a complex positive semidefinite matrix $\rho$ that satisfies Tr$(\rho)=1$. The trace of a psd. matrix is equal to the sum of its eigenvalues. The spectral theorem tells us that any $n \times n $ matrix can be decomposed as $\rho = \sum_{i=1}^n \lambda_i \vert \psi_i \rangle \langle \psi_i \vert$ with $\lambda_i$ being its eigenvalues and $\vert \psi_i \rangle$ the corresponding eigenvectors. We call a states \emph{pure} if it has rank $1$, i.e.$ \rho = \vert \psi \rangle \langle \psi \vert$ for some complex unit vector $\vert \psi \rangle$. This means every state is a convex combination of pure states. Note that complex unit vectors are often referred to as states even though states are defined as matrices. What is actually meant is the pure state $\vert \psi \rangle \langle \psi \vert$. The name state for these mathematical objects is chosen because with them the possible configurations of a quantum system can be modeled. A quantum system $X$ is said to be \textit{in} state $\rho$ and is associated with a positive integer $n$, referred to as its dimension and a copy of $\mathbb{C}^n$. The states in $\mathbb{C}^{n \times n}$ give the possible configurations of $X$. A quantum system $X$ may consist subsystems $X_1, \dots , X_N$, where each subsystem $X_i$ is a quantum system for itself. And $X$ is then associated with $\mathbb{C}^{n_1}\otimes \dots \otimes \mathbb{C}^{n_N}$ with $n_i$ being the dimensions of the subsystems. A state $\rho$ in which $X$ is then is a matrix of size $n_1\dots n_N$. 



\section{Local and Quantum correlation matrices} 
	\vspace{8pt}

	\begin{dfn}
xxxxx
\end{dfn}




Arnes Teil hier wird noch etwas stehen. 
\section{Grothendieck-Tsirelson Theorem} 
	\begin{lemma}[Grothendieck's identity]\label{lem:G_id}
	Let $x,y\in\mathbb{R}^d$ be unit vectors. Let $r\in\mathbb{R}^d$ be a random unit vector chosen from $O(d)$-invariant probability distribution on the unit sphere. Then
	\begin{enumerate}
		\item[i,] $\mathbb{P}[\sgn(\sclr{x}{r})\neq\sgn(\sclr{y}{r})]=\frac{\arccos(\sclr{x}{y})}{\pi}$
		\item[ii,] $\mathbb{E}[\sgn(\sclr{x}{r})\sgn(\sclr{y}{r})]=\frac{2}{\pi}\arcsin(\sclr{x}{y}).$
	\end{enumerate}
\end{lemma}
\begin{proof}
	For the proof of $i,$ assume that $x$ and $y$ are linearly dependent. Since both, $x$ and $y$, are unit vectors, that is, either $u=v$, then $\arccos(\sclr{x}{y}) = \arccos(1)=0$ or $x=-y$, then $\arccos(\sclr{x}{y}) = \arccos(-1) = \pi$.
	%Bild von arccos
	Conversely assume that $x$ and $y$ are linearly independent, i.\,e. $\operatorname{dim}(\spn\{x,y\})=2$. Now project $r$ orthogonally on the plane spanned by $x$ and $y$. This gives us a vector $s\in \spn\{x,y\}$ with $\sclr{x}{r} = \sclr{x}{s}$ and $\sclr{y}{r} = \sclr{y}{s}$. The unit vector $s/\norm{s}$ is uniformly distributed on the unit circle \textcolor{red}{contained in $\spn\{x,y\}$} by the $O(d)$-invariance of the probability distribution. \\
	
	\textcolor{red}{Noch naeher auf diese Gleichverteilung eingehen?}
	
	\begin{align*}
		\mathbb{P}[\sgn(\sclr{x}{r})\neq\sgn(\sclr{y}{r})]= \mathbb{P}[\sgn(\sclr{x}{s})\neq\sgn(\sclr{y}{s})] 
	\end{align*} 
	
	\noindent\begin{minipage}{\textwidth}
	If $s$ \textcolor{red}{lies on the green part of the unit circle}, the angle between $x$ and $s$ as well as between $y$ and $s$ is smaller than $\pi/2$, hence $\sclr{x}{s}$ and $\sclr{y}{s}$ are positive. Otherwise, if $s$ \textcolor{red}{lies on the}
		\begin{wrapfigure}{r}{0.4\textwidth}
			\vspace{-20pt}
			\begin{center}
				\includegraphics[width=0.38\textwidth]{chapters/fig_unit_circle.pdf}
			\end{center}
			\vspace{-20pt}
		\end{wrapfigure}
		 \textcolor{red}{blue part of the unit circle}, the angle between both $x$ and $s$ as well as $y$ and $s$ is greater than $3\pi/2$, hence $\sclr{x}{s}$ and $\sclr{y}{s}$ are negative.
		
		\hspace{12pt} Now, if we want to calculate the probability that the signs of the two scalar products disagree, we are interested in the undyed segments of the unit circle. Thus, it is sufficient to calculate the periphery of the undyed segments of the circle. In particular on the unit circle, the angle between two vectors equals the periphery of the segment of the circle between those two vectors. 
			
		\hspace{12pt} Because $\gamma$ and $\delta$ are vertical angles they are both equal. Furthermore, $\alpha$ and $\beta$ have to be equal too, since $\gamma$ and $\delta$ are equal and $\alpha+\delta = \beta+\gamma = \pi/2$. With $\alpha = \arccos(\sclr{x}{y}) - \pi/2$ the first part of Lemma \ref{lem:G_id} follows:
	\end{minipage}
	
	\begin{align*}
		\mathbb{P}[\sgn(\sclr{x}{s})\neq\sgn(\sclr{y}{s})]=2\frac{\frac{\pi}{2}+\alpha}{2\pi} = \frac{\arccos(\sclr{x}{y})}{\pi}.
	\end{align*}
	
	\noindent We conclude with the proof of the second part of Lemma \ref{lem:G_id}: 
	\begin{align*}
		&\mathbb{E}[\sgn(\sclr{x}{r}) \sgn(\sclr{y}{r})] \\
		&\qquad= 1\cdot\mathbb{P}[\sgn(\sclr{x}{r}) = \sgn (\sclr{y}{r} )] - 1\cdot \mathbb{P}[\sgn(\sclr{x}{r}) \neq \sgn(\sclr{y}{r})] \\
		&\qquad= 1 - 2\mathbb{P}[\sgn(\sclr{x}{r}) \neq \sgn(\sclr{y}{r})] \\
		&\qquad= 1 - 2 \frac{\arccos(\sclr{x}{y})}{\pi} \\
		&\qquad= \frac{2}{\pi} \arcsin(\sclr{x}{y}),
	\end{align*}
	because $\arcsin (t) = \arccos(t) = \pi/2$.
\end{proof}

\begin{lemma}[Krivine's trick]\label{lem:krivines_trick}
	Let $x_1,\dots,x_m,y_1,\dots,y_n\in S^{m+n-1}$ be given. Furthermore, let $r\in\mathbb{R}^d$ be a random unit vector chosen form the $O(d)$-invariant probability distribution on the unit sphere. Then there are $x_1^\prime,\dots,x_m^\prime, y_1^\prime,\dots,y_n^\prime\in S^{m+n-1}$ so that
	\begin{equation}
		\mathbb{E}[\sgn(\sclr{x_i^\prime}{r})\sgn(\sclr{y_j^\prime}{r})] = \beta \sclr{x_i}{y_j},
		\label{eq:krivines_trick}
	\end{equation}		
	with $\beta = \frac{2}{\pi} \ln (1+\sqrt{2}).$
\end{lemma}

\noindent For the proof of \ref{lem:krivines_trick} we need to use the tensor product. \textcolor{red}{We have already ... the tensor product for complex vector spaces. Now we give an alternative definition for the $\mathbb{R}^n$.} 

The $\mathbb{R}^n$ is an $n$-dimensional Euclidean space with inner product \sclr{\cdot}{\cdot} and orthonormal basis $e_1,\dots,e_n$.
Then the \emph{k-th tensor product of $\mathbb{R}^n$} is denoted by $(\mathbb{R}^n)^{\tensor k}$ and it is a Euclidean  vector space of dimension $n^k$ with orthonormal basis $e_{i_1}\tensor e_{i_2} \tensor \cdots \tensor e_{i_k}$, $i_j\in\{1,\dots,n\}$. In particular
\begin{align}
	\sclr{e_{i_1}\tensor \cdots \tensor e_{i_k}}{e_{j_1}\tensor \cdots \tensor e_{j_k}}
	&= \prod_{l=1}^k \sclr{e_{i_l}}{e_{j_l}}\nonumber\\
	&=\begin{cases}
		1 & , \text{ if } i_l=j_l \text{ for all } l=1,\dots,n,\\
		0 & , \text{ otherwise},
	\end{cases} \label{eq:orthonormtensor}
\end{align}
and for $v\in\mathbb{R}^n$ with $v=v_1e_1+\cdots +v_ne_n$ we define $v^{\tensor k} \in (\mathbb{R}^n)^{\tensor k}$ by 
\begin{equation}
	v^{\tensor k} = (v_1e_1 + \cdots + v_ne_n) \tensor \cdots \tensor (v_1e_1 + \cdots + v_ne_n) = \sum_{i_1,\dots,i_k} v_{i_1}\cdots v_{i_k} e_{i_1}\tensor\cdots\tensor e_{i_k},
\end{equation}
where the last equation follows by the distributive law (identity $ii,$ of the tensor product). 
Thus, for $v,w\in\mathbb{R}^n$ 
\begin{align}
	\sclr{v^{\tensor k}}{w^{\tensor k}}
%	&\overset{\textcolor{white}{\eqref*{eq:orthonormtensor}}}{=} \sum_{i_1,\dots,i_k} v_{i_1}\cdots v_{i_k}\left(e_{i_1}\tensor\cdots\tensor e_{i_k}\right)^\top \sum_{j_1,\dots,j_k} w_{j_1}\cdots w_{j_k}(e_{j_1}\tensor\cdots\tensor e_{j_k}) \nonumber\\
	&\overset{\textcolor{white}{\eqref*{eq:orthonormtensor}}}{=} \sum_{i_1,\dots,i_k} v_{i_1}\cdots v_{i_k} \sum_{j_1,\dots,j_k} w_{j_1}\cdots w_{j_k} \sclr{e_{i_1}\tensor\cdots\tensor e_{i_k}}{e_{j_1}\tensor\cdots\tensor e_{j_k}} \nonumber\\
	&\overset{\eqref{eq:orthonormtensor}}{=} \sum_{i_1,\dots,i_k} v_{i_1}\cdots v_{i_k}w_{i_1}\cdots w_{i_k} \nonumber\\
	&\overset{\textcolor{white}{\eqref*{eq:orthonormtensor}}}{=}(\sum_{i=1}^n v_iw_i)^k = \sclr{v}{w}^k \label{eq:kth_tensor}
\end{align}
\begin{proof}
	Define the function $E: [-1,+1] \to [-1,+1]$ by $E(t)=\frac{2}{\pi}\arcsin(t)$. Due to Grothendieck's identity (Lemma \ref{lem:G_id}):
	\begin{align*}
		E(\sclr{x_i^\prime}{y_j^\prime} ) &= \mathbb{E}[\sgn(\sclr{x_i^\prime}{r})\sgn(\sclr{y_j^\prime}{r})]\\
		&\overset{!}{=}\beta \sclr{x_i}{y_j}.
	\end{align*}
	
	Idea: To find $\beta,x_i^\prime,y_j^\prime$ we invert $E$:
	\[
		\sclr{x_i^\prime}{y_j^\prime} = E^{-1} (\beta \sclr{x_i}{y_j})	
	\]
	with 
	\begin{align*}
		E^{-1}(t) &= \sin(\pi/2 \cdot t) \\
		&= \sum_{k=0}^\infty \underbrace{\frac{(-1)^{2k+1}}{(2k+1)!}\left(\frac{\pi}{2}\right)^{2k+1}}_{g_{2k+1}}  t^{2k+1}
	\end{align*}
	which is valid for all $t\in[-1,+1]$.
	
	Define the infinite-dimensional Hilbert space
	\begin{equation}
		H= \bigoplus_{r=0}^\infty (\mathbb{R}^{m+n})^{\tensor 2k+1}.
	\end{equation}
	\textcolor{red}{Begruenden, dass H gross genug ist?}
	
	Define $\tilde{x}_i, \tilde{y}_j\in H$, $i=1,\dots,m,j=1,\dots,n$ componentwise:
	\begin{align*}
		(\tilde{x}_i)_k &= \sgn(g_{2k+1}) \sqrt{\modul{g_{2k+1}}\beta^{2k+1}}\, x_i^{\tensor 2k+1} \\
		(\tilde{y}_j)_k &= \sqrt{\modul{g_{2k+1}}\beta^{2k+1}} \,y_j^{\tensor 2k+1}
	\end{align*}
	Then 
	\begin{align*}
		\sclr{\tilde{x}_i}{\tilde{y}_j} &\overset{\textcolor{white}{\eqref*{eq:kth_tensor}}}{=} \sum_{k=0}^\infty g_{2k+1} \beta^{2k+1}\sclr{x_i^{\tensor 2k+1}}{y_j^{\tensor 2k+1}} \\
		&\overset{\eqref{eq:kth_tensor}}{=} \sum_{k=0}^\infty g_{2k+1} \beta^{2k+1} \sclr{x_i}{y_j}^{2k+1} \\
		&\overset{\textcolor{white}{\eqref*{eq:kth_tensor}}}{=} E^{-1}(\beta \sclr{x_i}{y_j}).
	\end{align*}
	Hence, $\beta$ is defined by the condition
	\[
		1 = \sclr{\tilde{x}_i}{x_i} = \sclr{\tilde{y}_j}{y_j}
		%= \sum_{k=0}^\infty \modul{g_{2k+1}} \beta^{2k+1} 
		= \sum_{k=0}^\infty \frac{1}{(2k+1)!}\left(\frac{\pi}{2}\right)^{2k+1}\beta^{2k+1}=\sinh(\frac{\pi}{2}\beta)
	\]
	and
	\[
		\beta = \frac{2}{\pi} \arcsinh(1) = \frac{2}{\pi}\ln(1+\sqrt(2)),	
	\]
	since $\arcsinh (t) = \ln(t+\sqrt{t^2+1})$.

	The only thing that is left to prove, is that the solution of the maximization problem yields vectors $x_1^\prime,\dots,x_m^\prime, y_1^\prime,\dots,y_n^\prime\in S^{m+n-1}$, since our vectors $\tilde{x}_1,\dots,\tilde{x}_m,\tilde{y}_1,\dots,\tilde{y}_n$ are infinite-dimensional. The matrix $G$ given by
%	\begin{equation}
%		G=\begin{pmatrix}
%			\sclr{x_1}{y_1} & \sclr{x_1}{y_2} & \cdots & \sclr{x_1}{y_n} \\
%			\sclr{x_2}{y_1} & \sclr{x_2}{y_2} & \cdots & \sclr{x_2}{y_n} \\
%			 \vdots		& \vdots	& \ddots & \vdots\\
%			\sclr{x_m}{y_1} & \sclr{x_m}{y_2} & \cdots & \sclr{x_m}{y_n}		
%		\end{pmatrix}
%	\end{equation}
	\begin{equation}
		G=\begin{pmatrix}
			\sclr{x_1}{x_1} & \cdots & \sclr{x_1}{x_m}& \sclr{x_1}{y_1} & \cdots & \sclr{x_1}{y_n} \\
			 \vdots		& \ddots	& \vdots & \vdots & \ddots & \vdots\\
			 \sclr{x_m}{x_1} & \cdots & \sclr{x_m}{x_m}& \sclr{x_m}{y_1} & \cdots & \sclr{x_m}{y_n} \\
			\sclr{y_1}{x_1} & \cdots & \sclr{y_1}{x_m}& \sclr{y_1}{y_1} & \cdots & \sclr{y_1}{y_n} \\
			 \vdots		& \ddots	& \vdots& \vdots & \ddots & \vdots\\
			 \sclr{y_n}{x_1} & \cdots & \sclr{y_n}{x_m}& \sclr{y_n}{y_1} & \cdots & \sclr{y_n}{y_n} 
		\end{pmatrix}
	\end{equation}
	is a $(m+n)\times (m+n)$ real matrix called \emph{Gram matrix}. Since the $G$ is positive semidefinite symmetric matrix, $G$ can be diagonalized by an orthogonal matrix. This means that there is a decomposition $G=Q\Lambda Q^\top$ with $Q$ a real orthogonal matrix with columns that are the eigenvectors of $G$ and $\Lambda$ a real and diagonal matrix having the eigenvalues of $G$ on the diagonal. Since the eigenvalues of a positive semidefinite matrix are non-negative, $\Lambda=\Lambda^{1/2}\Lambda^{1/2}$. Thus,
	\[
		G=(Q\Lambda^{1/2})(Q\Lambda^{1/2})^\top		
	\]
	and due to the symmetry of $G$ likewise
	\[
		G=(Q\Lambda^{1/2})^\top(Q\Lambda^{1/2}).	
	\]
	$A\coloneqq Q\Lambda^{1/2}$ is a real $(m+n)\times (m+n)$ matrix and its columns are the vectors we were looking for.
\end{proof}
\begin{dfn}
	For $M\in\mathbb{R}^{m\times n}$ define the quadratic program \textcolor{red}{wieso quadratisch?}
	\begin{align}
		\norm{M}_{\infty\to 1} &= \max \left\{ \sum_{i=1}^m \sum_{j=1}^n M_{ij} \xi_i \eta_j : \xi_i^2=1, i=1,\dots,m, \eta_j^2=1, j =1,\dots,n \right\} \\
		&=\max\left\{\trace{\xi^\top M \eta}: \xi\in\{-1,1\}^m,\eta\in\{-1,1\}^n\right\}.
	\end{align}
\end{dfn}
\textcolor{red}{computing NP-hard}
\begin{dfn}
	\textcolor{red}{sdp relaxation einfuehren}
	
	SDP relaxation of $\norm{M}_{\infty\to1}$:
	\begin{align*}
		\operatorname{sdp}_{\infty\to 1} (M) = \max 
		&\sum_{i=1}^m\sum_{j=1}^n M_{ij} \sclr{x_i}{y_j}\\
		&x_i,y_j\in\mathbb{R}^{m+n}\\
		&\norm{x_i}=1, i=1,\dots,m\\
		&\norm{y_j}=1, j=1,\dots,n.
	\end{align*}
\end{dfn}
\begin{theo}[Grothendieck's inequality] \label{theo:G_ineq}
	There exists a constant $K$ such that for all $M\in\mathbb{R}^{m\times n}$:
	\begin{equation}
		\norm{M}_{\infty\to 1} \leq \operatorname{sdp}_{\infty\to 1} (M) \leq K \norm{M}_{\infty\to 1}.
	\end{equation}
\end{theo}
\begin{proof}
	\textcolor{red}{wieso again? wo wurde das vorher schon einmal gemacht?}
	Again: approximation algorithm with randomized rounding
	
	\begin{algorithm}[H]
		\SetAlgoLined
		\caption{Approximation algorithm with randomized rounding for $\norm{M}_{\infty\to 1}$}
	\end{algorithm}
	\begin{itemize}
		\item[1.] Solve $\operatorname{sdp}_{\infty\to 1} (M)$. Let $x_1,\dots,x_m,y_1,\dots,y_n\in S^{m+n-1}$ be the optimal unit vectors
		\item[2.] Apply Krivine's trick (Lemma \ref{lem:krivines_trick}) and use vectors $x_i,y_j$ to create new unit vectors $x_1^\prime,\dots,x_m^\prime, y_1^\prime,\dots,y_n^\prime\in S^{m+n-1}$.
		\item[3.] Choose $r\in S^{m+n-1}$ randomly
		\item[4.] Round: $u_i = \sgn(\sclr{x_i^\prime}{r})$\\
					\textcolor{white}{Round: }$v_j = \sgn(\sclr{y_j^\prime}{r})$
	\end{itemize}
	
	\noindent Expected quality of the outcome:
	\begin{align*}
		\norm{M}_{\infty\to 1} &\geq \mathbb{E}\left[\sum_{i=1}^m\sum_{j=1}^n M_{ij}u_iv_j\right]\\
		&\overset{\textcolor{white}{\eqref*{eq:krivines_trick}}}{=} \sum_{i=1}^m\sum_{j=1}^n M_{ij} \mathbb{E}[\sgn(\sclr{x_i^\prime}{r}) \sgn(\sclr{y_j^\prime}{r})] \\
		&\overset{\textcolor{white}{\eqref*{eq:krivines_trick}}}{=}\sum_{i=1}^m\sum_{j=1}^n M_{ij}\beta \sclr{x_i}{y_j} \\
		&\overset{\eqref{eq:krivines_trick}}{=}\beta \operatorname{sdp}_{\infty\to 1}(M),
	\end{align*}
	where $\beta = \frac{2\ln(1+\sqrt(2)}{\pi}$, thus $K_G\leq \beta^{-1}$.
\end{proof}

%\begin{theo}[Grothendieck-Tsirelson]
%	There exists an absolute constant $K\geq 1$ such that, for any positive integers $m,n$, the following three equivalent conditions hold:
%	\begin{itemize}
%		\item[(1)] We have the inclusion 
%			\begin{equation}
%				\QC_{m,n} \subset K\LC_{m,n}.
%			\end{equation}
%		\item[(2)] For any $M\in\mathbb{R}^{m\times n}$ and for any $\rho,X_i,Y_j$ verifying the conditions of Definition \textcolor{red}{4.2.1} we have
%			\begin{equation}
%				\sum_{i,j} M_{ij} \trace{\rho(X_i\tensor Y_j)}\leq K \max_{\xi\in\{-1,1\}^m,\eta\in\{-1,1\}^n} \trace{\xi^\top M \eta}.
%			\end{equation}
%			\item[(3)] For any $M\in\mathbb{R}^{m\times n}$ and for any (real) Hilbert space vectors $x_i,y_j$ with $\modul{x_i}\leq 1$, $\modul{y_j}\leq 1$ we have
%				\begin{equation}
%					\sum_{i,j} M_{i,j}\sclr{x_i}{y_j} \leq K \max_{\xi\in\{-1,1\}^m,\eta\in\{-1,1\}^n} \trace{\xi^\top M \eta}. \label{eq:G_ineq}
%				\end{equation}
%	\end{itemize}
%\end{theo}
%\begin{proof}
%	Since \eqref{eq:G_ineq} is a direct consequence of Grothendieck's inequality the only thing left to prove is the equivalence between (1)-(3). The equivalence of (3) and (2) (the Tsirelson's bound) is a consequence of \textcolor{red}{the proof of Lemma \textcolor{red}{4.2.2}, where we proved that $\trace{\rho(X_i\tensor Y_j)} = \sclr{x_i}{y_j}$.}
%	\textcolor{red}{Dann brauchen wir hier noch Argumente aus Exercise 11.3.}
%	Finally, the equivalence between (2) and (1) ...
%\end{proof}
\newpage
 
 
\appendix

	Dinge, die definiert werden sollten. 
\begin{enumerate}
	\item Injective tensor product 
	\item norms
	\item Notation, operators of norms 
	\item  perhaps what a state is 
\end{enumerate}


Appendix alles was vorher keinen Platz findet. 	


%\bibliography{literature}						  % Literature
%\bibliographystyle{plain}
\newpage 
\begin{thebibliography}{99999}
	
	% Das Literaturverzeichnis wird mit diesem Befehl in das Inhaltsverzeichnis eingebunden
	%\addcontentsline{toc}{section}{Bibliography}
	
	\bibitem[Au]{Au} G. Aubrun; S. J. Szarek. {\em Alice and Bob Meet Banach: The Interface of Asympototic Geometry Analysis and Quantum Information Theory.} Graduate Studies in Mathematics, American Mathematical Society, 2017.
	\bibitem[Br]{Br}J. Bri\"{e}t. {\em Grothendieck Inequalities, Nonlocal Games and Optimization.} Institute for Logic, Language and Computation, Amsterdam, 2011.
	\bibitem[KNP]{KNP} F. Kirschner; Simon Nietz; Fabian Portner. {\em Seminar Paper Quantum Chromatic Number of a Graph.} University of Cologne, 2017.
	\bibitem[MR]{MR} L. Man$\check{c}$inska; D. E. Roberson. {\em Note on the Correspondence Between Quantum Correlations and the Completely Positive Semidefinite Cone}. Unpublished manuscript, 2014.
	
	
	
\end{thebibliography}
	
\end{document}