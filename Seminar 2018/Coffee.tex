\documentclass[11pt]{amsart}

\usepackage{amssymb}
\usepackage{dutchcal}        %nice Hilbertspace symbols
\usepackage{braket}          %for braket notation
\usepackage[utf8]{inputenc}
\usepackage{bm}              %bold writing
\usepackage{tikz}            %for drawing some nice graphics
\usepackage{tikzsymbols}
\usepackage[version=3]{mhchem}%chemical formulas nicely written
\usepackage{graphicx}        %some graphics stuff, like subfigures
\usepackage{subcaption}
\usepackage{wrapfig}         %figures in text float
\usepackage{algorithm2e}
\usepackage{cite}            %to cite
\usepackage{dsfont}           %for unity matrix
\usepackage{nicefrac}
\usepackage{xcolor}
\usepackage{xspace}
\usepackage{setspace}
\usepackage{tikz}
\usepackage{mathtools}
\usepackage{enumitem}
\usepackage[top=2.5cm, bottom=2.5cm, left=3.5cm, right=3.5cm]{geometry}
\usepackage{lmodern}		% nice schriftart
\usepackage{etoolbox}        %reformat sections etc.
\usepackage{float}									% forcieren von H bei Grafiken
\usepackage[unicode=true,							% Hyperlinks
bookmarks=true,bookmarksnumbered=false,bookmarksopen=false,
breaklinks=false,pdfborder={0 0 0},backref=false,colorlinks=false]
{hyperref}
\DeclareMathAlphabet{\mathpzc}{OT1}{pzc}{m}{it}

\makeatletter
\let\@afterindenttrue\@afterindentfalse
\makeatother
\pagestyle{plain}   %no title at every page
\graphicspath{ {images/} }





%--------------------------only for tikz---------------------------------------------------------
\newcommand{\xdownarrow}[1]{%
	{\left\downarrow\vbox to #1{}\right.\kern-\nulldelimiterspace}
}
\newcommand{\xuparrow}[1]{%
	{\left\uparrow\vbox to #1{}\right.\kern-\nulldelimiterspace}
}

\newcommand{\xswarrow}[1]{%
	{\left\swarrow\vbox to #1{}\right.\kern-\nulldelimiterspace}
}

\usetikzlibrary{shapes.callouts,calc,arrows,decorations.pathmorphing,intersections,
	decorations.pathreplacing
}
\tikzset{
	fatlevel/.style   = { ultra thick, black },
	level/.style   = {thick, black },
	vertex/.style = {very thick, black,opacity=1, shorten <= 0.15cm, shorten >= 0.15cm},
	virtual/.style={thick,densely dashed},
	connect/.style = { dashed, black },
	redtransition/.style = {thick,->,>=stealth',shorten >=1pt, red},
	transition/.style = {thick, black,->,>=stealth',shorten >=1pt},
	trans/.style={thick,<->,>=stealth},
	notice/.style  = { draw, rectangle callout, callout relative pointer={#1} },
	label/.style   = { text width=3cm },
	llabel/.style   = { text width=9.5cm },
	interface/.style={
		% The border decoration is a path replacing decorator. 
		% For the interface style we want to draw the original path.
		% The postaction option is therefore used to ensure that the
		% border decoration is drawn *after* the original path.
		postaction={draw,decorate,decoration={border,angle=135,
				amplitude=0.3cm,segment length=2mm}}},
}
%-----------------------------------------------------------

\definecolor{colororange}{HTML}{E65100} % orange
\definecolor{colordgray}{HTML}{795548} % dark gray for note
\definecolor{colorhgray}{HTML}{212121} % heavy dark gray for normal text
\definecolor{colorgreen}{HTML}{009688} % green
\definecolor{colorlgray}{HTML}{FAFAFA} % background light gray
\definecolor{colorblue}{HTML}{0277BB} % blue

%-----------------------------------------------------------








%------------------------- Theorem enviroments -----------------------------------------
\newtheoremstyle{mytheorem}% ⟨name⟩
{12pt}% ⟨Space above⟩
{12pt}% ⟨Space below⟩
{\itshape}% ⟨Body font⟩
{}% ⟨Indent amount⟩
{\bf}% ⟨Theorem head font⟩
{.}% ⟨Punctuation after theorem head⟩
{.5em}% ⟨Space after theorem head⟩
{}% ⟨Theorem head spec (can be left empty, meaning ‘normal’)⟩
\theoremstyle{mytheorem}
\newtheorem{theo}{Theorem}

\newtheoremstyle{mylemma}% ⟨name⟩
{12pt}% ⟨Space above⟩
{12pt}% ⟨Space below⟩
{\itshape}% ⟨Body font⟩
{}% ⟨Indent amount⟩
{\bf}% ⟨Theorem head font⟩
{.}% ⟨Punctuation after theorem head⟩
{.5em}% ⟨Space after theorem head⟩
{}% ⟨Theorem head spec (can be left empty, meaning ‘normal’)⟩
\theoremstyle{mytheorem}
\newtheorem{lemma}[subsubsection]{Lemma}

\newtheoremstyle{kor}% ⟨name⟩
{12pt}% ⟨Space above⟩
{12pt}% ⟨Space below⟩
{\itshape}% ⟨Body font⟩
{}% ⟨Indent amount⟩
{\bf}% ⟨Theorem head font⟩
{.}% ⟨Punctuation after theorem head⟩
{.5em}% ⟨Space after theorem head⟩
{}% ⟨Theorem head spec (can be left empty, meaning ‘normal’)⟩
\theoremstyle{mytheorem}
\newtheorem{kor}{Corollary}

\newtheoremstyle{prop}% ⟨name⟩
{12pt}% ⟨Space above⟩
{12pt}% ⟨Space below⟩
{\itshape}% ⟨Body font⟩
{}% ⟨Indent amount⟩
{\bf}% ⟨Theorem head font⟩
{.}% ⟨Punctuation after theorem head⟩
{.5em}% ⟨Space after theorem head⟩
{}% ⟨Theorem head spec (can be left empty, meaning ‘normal’)⟩
\theoremstyle{mytheorem}
\newtheorem{prop}{Proposition}

\newtheoremstyle{conj}% ⟨name⟩
{12pt}% ⟨Space above⟩
{12pt}% ⟨Space below⟩
{\itshape}% ⟨Body font⟩
{}% ⟨Indent amount⟩
{\bf}% ⟨Theorem head font⟩
{.}% ⟨Punctuation after theorem head⟩
{.5em}% ⟨Space after theorem head⟩
{}% ⟨Theorem head spec (can be left empty, meaning ‘normal’)⟩
\theoremstyle{mytheorem}
\newtheorem{conj}{Conjecture}

\newtheoremstyle{mydef}% ⟨name⟩
{12pt}% ⟨Space above⟩
{16pt}% ⟨Space below⟩
{}% ⟨Body font⟩
{}% ⟨Indent amount⟩
{\bf}% ⟨Theorem head font⟩
{.}% ⟨Punctuation after theorem head⟩
{.5em}% ⟨Space after theorem head⟩
{}% ⟨Theorem head spec (can be left empty, meaning ‘normal’)⟩
\theoremstyle{mydef}
\newtheorem{dfn}[subsubsection]{Definition}


\newtheoremstyle{myremark}% ⟨name⟩
{12pt}% ⟨Space above⟩
{16pt}% ⟨Space below⟩
{\itshape}% ⟨Body font⟩
{}% ⟨Indent amount⟩
{\bf}% ⟨Theorem head font⟩
{}% ⟨Punctuation after theorem head⟩
{.5em}% ⟨Space after theorem head⟩
{}% ⟨Theorem head spec (can be left empty, meaning ‘normal’)⟩
\theoremstyle{myremark}
\newtheorem{rmk}{Remark}
\newtheorem*{rmk*}{Remark}
\newtheorem*{qst*}{Question}


\newtheoremstyle{myex}% ⟨name⟩
{0pt}% ⟨Space above⟩
{16pt}% ⟨Space below⟩
{\itshape}% ⟨Body font⟩
{}% ⟨Indent amount⟩
{\bf}% ⟨Theorem head font⟩
{}% ⟨Punctuation after theorem head⟩
{.5em}% ⟨Space after theorem head⟩
{}% ⟨Theorem head spec (can be left empty, meaning ‘normal’)⟩
\theoremstyle{myex}
\newtheorem{ex}{Example}
\newtheorem*{ex*}{Example}





%-------------------------- reformat sections and subsections
\patchcmd{\subsection}{\bfseries}{\itshape\bfseries\centering}{}{}
\patchcmd{\subsection}{-.5em}{.5em}{}{}
\patchcmd{\section}{\normalfont}{\normalfont\bfseries\Large}{}{}
\patchcmd{\section}{-.5em}{.5em}{}{}
\patchcmd{\subsubsection}{\normalfont}{\itshape\bfseries\centering}{}{}
\patchcmd{\subsubsection}{-.5em}{.5em}{}{}


%---------------- space between page numbering
	\addtolength{\textheight}{-\baselineskip}
	\addtolength{\footskip}{\baselineskip}


%---------------- section spacing and table of contents formatting
	\setcounter{tocdepth}{3}% to get subsubsections in toc
	\let\oldtocsection=\tocsection
	\let\oldtocsubsection=\tocsubsection
	\let\oldtocsubsubsection=\tocsubsubsection
	\renewcommand{\tocsection}[2]{\hspace{0em}\oldtocsection{#1}{#2}}
	\renewcommand{\tocsubsection}[2]{\hspace{2em}\oldtocsubsection{#1}{#2}}
	\renewcommand{\tocsubsubsection}[2]{\hspace{4em}\oldtocsubsubsection{#1}{#2}}





%---------------------------- definitions
\newcommand{\el}{ {\scriptstyle{\in}}\; }
\newcommand{\sumel}{ {\scriptscriptstyle{\in}} }
\newcommand{\Hilb}{ {\mathcal{H}} }
\newcommand{\dualHilb}{ {\mathcal{H}^*} }
\newcommand{\Haml}{ {\hat{H}} }
\newcommand{\modul}[1]{{\left|{#1}\right|}}
\newcommand{\norm}[1]{\ensuremath{ \Vert {#1} \Vert}}
\newcommand{\trace}[1]{\ensuremath{\text{Tr} \, {#1} }}
\newcommand{\QC}{\ensuremath{\textup{QC}}}
\newcommand{\LC}{\ensuremath{\textup{LC}}}
\newcommand{\sclr}[2]{\ensuremath{\langle{#1},{#2}\rangle}}
\newcommand{\tensor}{\ensuremath{\otimes}}

\makeatletter
\def\eqref{\@ifstar\@eqref\@@eqref}
\def\@eqref#1{\textup{\tagform@{\ref*{#1}}}}
\def\@@eqref#1{\textup{\tagform@{\ref{#1}}}}
\makeatother

\DeclareMathOperator*{\esssup}{ess\,sup} % essentiellen Supremums
\DeclareMathOperator{\spn}{span} % Span
\DeclareMathOperator{\supp}{supp} % Träger
\DeclareMathOperator{\proj}{proj} % Träger
\DeclareMathOperator{\tr}{Tr} 
\DeclareMathOperator{\sgn}{sign}
\DeclareMathOperator{\arcsinh}{arcsinh}
 %----- put in Preamble


\title{Seminar paper \\ Ungleichungen und ähnlich verwirrende Konzepte}
\author{Maxi Brandstetter, Arne Heimendahl, Felix Kirschner}

\begin{document}
\pagenumbering{roman}	



\begin{abstract}
Dieses Paper ist eine Ausarbeitung unseres Vortrags im Seminar "Introduction to Quantum Information and Quantum Computing", das zwischen dem 19.9. und 21.9.2018 in Köln stattgefunden hat. Es wird eine kleine Einführung in die mathematischen Methoden gegeben, mit denen die Welt der Quantenfunktion beschrieben werden kann, woraufhin wir "Nonlocal Games" einführen, die Brücke zur (semidefiniten) Optimierung schlagen und hoffentlich noch genug Zeit für die Grothendieck Ungleichungen haben. Die Grothendieck Ungleichungen finden erstaunlicherweise in einer Vielzahl an mathematischen Teilgebieten Verwendung.
\end{abstract}


\maketitle	
\tableofcontents
	
	
\newpage	
\pagenumbering{arabic}



\section{Einleitung}
	\vspace{8pt}
	Einleitung. Es geht um Grothendieck Ungleichung und so weiter. 


%---------------------------------------------------------------------------
\let \oldsection \section
\renewcommand{\section}{\vspace{15pt plus 8pt}\oldsection}
\let \oldsubsection \subsection
\renewcommand{\subsection}{\vspace{10pt plus 3pt}\oldsubsection}
%---------------------------------------------------------------------------



\newpage
\section{Quantum Grundlagen} %\label{sec_coloringgame}
	\vspace{8pt}
	\subsection{Basic definitions}
In order to make sure everyone is on the same page we will have the following introduction where the necessary groundwork is done. The aim is to call a few basic definitions back to our memory so one can fluently read through this paper. \\
A complex matrix $A \in \mathbb{C}^{n \times n}$ is called Hermitian if $A^*=A$, where $A^*$ denotes the conjugate transpose of $A$. A complex Hermitian matrix $A$ is called positve semidefinite (abbreviated psd.) if one of the following holds: 
\begin{enumerate}
\item[i,] The matrix has only real non-negative eigenvalues 
\item[ii,] There exist complex $n-$dimensional vectors $z_1, \dots, z_n$ s.t. $A_{i,j} = \langle z_i, z_j \rangle = \sum_{k=1}^n \overline{z_{i_k}}z_{j_k}$
\item[iii,] For every $z \in \mathbb{C}^n$ we have $z^*Az\ge 0$
\item[iv,] There exists a complex matrix $B$ s.t. $A=B^*B$
\item[v,] $\trace{A,B} \ge 0 $ for all positive semidefinite operators $ B $ defined in the same space. 
\end{enumerate}
It can be shown that $i,$ - $iv,$ are in fact equivalent. The set of positive semidefinite matrices is a cone, meaning for two psd. $n \times n$ matrices $A,B$ and $\alpha, \beta \in \mathbb{R}_+$ we have that $\alpha A + \beta B $ is also positive semidefinite. The set of real $n \times n$ matrices is denoted by $\mathcal{S}_n^+$.\\
\subsection{Tensor products and Dirac notation}
In case someone is not familiar with the \textit{tensor product} a short introduction with an example or two is given here: 
Let $\mathcal{X} = \mathbb{C}^{n_1 \times m_1}$ and $\mathcal{Y}= \mathbb{C}^{n_2 \times m_2}$. Then the tensor product of the vector spaces $\mathcal{X}$ and $\mathcal{Y}$ is defined as $\mathcal{X} \otimes \mathcal{Y} = \mathbb{C}^{n_1n_2 \times m_1m_2}$. The tensor product of complex matrices can be obtained as follows: Index the rows and columns of a matrix by $\mathcal{R}$ and $\mathcal{C}$ and think of the matrix as a map from $\mathcal{R} \times \mathcal{C} \rightarrow \mathbb{C}$. For two complex matrices $A: \mathcal{R_1} \times \mathcal{C_1} \rightarrow \mathbb{C}$ and $B: \mathcal{R_2} \times \mathcal{C_2} \rightarrow \mathbb{C}$ their tensor product is the matrix $A \otimes B : (\mathcal{R_1} \times \mathcal{R_2}) \times (\mathcal{C_1} \times \mathcal{C_2})\rightarrow \mathbb{C}$ defined by $(A \otimes B)( (r_1,r_2),(c_1,c_2))= A(r_1,c_1)B(r_2,c_2)$. Considering a lexicographic order understand tensor products in the following way: $A \otimes B = \begin{pmatrix}
a_{11}B & \dots & a_{1n}B \\
\vdots && \vdots \\
a_{n1}B & \dots & a_{nn}B
\end{pmatrix}$, which is the \textit{Kronecker product} and coherent with the definition. For two complex vectors $v_1,v_2$ when we talk about their tensor products we will mean $v_1 \otimes v_2 = ( v_{1_1}v_2, v_{1_2}v_2, \dots , v_{1_n}v_2)^\top$ from which we can deduce that $\langle x_1 \otimes x_2 , y_1 \otimes y_2 \rangle = \langle x_1, x_2 \rangle \langle y_1 , y_2 \rangle$. Also for any matrices $A,B,C,D$ (assuming fitting dimensions) we have the following identities: 
\begin{enumerate}
\item[i,] $(A \otimes B ) \otimes C = A \otimes (B \otimes C) $
\item[ii,] $A \otimes (B + C) = A \otimes B + A \otimes C $
\item[iii,] $(A \otimes B)(C \otimes D) = (AC) \otimes (BD)$
\end{enumerate}
Also throughout this paper we will stick to the \textit{Dirac notation}, which is the standard notation for describing quantum states. In Dirac notation $\vert \psi \rangle$ refers to a vector in $\mathbb{C}^n$. The conjugate transpose of this vector is written $ \langle \psi \vert$. The non-negative integers, by convention, represent the canonical basis vectors, i.e. 
\begin{equation}
\vert 0 \rangle = \begin{pmatrix}
1 \\
0 \\
0 \\
\vdots \\
0
\end{pmatrix}, \vert 1 \rangle = \begin{pmatrix}
0 \\
1 \\
0 \\
\vdots \\
0 
\end{pmatrix} , \dots , \vert n-1 \rangle = \begin{pmatrix}
0 \\
0 \\
\vdots \\
0 \\
1 
\end{pmatrix}
\end{equation}
Usually the tensor product symbol is omitted when taking the tensor product of two vectors in Dirac notation. This means we write $ \vert \psi \rangle \vert \phi \rangle$ instead of $ \vert \psi \rangle \otimes \vert \phi \rangle$. We also would like to quickly remind ourselves what a Hilbert space is. Let $\mathcal{H}$ be an inner product space. Endow $\mathcal{H}$ with a norm $\vert \vert x \vert \vert = \sqrt{\langle x,x \rangle}$ and a metric $d(x,y) = \vert \vert x-y \vert \vert$. If every Cauchy sequence in $\mathcal{H}$ converges to an element in $\mathcal{H}$, i.e. $\mathcal{H}$ is complete, then $\mathcal{H}$ is a Hilbert space. 
\subsection{States and measurements}
Now we can define a \emph{state}.\\
A state is a complex positive semidefinite matrix $\rho$ that satisfies Tr$(\rho)=1$. The trace of a psd. matrix is equal to the sum of its eigenvalues. The spectral theorem tells us that any $n \times n $ matrix can be decomposed as $\rho = \sum_{i=1}^n \lambda_i \vert \psi_i \rangle \langle \psi_i \vert$ with $\lambda_i$ being its eigenvalues and $\vert \psi_i \rangle$ the corresponding eigenvectors. We call a state \emph{pure} if it has rank $1$, i.e.$ \rho = \vert \psi \rangle \langle \psi \vert$ for some complex unit vector $\vert \psi \rangle$. This means every state is a convex combination of pure states. Note that complex unit vectors are often referred to as states even though states are defined as matrices. What is actually meant is the pure state $\vert \psi \rangle \langle \psi \vert$. The name state for these mathematical objects is chosen because with them the possible configurations of a quantum system can be modeled. A quantum system $X$ is said to be \textit{in} state $\rho$ and is associated with a positive integer $n$, referred to as its dimension and a copy of $\mathbb{C}^n$. The states in $\mathbb{C}^{n \times n}$ give the possible configurations of $X$. A quantum system $X$ may consist subsystems $X_1, \dots , X_N$, where each subsystem $X_i$ is a quantum system for itself. And $X$ is then associated with $\mathbb{C}^{n_1}\otimes \dots \otimes \mathbb{C}^{n_N}$ with $n_i$ being the dimensions of the subsystems. A state $\rho$ in which $X$ is then is a matrix of size $n_1\dots n_N$. \\
What physicists usually do is building some mathematical model that is supposed to describe how the universe behaves and then test this model by performing experiments and comparing the results to what the model predicts. We will now define an experiment, a \emph{measurement} of a quantum state, in a mathematical way. It shall be stated that the measurements we are talking about do not compare to a physical measurement like measuring the temperature, atmospheric pressure or any other continuous physical quantity. The outcome of measuring a temperature in Kelvin may be a real number $T \in [0 , \infty)$ but we are going to assume the measurements we consider only have a finite set of outcomes $\mathcal{A}$. As before we will have an $n$-dimensional quantum system $X$ and a measurement on $X$ in state $\rho$ with outcomes in $\mathcal{A}$ is defined by a set of psd. matrices $\{ F^a \}_{a \in \mathcal{A}} \subseteq \mathbb{C}^{n \times n}$ that sum up to the identity matrix, i.e. $\sum_{a \in \mathcal{A}} F^a = I$. A \emph{projective} measurement is defined by matrices that satisfy $F^aF^b = \delta_{ab}F^a$ for all $a,b \in \mathcal{A}$. The outcome of a measurement is a random variable $\chi$ and its probability distribution is given by $\mathbb{P} \left[ \chi = a \right] = \text{Tr}(\rho F^a)$. In order to be able to define an expected value for an projective measurement it is convenient to define the outcomes in $\mathcal{A}$ as real numbers. In that case we have: 
\begin{equation}
\mathbb{E}\left[ \chi \right] = \sum_{a \in \mathcal{A}} a \text{Tr}(\rho F^a) = \text{Tr} ( \rho ( \sum_{a \in \mathcal{A}} a F^a))
\end{equation}
The sum of the matrices times their outcome value is called an \emph{observable} associated to a projective measurement. A very simple case of this would be a $\{ -1, 1 \}$-valued observable, where $\{ -1, 1 \}$ is the set of outcomes. 
Such an observable is defined as $\sum_{a \in \{ -1, 1 \} } a F^a = (-1)F^- + (1)F^+ = F^+ - F^-$. Since we are considering projective measurements, squaring the difference yields
\begin{equation}
(F^+-F^-)^2 = \underbrace{F^{+^2}}_{= F^+}- \underbrace{F^+F^-}_{\delta_{+-}=0} + \underbrace{F^{-^2}}_{F^-} = F^+ + F^- = I
\end{equation}
i.e.  a $\{ -1, 1 \}$-valued observable is both unitary and Hermitian.\\
Now let us take a quantum system $X$ consisting of subsystems $X_1, \dots, X_N$ an distribute the subsystems among N parties. If $X$ is in state $\rho$ we say that state $\rho$ is shared by the parties. The parties may have an arbitrary distance to each other, i.e. may be located anywhere in the universe. Every party can perform a measurement on their subsystem. This means there are $N$ sets of psd. matrices $\{F^{a_1}\}_{a_1 \in \mathcal{A}_1} \in \mathbb{C}^{n_1 \times n_1}, \dots, \{F^{a_N}\}_{a_N \in \mathcal{A}_N} \in \mathbb{C}^{n_N \times n_N}$ and the joint probability distribution of the $N$ measurement outcomes $\chi_1 , \dots , \chi_N$ is 
\begin{equation}
\mathbb{P}\left[ \chi_1 = a_1, \chi_2 = a_2, \dots , \chi_N = a_N \right] = \text{Tr}(\rho F_1^{a_1} \otimes \dots \otimes F_N^{a_N}) 
\end{equation}
As we defined earlier a pure state $\rho$ has rank $1$ and in the case that the system $X$ we have consists of subsystems there is a $\vert \psi \rangle \in \mathbb{C}^{n_1}\otimes \dots \otimes \mathbb{C}^{n_N}$ such that $\rho = \vert \psi \rangle \langle \psi \vert$. The state is called \emph{product state} if it is of the form $\vert \psi \rangle = \vert \psi_1 \rangle \vert \psi_2 \rangle \dots \vert \psi_N \rangle$. A state that is not a product state is said to be entangled. A mixed state, i.e. a state with rank greater than 1 is said to be separable if it is a convex combination of pure states. The interesting thing and what makes quantum mechanics interesting is that entangled states can give correlated measurement outcomes. What makes this especially mind-boggling is the fact that the parties can be located anywhere in the universe. This means that the information of a measurement can travel at an instant.\\
In the following example we would like to show that if two players, call them Alice and Bob, share a product state, the result is in fact a product distribution, i.e. the measurement outcome do not correlate. 
So, let $\vert \psi \rangle = \vert \psi_A \rangle \vert \psi_B \rangle$ and let Alice perform a measurement $\{F^a \}_{a \in \mathcal{A}}$ on her $\vert \psi_A \rangle$ and let Bob perform a measurement $\{ G^b \}_{b \in \mathcal{B}}$ on his $\vert \psi_B \rangle$. 
The probability of Alice getting measurement outcome $\chi_A = a$ and Bob getting $\chi_B = b$ is equal to: 
\begin{flalign*}
Tr(\vert \psi \rangle \langle \psi \vert F^a \otimes G^b ) & = \langle \psi \vert F^a \otimes G^b \vert \psi \rangle \\
& = (\langle \psi_A \vert \otimes \langle \psi_B \vert) (F^a \otimes G^b )(\vert \psi_A \rangle  \otimes \vert \psi_B \rangle )\\
& = ((\langle \psi_A \vert F^a)\otimes (\langle \psi_B \vert  G^b))( \vert \psi_A \rangle \otimes \vert \psi_B \rangle) \\
& = \langle \psi_A \vert F^a \vert \psi_A \rangle \otimes  \langle \psi_B \vert G^b \vert \psi_B \rangle \\
&= \langle \psi_A \vert F^a \vert \psi_A \rangle  \langle \psi_B \vert G^b \vert \psi_B \rangle
\end{flalign*}
Where the third and fourth equality follow from the fact that $(A\otimes B) ( C \otimes D) = (AC) \otimes (BD)$ and that the tensor product of two real numbers is equal to their ordinary product. 
The result is just the product of the probability of Alice measuring $a$ and Bob measuring $b$, as desired. 



\section{Local and Quantum correlation matrices} %\label{sec_mathreform}
	\vspace{8pt}

	\subsection{Local Correlation matrices}
So far, we have been dealing with quite specific strategies. The idea is to generalize the concept of strategies into mathematical objects. Suppose the referee sends an element $ s \in \mathcal{S} $ to Alice, respectively $ t \in\mathcal{T} $ to Bob. Suppose Alice and Bob answer according to a deterministic strategy. We will interpret their answers as vectors $ a \in [-1,1]^{\mathcal{S}}, b \in [-1,1]^{\mathcal{T}} $.  Their common answer is the product $ a_sb_t $. So, their strategy can be uniquely described by a  matrix $ ab^\top $.  Instead of playing a deterministic strategy they could answer in a probabilistic way, meaning that given the input $ s \in \mathcal{S} $, respectively $ t \in \mathcal{T} $ their answers are determined by random variables $ X_s $ and $ Y_t $. Then, their expected common answer is $ \mathbb{E}[X_sY_t] $. In the following definition we define the set of all such matrices encoding the common strategy of Alice and Bob.

\begin{dfn}
	Let $ (X_i)_{1 \le i \le m } $ and $ (Y_j)_{1 \le j \le n} $ be families of random variables on a common probability space such that $ \vert X_i \vert, \vert Y_j \vert \le 1 $ almost surely. Then $ A=(a_{ij}) $ is the corresponding {\itshape classical (or local) correlation matrix} if 
	\begin{align*}
		a_{ij} = \mathbb{E}[X_iY_j]
	\end{align*}
	for all $ 1 \le i \le m, 1 \le j \le n $.
\end{dfn}
As we will see in the sequel, the set of $ m \times n $ correlation matrices is a polytope, denoted by $ \LC_{m,n} $.

As we will see in the following two lemmata, both sets can be described in a more useful way. 

\begin{lemma}\label{LemLC}
	An alternative description of $ \LC_{m,n} $ is given by 
	\begin{align}\label{EqLC}
		\LC_{m,n} = \textup{conv} \{  \xi\eta^T \, | \, \xi \in \{-1,1\}^m, \eta \in \{-1,1\}^n     \}.
	\end{align}
\end{lemma}
What does the lemma tell us in addition to a simples description? It states that the matrices encoding all possible strategies are the convex hull of the matrices defined by deterministic ones. We can interpret this as follows: no matter which probabilistic strategy Alice and Bob play there will be a deterministic strategy that is at least as good as theirs. 
\begin{proof}
	Let us denote the right hand side of \ref{EqLC} by $ M $ and let $ \xi\eta^T \in M $ with $ \xi \in \{-1,1\}^m, \eta \in \{-1,1\}^n $. Clearly $ \xi_i, \eta_j \in \{-1,1\} $ define constant $ \mathbb{R} $-valued random variables satisfying $ \vert \xi_i \vert, \, \vert \eta_j \vert \le 1 $. Hence, it suffices to show that $ \LC_{m,n} $ is convex since it contains the vertices of $ M $. Therefore, consider two classical correlation matrices $ a_{ij}^{(k)} = \mathbb{E}[X_i^{(k)}Y_{j}^{(k)}] $ for $ k \in \{0,1\} $ which are defined on a common probability space such that $ \modul{X_i}^{(k)},\modul{Y}_j^{(k)} \le 1 $ We have to show that there exists random variables $ (X_i),(Y_j) $ with $ \modul{X_i},\modul{Y_j} \le 1 $ almost surely such that
	\begin{align}
		\beta a_{ij}^{(0)}+ (1-\beta)a_{ij}^{(1)} = \mathbb{E}[X_iY_j]
	\end{align}
	for all $ \beta \in [0,1] $.
	Let $ \alpha $ be a Bernoulli random variable, i.e. $ \mathbb{P}(\alpha = 0) = \beta $, $ \mathbb{P}(\alpha = 1) = 1 - \beta$ and set $ X_i = X_i^{(\alpha)}, Y_j = Y_j^{(\alpha)} $.
	Clearly, it holds $ \modul{X_i}, \modul{Y_j} \le 1 $ almost surely. 
	Then 
	\begin{align*}
		\mathbb{E}[X_iY_j] &= \mathbb{E}[X_i^{(\alpha)}Y_j^{(\alpha)}  \mathds{1}_{ \{\alpha = 0\}}] + \mathbb{E}[X_i^{(\alpha)}Y_j^{(1)}]\mathds{1}_{\{\alpha = 1\}}] \\
		&= \beta \mathbb{E}[X_i^{(0)}Y_j^{(0)} ] + (1-\beta) \mathbb{E}[X_i^{(1)}Y_j^{(1)}],
	\end{align*} 
	which proofs that $ \LC_{m,n} $ is convex.
	
	
	
	For the other inclusion, let $ (a_{ij}) \in \LC_{m,n} $, i.e. $ a_{ij} = \mathbb{E}[X_iY_j] $ for $ \mathbb{R} $-valued random variables $ (X_i),(Y_j) $, defined on a common probability space $ \Omega $ with $ \modul{X_i},\modul{Y_j} \le 1 $ almost surely. 
	We will use the characterization of the $ d-$dimensional cube by its vertices, that is $ [-1,1]^d = \textup{conv} \{\xi \, | \, \xi \in \{-1,1 \}^d \}$ (this can be proved by induction). 
	If we define the random variables $ X= (X_1,\hdots,X_m) $ and $ Y= (Y_1,\hdots,Y_n) $ they satisfy $ X \in [-1,1]^m, \, Y \in [-1,1]^n $ almost surely. Using the characterization of the hypercube we can define random variables $ \lambda_{\xi}^{(X)}: \Omega^m \to [0,1] $ such that 
	\begin{align*}
		X(\omega) = \sum_{\xi \in \{-1,1\}^m}\lambda_{\xi}^{(X)}(\omega)\xi
	\end{align*} 
	almost surely 
	and $ \sum_{\xi \in \{-1,1\}^m}\lambda_{\xi}^{(X)}(\omega) = 1  $ for all $ \omega \in \Omega $. We can deduce that 
	$X_i =  \sum_{\xi \in \{-1,1\}^m}\lambda_{\xi}^{(X)}\xi_i $ almost surely.
	If we proceed analogously for $ Y $ we obtain
	\begin{align*}
		a_{ij} = \mathbb{E}[X_iY_j] &= \mathbb{E} \big [  (\sum_{\xi \in \{-1,1\}^m}\lambda_{\xi}^{(X)}\xi_i ) (\sum_{\eta \in \{-1,1\}^n}\lambda_{\eta}^{(Y)}\eta_j ) \big ]   \\
		&= \sum_{\xi \in \{-1,1\}^m, \eta \in \{-1,1\}^n} \mathbb{E}\big [\lambda_{\xi}^{(X)}\lambda_{\eta}^{(Y)} \big ] \xi_i \eta_j  \\
		&= \big (\sum_{\xi \in \{-1,1\}^m, \eta \in \{-1,1\}^n}\mathbb{E} [\lambda_{\xi}^{(X)} ]\mathbb{E}[\lambda_{\eta}^{(Y)}]\big )\xi_i\eta_j
	\end{align*}
	where we used that $ \lambda_{\xi}^{(X)} $ and $ \lambda_{\eta}^{(Y)} $ are independent.
	Since $
		\sum_{\xi \in \{-1,1\}^m, \eta \in \{-1,1\}^n}\mathbb{E} [\lambda_{\xi}^{(X)} ]\mathbb{E}[\lambda_{\eta}^{(Y)}] = 1 $ it follows that $ (a_{ij}) \in  \textup{conv} \{  \xi\eta^T \, | \, \xi \in \{-1,1\}^m, \eta \in \{-1,1\}^n     \}$
 which finishes the proof.
\end{proof}
Now we are able to count the vertices of $ \LC_{m,n} $. Observing that $ \xi \eta^T = \tilde{\xi} \tilde{\eta}^T $ if and only if $ \xi = \tilde{\xi} $ and $ \eta = \tilde{\eta} $ or $ \xi = -\tilde{\xi} $ and $ \eta = -\tilde{\eta} $ it follows that we have $ 2^{n+m}/2 = 2^{n+m-1} $ different matrices $ \xi \eta $, hence $ \LC_{m,n} $ has $ 2^{n+m-1} $ vertices. To analyze the facial structure of $ \LC_{m,n} $ is rather complicated. 
However, we will do it later on for $ n=m=2 $ and compare it to $ \QC_{m,n} $.

\subsection{Quantum correlation matrices}
Before we introduce the quantum correlation matrices ww will shortly go back to the framework of nonlocal games. As before, Alice and Bob share a state $ \rho $ and get inputs $ s \in \mathcal{S}, t \in \mathcal{T} $ and perform measurements $ \{ F_s^{\xi} \}_{\xi = \pm 1} $, respectively $ \{ G_t^{\eta} \}_{\eta = \pm 1} $.
As we have seen before the probability that they response is $ (\xi,\eta) $ for inputs $ (s,t) $ is given by 
$ a_{st} = \trace{(\rho F_s^\xi \otimes G_t^\eta )} $. Again, we can encode these information in a matrix $ A=(a_{st}) $. This motivates the following (slightly more general) definition. 
\begin{dfn}
	Let $ (X_i)_{1 \le i \le m } $ and $ (Y_j)_{1 \le j \le n} $ be self-adjoint operators on $ \mathbb{C}^{d_1} $, respectively $ \mathbb{C}^{d_2} $ for some positive integers $ d_1,d_2 $, satisfying $ \norm{X_i}_{\infty}, \norm{Y_j}_{\infty} \le 1 $. $ A = (a_{ij}) $ is called {\itshape quantum correlation matrix} if there exists a state \textbf{Introduce a symbol zo define operators form one space to another} $ \rho \in D(\mathbb{C}^{d_1} \otimes \mathbb{C}^{d_2})$ such that 
	\begin{align}\label{QCaij}
	a_{ij} = \trace{\rho (X_i \otimes Y_j)}.
	\end{align}
\end{dfn}
We will write $ \QC_{m,n} $ for the set of all $ m \times n $ quantum correlation matrices.

With regard to quantum information theory it is interesting to analyze the geometry of $ \LC_{m,n} $ and $ \QC_{m,n} $. 


In the following, we will proof a similar result for $ \QC_{m,n} $ that is: 
\begin{lemma}\label{LemQC}
	\begin{align*}\label{EqQC}
		QC_{m,n} = \{ (\langle x_i,y_j \rangle)_{1 \le 1 \le m, 1 \le j \le n} \,| \, x_i,y_j \in \mathbb{R}^{ \min \{m,n \} }, \vert x_i  \vert \le 1, \vert y_j \vert \le 1  \},
	\end{align*}
	where $ \langle \cdot , \cdot \rangle $ denotes the standard scalar product. 
\end{lemma}
In order to proof this we have to review some definitions and introduce a special class of matrices, namely the {\itshape Pauli matrices}.

For the first inclusion we review the definition of an inner product. The basic idea is to define an inner product via the  definition of the $ a_{ij} $ in \ref{QCaij}.
Let $ V $ and $ W $ be two vector spaces and $ k  $ a field. A {\itshape bilinear form} is a map $ \beta: \, V \times W \to k $ that is linear in both variables, that is 
\begin{enumerate}
	\item $\beta(v_1+v_2,w) = \beta(v_1,w) + \beta(v_2,w)  $
	\item  $\beta(\lambda v,w)= \lambda \beta(v,w) $
	\item $ \beta(v,w_1+w_2) = \beta(v,w_1)+ \beta(v,w_2) $
	\item $ \beta(v,\lambda w) = \lambda\beta(v,w) $
\end{enumerate}
for all $ v,v_1,v_2 \in V, \, w,w_1,w_2 \in W, \, \lambda \in k $. 
If $ V = W $, we call $ \beta $ {\itshape symmetric} if $ \beta(v,w) = \beta(w,v) $, {\itshape positive semidefinite} if 
$ \beta(v,v) \ge 0 $ and {\itshape positive definite} if $ \beta $ is positive semidefinite and $ \beta(v,v)= 0 $ implies that $ v = 0 $. 
If $ \beta: \, V \times V \to k $ is a symmetric positive definite bilinear form it is called an {\itshape inner product} and usually denoted by $ \langle \cdot \, , \, \cdot \rangle $.
Again, we will write $ M $ for the right hand side of equation \ref{EqQC}.
\begin{proof}[Proof of $ \QC_{m,n} \subset M $]
	Let $ (a_{ij}) \in \QC_{m,n} $. Then there is a sate $ \rho $ on a Hilbert space $ \mathcal{H} = \mathbb{C}^{d_1} \otimes\mathbb{C}^{d_2} $ and Hermitian operators $ (X_i)_{1 \ge m}, \, (Y_j)_{1 \ge n} $ on $ \mathbb{C}^{d_1} $, respectively $ \mathbb{C}^{d_2} $ satisfying $ \norm{X_i}_{\infty}, \norm{Y_j}_{\infty} \le 1 $ such that 
	$ a_{ij} = \trace{\rho X_i \otimes Y_j} $.
	We define a positive semidefinite symmetric bilinear form on the space of Hermitian operators on $ \mathcal{H} $ by 
	$ \beta: \textup{B}^{sa}(\mathcal{H}) \times \textup{B}^{sa}(\mathcal{H}) \to \mathbb{R} $ where $ \beta(S,T) =\textup{Re}( \trace{\rho ST}) $.
	We have to check that it indeed satisfies the mentioned properties. 
	Obviously, $ \beta $ is homogeneous in both variables due to the fact that the trace and the real part of a complex number are linear functions and thus homogeneous. We will show additivity for the first variable, the result follows analogously for the second one. It holds
	\begin{align*}
		\beta(S_1+S_2,T) = \textup{Re}(\trace{\rho(S_1+S_2)T}) &= \textup{Re}(\trace{\rho S_1T}) + \textup{Re}(\trace{\rho S_2T}) \\
		&= \beta(S_1,T)+ \beta(S_2+T).
	\end{align*}
	Symmetry follows form 
	\begin{align*}
		\beta(S,T)&= \textup{Re}\trace{\rho ST} =\textup{Re}\trace{(\rho ST)^*}  = \textup{Re \trace{T^*S^* \rho^*}}  \\
		 &=\textup{Re} \trace{\rho^* T^*S^*} = \textup{Re} \trace{\rho T S} = \beta(T,S),
	\end{align*}
	\textbf{Perhaps explanation}
	Moreover, since $ S^*S $ is a positive semidefinite operator for all complex operators $ S $ we obtain $ \beta(S,S) = \textup{Re}\trace{\rho SS} = \textup{Re} \trace{\rho S^*S} \ge 0 $ and $ \beta $ is positive semidefinite. 
	
	Following the steps in appendix \textbf{XXX} we can factorize the kernel and transform $ \beta $ to an inner product on the vector space of self-adjoint operators modulo the kernel of $ \beta $. 
	Equipped with an inner product, we can regard $ \textup{B}^{sa}(\mathcal{H}) $ \textbf{Introduce notation} as a real Euclidean space. 
	We immediately get that $ a_{ij} =\trace{\rho X_i Y_j}=  \beta(X_i \otimes I,I \otimes Y_j) $. In order to show that the norms of our vectors are bounded by one we have to show
	$ \beta(X \otimes I, X \otimes I), \beta(I \otimes Y, I \otimes Y) \le 1$ for all $ X \in \{X_1,\hdots,Y_m \} $, $ Y \in \{Y_1,\hdots,Y_n \} $. 
	Therefore, let $ d = d_1d_2 $ and $ \rho = \vert \phi \rangle \langle \phi \vert $ for $ \vert \phi \rangle = \sum_{i}^{d}\lambda_i \xi_i \otimes \eta_i $, where $ \{ \xi_1,\hdots,\xi_d \} \subset \mathbb{C}^{d_1}$ and $ \{  \eta_1,\hdots,\eta_d\} \subset \mathbb{C}^{d_2}$ are orthonormal sets, i.e. $ \braket{\xi_i \vert \xi_j}, \braket{\eta_i \vert \eta_j} = 0 $ for $ i \neq j $, and $ \sum_{i=1}^d \lambda_i^2 = 1 $. This decomposition is called {\itshape Schmidt decomposition} and a consequence of the singular value decomposition. Writing $ \rho $ in this form we get 
	\begin{align*}
		\beta(X \otimes I, X \otimes I) &= \textup{Re} \trace{\rho X^2 \otimes I}=  \sum_{i=1}^d \lambda_i^2 \trace{(\ket{\xi_i} \bra{\xi_i} \otimes \ket{\eta_i}\bra{\eta_i})(X^2 \otimes I) } \\
		&= \sum_{i=1}^{d}\lambda_i^2 \trace{(\ket{\xi_i} \bra{\xi_i}X^2)}\trace{(\ket{\eta_i}\bra{\eta_i})} =  \sum_{i=1}^{d}\lambda_i^2 \trace{(\ket{\xi_i} \bra{\xi_i}X^2)}.
	\end{align*}
	In order to get the desired result we have to show that $ \trace{(\ket{\xi_i} \bra{\xi_i}X^2)} = \trace{(X^2\ket{\xi_i} \bra{\xi_i})} \le 1 $. 
	Note that $1 \ge \norm{X}_{\infty} := \sup_{\modul{y}\le 1} \modul{Xy} $ implies that $ \modul{X^2\ket{\xi_i}} \le 1 $. So the problem can be reduced to $ \vert \trace{uv^*}\vert^2 \le 1 $ for complex vectors $ u,v $ with $ \modul{u},\, \modul{v} \le 1 $. But this holds since due to the Cauchy-Schwarz inequality 
	$ \vert\trace{uv^*}\vert^2 = \vert\sum u_i \bar{v_i}\vert^2 \le \modul{u}^2\modul{v}^2 \le 1 $.
	
	If we now identify the operators $ X_i \otimes I $ and $  I \otimes Y_j$ with vectors
	$ (x_i) $ and $ (y_j) $ we have found vectors that almost satisfy the desired properties but they do not have the right dimension and we do not consider the standard scalar product yet. Without loss of generality let $ m \le n $. To obtain the required dimension we will project $ (y_j) $ orthogonally onto $ \textup{span}\{ x_1,\hdots,x_m \} $.
	Let $ \{a_1,\hdots,a_r\} $ be an orthonormal basis of $ \textup{span} \{ x_1,\hdots,x_m \} $ with respect to $ \beta $. 
	The orthogonal projection of $ y_j $ is $ \pi(y):= \sum_{i=1}^{r}\beta(a_i,y_j)a_i $ and fulfills
	$ \beta(x_i,y_j) = \beta(x_i,\pi(y_j)) $. 
	Let $ x_i $ and $ \pi(y_j) $ admit the descriptions 
	$ x_i = \sum_{k=1}^{r}\alpha_k^{(i)}a_k$ and $  \pi(y_j) = \sum_{k=1}^r \gamma_k^{(j)} a_k$ for $ \alpha^{(i)}, \gamma^{(j)} \in \mathbb{R}^r $. Then
	\begin{align*}
		a_{ij}= \beta(x,y)= \beta(x,\pi(y)) = \sum_{1 \le k,l \le r} \alpha_k^{(i)} \gamma_l^{(j)} \beta(a_k,a_l) = \sum_{k=1}^{r}\alpha_k^{(i)}\gamma_k^{(j)} = \langle \alpha^{(i)}, \gamma^{(j)} \rangle.
	\end{align*}
	Moreover, since $ \modul{\alpha^{(i)}} = \modul{x_i} \le 1$ and $\modul{ \gamma^{(j)}} \modul{y_j}= \modul{\pi(y_j)} \le 1 $ the vectors $ (\alpha^{(i)}) $ and$ (\gamma^{(j)}) $ have all the properties of the right hand side of \ref{EqQC}.
	Additionally, we also proved that we can take an even lower dimension for our vectors, precisely 
	$ \min \{ \dim( \textup{span} \{ (x_i) \}),$ $ \dim(\textup{span} \{ (y_j) \} )\} $.
\end{proof}

In order to prove the other inclusion we will use the following proposition. 
\begin{prop} \label{PauliProp}
	For all $ n \ge 1 $ there is a subspace of the $ 2^n \times 2^n $ Hermitian matrices where every vector is the multiple of a unitary matrix. 
\end{prop}
	\begin{proof}
		The proof is based on $ n- $fold tensor products of the Pauli matrices which are the three matrices 
		\begin{align*}
		X = \begin{pmatrix}
		0 & 1 \\ 1 & 0
		\end{pmatrix}, \, Y = \begin{pmatrix}
		0 & -i \\ i & 0
		\end{pmatrix}, \, Z = \begin{pmatrix}
		1 & 0 \\ 0 & -1
		\end{pmatrix}
		\end{align*}
		together with the $ 2 \times 2 $ identity matrix .
		They are all trace $ 0 $ unitary Hermitian matrices and anti-commute pairwise. 
		Moreover we define the $ 2^n \times 2^n $ Hermitian matrix 
		\begin{align*}
		\sigma_A^i = I^{\otimes (i-1)} \otimes A \otimes I^{\otimes (n-i)}
		\end{align*}
		for $ A \in \{ X,Y,Z \}$ and where $ I $ is the $ 2 \times 2 $ identity matrix. The Hermitian property follows directly from the observation $ (M \otimes N)^* = M^* \otimes N^* $.
		Note that $ \sigma_A^i $ and $ \sigma_{A^{'}}^j $ anti-commute if $ i = j $ and $ A = A^{'} $ and commute otherwise. We use these operators in order to define 
		\begin{align*}
		U_i = \sigma_X^i \prod_{k = i+1}^{n}\sigma_Y^k,  \\
		U_{i+n} = \sigma_Z^i \prod_{k = i+1}^n \sigma_Y^k
		\end{align*}
		for $ i = 1,\hdots,n $. Note that these operators are also traceless Hermitian matrices and anti-commute for $ i \neq j $:    
		for $ 1 \le i < j  \le n$ we have 
		\begin{align*}
		U_iU_j = (\sigma_X^i \prod_{k = i+1}^{n}\sigma_Y^k) \cdot( \sigma_X^j \prod_{k = j+1}^{n}\sigma_Y^k) &= \sigma_X^i \sigma_Y^{i+1} \cdots \sigma_Y^j ( \sigma_X^j \prod_{k = j+1}^{n}\sigma_Y^k) \sigma_Y^{j+1} \cdots \sigma_Y^n \\
		&= - \sigma_X^i \sigma_Y^{i+1} \cdots \sigma_Y^{j-1} ( \sigma_X^j \prod_{k = j+1}^{n}\sigma_Y^k) \sigma_Y^{j+1} \cdots \sigma_Y^n  \\
		&= - ( \sigma_X^j \prod_{k = j+1}^{n}\sigma_Y^k)(\sigma_X^i \prod_{k = i+1}^{n}\sigma_Y^k)  \\
		&= -U_jU_i
		\end{align*}
		and 
		\begin{align*}
		U_iU_{n+j} =  (\sigma_X^i \prod_{k = i+1}^{n}\sigma_Y^k) \cdot( \sigma_Z^j \prod_{k = j+1}^{n}\sigma_Y^k) &=
		\sigma_X^i \sigma_Y^{i+1} \cdots \sigma_Y^j ( \sigma_Z^j \prod_{k = j+1}^{n}\sigma_Y^k) \sigma_Y^{j+1} \cdots \sigma_Y^n \\
		&= - \sigma_X^i \sigma_Y^{i+1} \cdots \sigma_Y^{j-1} ( \sigma_Z^j \prod_{k = j+1}^{n}\sigma_Y^k) \sigma_Y^{j+1} \cdots \sigma_Y^n  \\
		&= - ( \sigma_Z^j \prod_{k = j+1}^{n}\sigma_Y^k)(\sigma_X^i \prod_{k = i+1}^{n}\sigma_Y^k)  \\
		&= -U_jU_i.
		\end{align*}
		Since $ U_iU_i^* = U_iU_i =  I$ they are also unitary. 
		Moreover, taking the product of two linear combinations $ X = \sum_{i = 1}^{2n}\xi_i U_i$, $ Y = \sum_{i = 1}^{2n} = \eta_iU_i $ we can calculate 
		\begin{align*}
		XY = \sum_{i = 1}^{2n} \xi_i \eta_i I + \sum_{1 \le i,j \le \le 2n}\xi_i\eta_j U_i U_j &= \sum_{i = 1}^{2n} \xi_i \eta_i I + \sum_{1 \le i < j \le \le 2n}\xi_i\eta_jU_iU_j - \sum_{1 \le i < j \le \le 2n}U_iU_j \\
		&=\sum_{i = 1}^{2n} \xi_i \eta_i I \\
		&= \langle \xi, \eta \rangle I.
		\end{align*}
		So, if we set $ Y = X $ we get the desired result by taking the subspace $ \textup{span} \{ U_i \, | \, i = 1,\hdots,2n \} $
		
	\end{proof}

We are now ready to prove the other inclusion of lemma \ref{LemQC}.

\begin{proof}[Proof of $ \LC_{m,n} \supset M $]
	Let $ (x_i)_{1 \le i \le m}, \, (y_j)_{1 \le j \le n} $ be vectors in $\mathbb{R}^{\min \{ m,n \}}$ that satisfy 
	$ \modul{x_i},$$ \modul{y_j} \le 1 $. 
	Using the notation of the previous proposition's proof we set $ X_i = \sum_{k=1}^{\min \{m,n\}} x_i(k)U_i $ and $ Y_j^{T } = \sum_{k=1}^{\min \{m,n\}}y_j(k)U_k $ where the $ U_i $'s are $ d \times d $ matrices with $  d = 2^{\lceil \min \{m,n\}/2 \rceil} $
	Then $ \trace{(X_iY_j^T)} = d\cdot \langle x_i, y_j \rangle  $ and $ \norm{X_i}_{\infty} \le 1 $ since $ X_iX_i^* = \modul{x_i}^2I $ and $ \modul{x_i}^2 \le 1  $. The same holds for $ Y $ since $ Y_j^TY_j^T =   \modul{y_j}^2I = \modul{y_j}I^T = (Y_j^TY_j^T)^T = Y_jY_j  $.
	Let $ \ket{\phi} = \frac{1}{\sqrt{d}}\sum_{i= 1}^{d}\ket{ii} $ and $ \rho = \ket{\phi}\bra{\rho} $. Note that we can write $ \rho $ as
	\begin{align*}
		\rho = \ket{\phi}\bra{\phi} = \frac{1}{d}\sum_{1 \le k,l \le d}\ket{kk}\bra{ll} = \frac{1}{d} \sum_{1 \le k,l \le d}\ket{k}\bra{l} \otimes \ket{k}\bra{l}
	\end{align*}
	where $  (\ket{k}\bra{l})_{kl} = 1 $ and $ (\ket{k}\bra{l})_{ij} = 0 $ for all $ (i,j)\neq (k,l) $.
	
	Then we get 
	\begin{align}
		\trace{(\rho X_i \otimes Y_j)} &=\frac{1}{d} \sum_{1 \le k,l \le d} \trace{ \big (\ket{k}\bra{l}X_i \otimes \ket{k}\bra{l}Y_j \big )} = \frac{1}{d} \sum_{1 \le k,l \le d} \trace{\big (\ket{k}\bra{l}X_i \big )} \trace{\big (\ket{k}\bra{l}Y_j \big )} \\
		&=  \frac{1}{d} \trace{X_iY_j^T} = \langle x_i,y_j \rangle.
	\end{align}
\end{proof}
We can easily see that $ \QC_{m,n} $ is convex. Consider $ (a_{ij}), (\bar{a}_{ij}) \in \QC_{m,n} $ with $ a_{ij} = \langle x_i, y_j \rangle $ and $ \bar{a}_{ij} = \langle \bar{x}_i, \bar{y}_j \rangle $ for $ x_{i},y_j, \bar{x}_i,\bar{y}_j \in \mathbb{R}^{\min \{m,n\}} $ such that $ \modul{x_i},\modul{y_j}, \modul{\bar{x}_i}, \modul{\bar{y}_j} \le 1 $.
For $ \lambda \in [0,1] $ we define vectors $\tilde{x}_i:= (\sqrt{\lambda}x_i,\sqrt{1-\lambda}\bar{x_i}), \, \tilde{y}_j:= (\sqrt{\lambda}y_j, \sqrt{1-\lambda}\bar{y}_j) $ and due to  $ \modul{\tilde{x}_i} \le \lambda \modul{(x_i,0)} + (1-\lambda)\modul{(0,\bar{x}_i)} \le 1 $ they are unit vectors. Moreover, $ \langle \tilde{x}_i, \tilde{y}_j \rangle = \lambda \langle x_i,y_j \rangle + (1-\lambda) \langle \tilde{x_i},\tilde{y}_j \rangle$. If we proceed in the same fashion as in the proof of lemma \ref{LemQC} we obtain vectors 
$ \alpha^{(i)},\gamma^{(j)} $ that satisfy $= \langle \alpha^{i},\gamma^{j} \rangle = \langle \tilde{x} _i, \tilde{y}_j$ and have dimension smaller or equal to $ \min \{m,n\} $.


\subsection{The relations between quantum correlation and local correlation matrices}

Using these both descriptions we can derive some relations between the two sets. 
Let $ \xi\eta^T $ be a vertex of $ \LC_{m,n} $. If we just choose $ x_i = \xi_i \ket{0}$ and $ y_j = \eta_j\ket{0} $ we immediately see that $ \xi_i\eta_j = \langle x_i, y_j \rangle $. Hence, $ \xi\eta^T \in \QC_{m,n} $ and combined with the convexity of $ \QC_{m,n} $ we get $ \LC_{m,n} \subset \QC_{m,n} $.

However, the inclusion is strict in general. Let us consider $ n=m=2 $. 
For $ \LC_{2,2} $ we obtain 
\begin{align*}
	\LC_{2,2}= \textup{conv} \{ \pm \begin{pmatrix}
	1 & 1 \\
	1 & 1
	\end{pmatrix} , \, \pm \begin{pmatrix}
	-1 & -1 \\
	1 & 1
	\end{pmatrix} , \, \pm \begin{pmatrix}
	-1 & 1 \\
	-1 & 1
	\end{pmatrix}, \, \pm \begin{pmatrix}
	-1 & 1 \\
	1 & -1
	\end{pmatrix}  \}.
\end{align*}
We can easily see that $ \sigma(\begin{pmatrix}
-1 & 1 \\ 1 & 1 
\end{pmatrix}) \notin \LC_{2,2} $ for $ \sigma \in \Sigma_4 $ where for $ A = (a_{ij}) $ we define $ \sigma (A)  $ by $ (\sigma( A))_{ij}:= a_{\sigma(1)\sigma(j)} $.
We claim that
\begin{equation}\label{facetsLC}
	\LC_{2,2} = \{ A \in \mathbb{R}^{2 \times 2} \, | \, -1 \le  \trace{(AM)} \le 1 \textup{ for all } M \in \mathcal{K} \},
\end{equation}
where $ \mathcal{K}  = \{ \frac{1}{2}\sigma(\begin{pmatrix}
-1 & 1 \\
1 & 1
\end{pmatrix}), \sigma(\begin{pmatrix}
1 & 0 \\
0 & 0
\end{pmatrix}) \, | \,  \sigma \in \{ \textup{id} (1 \, \, 2), (1 \, \, 3), (1 \, \, 4) \} \} $.
The crucial observation is to note that $ \LC_{2,2} $ is affinely isomorphic to the cross polytope scaled  by two, i.e. 
$ \LC_{2,2} \cong 2\textup{CP}_4 := 2\textup{conv} \{  \pm e_i \, | \, i = 1,\hdots,4 \} $, where $ e_i $ are the vectors of the standard basis of $ \mathbb{R}^4 $. For example, this can be seen by interpreting the vertices of $ \LC_{2,2} $ as elements of $ \mathbb{R}^4 $ and then apply the linear transformation given by the matrix 
\begin{align*}
	\begin{pmatrix}
		-1 & -1 & 1 & 1 \\
		-1 & 1 & -1 & 1 \\
		-1 & 1 & 1 & -1 \\
		1 & 1 & 1 & 1 
	\end{pmatrix}.
\end{align*}
Since the polar dual of the cross polytope is the hypercube, i.e. $ (CP_n)^o = [-1,1]^n $,  the face lattice of $ \textup{CP}_4 $ is isomorphic to the opposite lattice of the hypercube's face lattice which implies that the number of facets of $ \textup{CP}_4 $ coincides with the number of vertices of $ [-1,1]^4 $ which is $ 2^4 $. Due to $  \LC_{2,2} \cong 2\textup{CP}_4$ their face lattices of $ \LC_{2,2} $ and $ \textup{CP}_4 $ are isomorphic, so $ \LC_{2,2} $ has $ 2^4 $ facets as well. 
Since all constraints of the right hand side of \ref{facetsLC} clearly define non-empty proper faces of $ \LC_{2,2} $, it suffices to show that the characterization is a non-redundant hyperplane description of $ \LC_{2,2} $, implying that all constraints define facets of $ \LC_{2,2} $. But this is indeed true since if we omit for example the constraint 
$ 1/2 \trace{(AM)} \le 1 $ for $ M = \{ \begin{pmatrix}
1 & 0 \\ 0 & 0
\end{pmatrix} \} $, respectively $ M = \begin{pmatrix}
-1 & 1 \\ 1 & 1 
\end{pmatrix} $
the matrices $ \begin{pmatrix}
2 & 0 \\ 0 & 0 
\end{pmatrix} $, respectively $ \begin{pmatrix}
-1 & 1 \\ 1 & 1
\end{pmatrix} $ 
satisfy all other constraints. 
Eventually, we have all information to show that $ \LC_{2,2} $ is a proper subset of $ \QC_{2,2} $. 
Assume we want to maximize in the direction of the facet induced by $M = \begin{pmatrix}
1 & 1 \\ 1 & -1 
\end{pmatrix} $. Note that $ M $ coincides with the matrix $ \Sigma_{s,t} = (-1)^{f(s,t)} $ for the CSHS game in section \ref{??} \textbf{Label in CSHS section}. Since $ \max \{  \trace{(AM)}\, | \, A \in \LC_{2,2} \} = 2 $ it suffices to show that there is $ A \in \QC_{m,n} $ achieving a better value. For $ A \in \QC_{2,2} $ we obtain, by lemma \ref{LemQC}, Cauchy-Schwarz and $ \modul{y_i}\le 1 $,
\begin{align*}
	\trace{(AM)}&= \langle x_1,y_1 \rangle + \langle x_1,y_2 \rangle + \langle x_2,y_1 \rangle - \langle x_2,y_2 \rangle   \\
	&= \langle x_1+x_2,y_1 \rangle + \langle x_1-x_2,y_2 \rangle  \le \modul{x_1+x_2}\modul{y_1} + \modul{x_1-x_2}\modul{y_2}  \\
	&\le \modul{x_1+x_2} + \modul{x_1-x_2}.
\end{align*} 
Observing that for $ \modul{x_1},\modul{x_2}\le  1 $
\begin{align*}
	&(\modul{x_1+x_2} + \modul{x_1-x_2})^2 \\
	&= \langle x_1+x_2,x_1+x_2\rangle  +\sqrt{\langle x_1+x_2,x_1+x_2\rangle  \langle x_1-x_2,x_1-x_2\rangle} +  \langle x_1-x_2,x_1-x_2\rangle  \\
	&\le 2\modul{x_1}^2+2\modul{x_2}^2 + \sqrt{\modul{x_1}^4+2\modul{x_1}^2\modul{x_2}^2-4\langle x_1,x_2\rangle^2+\modul{x_2}^4}  \\
	&\le  2\modul{x_1}^2+2\modul{x_2}^2 + \sqrt{4(\modul{x_1}^2+\modul{x_2}^2)^2}  \\
	&= 4(\modul{x_1}^2+\modul{x_2}^2)
\end{align*}
we can give a precise upper bound for $ \trace{(AM)} $ by 
\begin{align*}
	\modul{x_1+x_2} + \modul{x_1-x_2} \le 2\sqrt{\modul{x_1}^2+\modul{x_2}^2} \le 2 \sqrt{2}.
\end{align*}
Thus, we just have to find a matrix that satisfies this bound. A possible choice is induced by the vectors $ x_1 = x_2 = \frac{1}{\sqrt{2}}(1,1) $ and $ y_1 = y_2 =(1,0) $, that is $ A = \frac{1}{\sqrt{2}}\begin{pmatrix}
1 & 1 \\ 1 & 1 
\end{pmatrix} $ yielding the value $ \trace{(AM)} = 2 \sqrt{2} $.
So as we have seen, the inclusion $ \LC_{m,n} \subset \QC_{m,n} $ is strict in general. Elements in $ \QC_{m,n} \setminus \LC_{m,n} $ are called {\itshape non-local}.
Generally, linear functionals $ f: \to \mathbb{R} $ \textbf{Notation} that satisfy $ f(A) \le 1 $ for all $ A \in \LC_{m,n} $ are called {\itshape Bell correlation inequalities}. In the case $ f(A) > 1 $ for some $ A \in \QC_{m,n} $ we talk about {\itshape quantum violations}. 

\section{Grothendieck Inequality} %\label{sec_conicapproach}
	\begin{mylemma}[Grothendieck's identity]
	Let $u,v\in\mathbb{R}^d$ be unit vector. Let $r\in\mathbb{R}^d$ be a random unit vector chosen from $O(d)$-invariant probability distribution on the unit sphere. Then
	\begin{enumerate}
		\item[i,] $\mathbb{P}[\sgn(u^\top r)\neq\sgn(v^\top r)]=\frac{\arccos(u^\top v}{\pi}$
		\item[ii,] $\mathbb{E}[\sgn(u^\top r)\sgn(v^\top r)]=\frac{2}{\pi}\arcsin(u^\top v)$
	\end{enumerate}
\end{mylemma}
 
\section*{Appendix}
	\section{Bilinear forms and inner products}
\subsection{Basic definitions}

Let $ V $ and $ W $ be two vector spaces and $ k  $ a field. A {\itshape bilinear form} is a map $ \beta: \, V \times W \to k $ which is linear in both variables, that is 
\begin{enumerate}
	\item [i)] $\beta(v_1+v_2,w) = \beta(v_1,w) + \beta(v_2,w)  $
	\item  [ii)]$\beta(\lambda v,w)= \lambda \beta(v,w) $
	\item [iii)]$ \beta(v,w_1+w_2) = \beta(v,w_1)+ \beta(v,w_2) $
	\item [iv)]$ \beta(v,\lambda w) = \lambda\beta(v,w) $
\end{enumerate}
for all $ v,v_1,v_2 \in V, \, w,w_1,w_2 \in W, \, \lambda \in k $. 
If $ V = W $, we call $ \beta $ {\itshape symmetric} if $ \beta(v,w) = \beta(w,v) $, {\itshape positive semidefinite} if 
$ \beta(v,v) \ge 0 $ and {\itshape positive definite} if $ \beta $ is positive semidefinite and $ \beta(v,v)= 0 $ implies that $ v = 0 $. 
If $ \beta: \, V \times V \to k $ is a symmetric positive definite bilinear form it is called an {\itshape inner product}.
Note that if $ H $ is a positive semidefinite operator then $ \beta(v,w) = v^TH\bar{w} $ defines an positive semidefinite bilinear form and an inner product if $ H $ is positive definite. 
Conversely, each positive semidefinite bilinear form $ \beta $ can be written as $ \beta(v,w) = v^TH\bar{w} $ for an Hermitian operator $ H $.

\subsection{How to derive an inner product from a symmetric positive semidefinite bilinear form}\label{App1}
Suppose we have a $ k- $vector space $ V $ equipped with symmetric positive semidefinite bilinear form $ \beta: \, V \times V \to k $. We want to derive a vector space $ U $ that is equipped with an inner product which is induced by $ \beta $. The idea is to to consider the quotient space $ U:=V/\ker\beta $ where $ \ker \beta = \{ v \in V \, | \, \beta(v,w)= 0 \text{ for all } w \in V \} $. Note that the Cauchy-Schwartz inequality $ \beta(v,w)^2 \le \beta(v,v)\beta(w,w) $ implies that 
$ \ker \beta = \{ v \in V \, | \, \beta(v,v)= 0 \} $.
We define $ \tilde{\beta}:  \, U \times U \to k $ by $ \tilde{\beta}([v],[w]) = \beta(v,w) $ where $ [v] = v + \ker\beta, \,$$[w]=w+\ker\beta$.

We have to show that $ \tilde{\beta} $ is well-defined. Therefore, let $ [v]=[v^{\prime}] $, so $v^{\prime}-v \in \ker \beta $. 
For an arbitrary $ [w] \in U $ yields 
\begin{align*}
	\beta([v],[w]) = \beta(v,w) = \beta(v,w)+\beta(v^{\prime}-v,w)= \beta(v^{\prime},w) = \tilde{\beta}([v^{\prime}],[w]).
\end{align*}
The symmetry of $ \beta $ combined with the observation above ensures $ \tilde{\beta}([v],[w]) = \tilde{\beta}([v],[w^{\prime}]) $ for $ [w]= [w^{\prime}] $.

Finally, we get the following equivalence relations: 
\begin{align*}
	\tilde{\beta}([v],[v]) = 0 \Leftrightarrow \beta(v,v) = 0 \Leftrightarrow v \in \ker \beta \Leftrightarrow [v]= \ker \beta,
\end{align*}
which implies that $ \tilde{\beta} $ defines an inner product on $ U $. 

We are also able to analyze the structure of $ U $ without big effort. Let $ \{ v_1,\hdots,v_k, \hdots v_n \} $ be a basis for $ V $ such that $ \ker \beta = \textup{span} \{ v_1,\hdots v_k \}$. So, if we take two elements 
$ v = \sum_{i=1}^{k}a_iv_i + \sum_{i=k+1}^{n}a_iv_i$ and $ w = \sum_{i=1}^{k}b_iv_i + \sum_{i=k+1}^{n}b_iv_i $ then 
\begin{align*}
	[v] = [w] \Leftrightarrow v-w \in \textup{span} \{  v_1,...,v_k \} \Leftrightarrow (a_{k+1},\hdots a_n) = (b_{k+1},\hdots b_n).
\end{align*} 
Hence, we can deduce that $ U \cong \textup{span} \{ v_{k+1},...,v_n \} $. More generally, if $ V = V_1 \oplus V_2 $, then $ V/V_1 \cong V_2 $.	


\bibliography{literature}						  % Literature
\bibliographystyle{plain}
	
\end{document}