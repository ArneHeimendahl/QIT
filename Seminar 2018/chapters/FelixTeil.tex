\subsection{Basic definitions}
In order to make sure everyone is on the same page we will have the following introduction where the necessary groundwork is done. The aim is to call a few basic definitions back to our memory so one can fluently read through this paper. \\
A complex matrix $A \in \mathbb{C}^{n \times n}$ is called Hermitian if $A^*=A$, where $A^*$ denotes the conjugate transpose of $A$. A complex Hermitian matrix $A$ is called positve semidefinite (abbreviated psd.) if one of the following holds: 
\begin{enumerate}
\item[i,] The matrix has only real non-negative eigenvalues 
\item[ii,] There exist complex $n-$dimensional vectors $z_1, \dots, z_n$ s.t. $A_{i,j} = \langle z_i, z_j \rangle = \sum_{k=1}^n \overline{z_{i_k}}z_{j_k}$
\item[iii,] For every $z \in \mathbb{C}^n$ we have $z^*Az\ge 0$
\item[iv,] There exists a complex matrix $B$ s.t. $A=B^*B$
\item[v,] $\trace{(A\cdotB)} \ge 0 $ for all positive semidefinite operators $ B $ defined in the same space. 
\end{enumerate}
It can be shown that $i,$ - $iv,$ are in fact equivalent. The set of positive semidefinite matrices is a cone, meaning for two psd. $n \times n$ matrices $A,B$ and $\alpha, \beta \in \mathbb{R}_+$ we have that $\alpha A + \beta B $ is also positive semidefinite. The set of real $n \times n$ matrices is denoted by $\mathcal{S}_n^+$.\\
\subsection{Tensor products and Dirac notation}
In case someone is not familiar with the \textit{tensor product} a short introduction with an example or two is given here: 
Let $\mathcal{X} = \mathbb{C}^{n_1 \times m_1}$ and $\mathcal{Y}= \mathbb{C}^{n_2 \times m_2}$. Then the tensor product of the vector spaces $\mathcal{X}$ and $\mathcal{Y}$ is defined as $\mathcal{X} \otimes \mathcal{Y} = \mathbb{C}^{n_1n_2 \times m_1m_2}$. The tensor product of complex matrices can be obtained as follows: Index the rows and columns of a matrix by $\mathcal{R}$ and $\mathcal{C}$ and think of the matrix as a map from $\mathcal{R} \times \mathcal{C} \rightarrow \mathbb{C}$. For two complex matrices $A: \mathcal{R_1} \times \mathcal{C_1} \rightarrow \mathbb{C}$ and $B: \mathcal{R_2} \times \mathcal{C_2} \rightarrow \mathbb{C}$ their tensor product is the matrix $A \otimes B : (\mathcal{R_1} \times \mathcal{R_2}) \times (\mathcal{C_1} \times \mathcal{C_2})\rightarrow \mathbb{C}$ defined by $(A \otimes B)( (r_1,r_2),(c_1,c_2))= A(r_1,c_1)B(r_2,c_2)$. Considering a lexicographic order understand tensor products in the following way: $A \otimes B = \begin{pmatrix}
a_{11}B & \dots & a_{1n}B \\
\vdots && \vdots \\
a_{n1}B & \dots & a_{nn}B
\end{pmatrix}$, which is the \textit{Kronecker product} and coherent with the definition. For two complex vectors $v_1,v_2$ when we talk about their tensor products we will mean $v_1 \otimes v_2 = ( v_{1_1}v_2, v_{1_2}v_2, \dots , v_{1_n}v_2)^\top$ from which we can deduce that $\langle x_1 \otimes x_2 , y_1 \otimes y_2 \rangle = \langle x_1, x_2 \rangle \langle y_1 , y_2 \rangle$. Also for any matrices $A,B,C,D$ (assuming fitting dimensions) we have the following identities: 
\begin{enumerate}
\item[i,] $(A \otimes B ) \otimes C = A \otimes (B \otimes C) $
\item[ii,] $A \otimes (B + C) = A \otimes B + A \otimes C $
\item[iii,] $(A \otimes B)(C \otimes D) = (AC) \otimes (BD)$
\end{enumerate}
Also throughout this paper we will stick to the \textit{Dirac notation}, which is the standard notation for describing quantum states. In Dirac notation $\vert \psi \rangle$ refers to a vector in $\mathbb{C}^n$. The conjugate transpose of this vector is written $ \langle \psi \vert$. The non-negative integers, by convention, represent the canonical basis vectors, i.e. 
\begin{equation}
\vert 0 \rangle = \begin{pmatrix}
1 \\
0 \\
0 \\
\vdots \\
0
\end{pmatrix}, \vert 1 \rangle = \begin{pmatrix}
0 \\
1 \\
0 \\
\vdots \\
0 
\end{pmatrix} , \dots , \vert n-1 \rangle = \begin{pmatrix}
0 \\
0 \\
\vdots \\
0 \\
1 
\end{pmatrix}
\end{equation}
Usually the tensor product symbol is omitted when taking the tensor product of two vectors in Dirac notation. This means we write $ \vert \psi \rangle \vert \phi \rangle$ instead of $ \vert \psi \rangle \otimes \vert \phi \rangle$. We also would like to quickly remind ourselves what a Hilbert space is. Let $\mathcal{H}$ be an inner product space. Endow $\mathcal{H}$ with a norm $\vert \vert x \vert \vert = \sqrt{\langle x,x \rangle}$ and a metric $d(x,y) = \vert \vert x-y \vert \vert$. If every Cauchy sequence in $\mathcal{H}$ converges to an element in $\mathcal{H}$, i.e. $\mathcal{H}$ is complete, then $\mathcal{H}$ is a Hilbert space. 
\subsection{States and measurements}
Now we can define a \emph{state}.\\
A state is a complex positive semidefinite matrix $\rho$ that satisfies Tr$(\rho)=1$. The trace of a psd. matrix is equal to the sum of its eigenvalues. The spectral theorem tells us that any $n \times n $ matrix can be decomposed as $\rho = \sum_{i=1}^n \lambda_i \vert \psi_i \rangle \langle \psi_i \vert$ with $\lambda_i$ being its eigenvalues and $\vert \psi_i \rangle$ the corresponding eigenvectors. We call a state \emph{pure} if it has rank $1$, i.e.$ \rho = \vert \psi \rangle \langle \psi \vert$ for some complex unit vector $\vert \psi \rangle$. This means every state is a convex combination of pure states. Note that complex unit vectors are often referred to as states even though states are defined as matrices. What is actually meant is the pure state $\vert \psi \rangle \langle \psi \vert$. The name state for these mathematical objects is chosen because with them the possible configurations of a quantum system can be modeled. A quantum system $X$ is said to be \textit{in} state $\rho$ and is associated with a positive integer $n$, referred to as its dimension and a copy of $\mathbb{C}^n$. The states in $\mathbb{C}^{n \times n}$ give the possible configurations of $X$. A quantum system $X$ may consist subsystems $X_1, \dots , X_N$, where each subsystem $X_i$ is a quantum system for itself. And $X$ is then associated with $\mathbb{C}^{n_1}\otimes \dots \otimes \mathbb{C}^{n_N}$ with $n_i$ being the dimensions of the subsystems. A state $\rho$ in which $X$ is then is a matrix of size $n_1\dots n_N$. \\
What physicists usually do is building some mathematical model that is supposed to describe how the universe behaves and then test this model by performing experiments and comparing the results to what the model predicts. We will now define an experiment, a \emph{measurement} of a quantum state, in a mathematical way. It shall be stated that the measurements we are talking about do not compare to a physical measurement like measuring the temperature, atmospheric pressure or any other continuous physical quantity. The outcome of measuring a temperature in Kelvin may be a real number $T \in [0 , \infty)$ but we are going to assume the measurements we consider only have a finite set of outcomes $\mathcal{A}$. As before we will have an $n$-dimensional quantum system $X$ and a measurement on $X$ in state $\rho$ with outcomes in $\mathcal{A}$ is defined by a set of psd. matrices $\{ F^a \}_{a \in \mathcal{A}} \subseteq \mathbb{C}^{n \times n}$ that sum up to the identity matrix, i.e. $\sum_{a \in \mathcal{A}} F^a = I$. A \emph{projective} measurement is defined by matrices that satisfy $F^aF^b = \delta_{ab}F^a$ for all $a,b \in \mathcal{A}$. The outcome of a measurement is a random variable $\chi$ and its probability distribution is given by $\mathbb{P} \left[ \chi = a \right] = \text{Tr}(\rho F^a)$. In order to be able to define an expected value for an projective measurement it is convenient to define the outcomes in $\mathcal{A}$ as real numbers. In that case we have: 
\begin{equation}
\mathbb{E}\left[ \chi \right] = \sum_{a \in \mathcal{A}} a \text{Tr}(\rho F^a) = \text{Tr} ( \rho ( \sum_{a \in \mathcal{A}} a F^a))
\end{equation}
The sum of the matrices times their outcome value is called an \emph{observable} associated to a projective measurement. A very simple case of this would be a $\{ -1, 1 \}$-valued observable, where $\{ -1, 1 \}$ is the set of outcomes. 
Such an observable is defined as $\sum_{a \in \{ -1, 1 \} } a F^a = (-1)F^- + (1)F^+ = F^+ - F^-$. Since we are considering projective measurements, squaring the difference yields
\begin{equation}
(F^+-F^-)^2 = \underbrace{F^{+^2}}_{= F^+}- \underbrace{F^+F^-}_{\delta_{+-}=0} + \underbrace{F^{-^2}}_{F^-} = F^+ + F^- = I
\end{equation}
i.e.  a $\{ -1, 1 \}$-valued observable is both unitary and Hermitian.\\
Now let us take a quantum system $X$ consisting of subsystems $X_1, \dots, X_N$ an distribute the subsystems among N parties. If $X$ is in state $\rho$ we say that state $\rho$ is shared by the parties. The parties may have an arbitrary distance to each other, i.e. may be located anywhere in the universe. Every party can perform a measurement on their subsystem. This means there are $N$ sets of psd. matrices $\{F^{a_1}\}_{a_1 \in \mathcal{A}_1} \in \mathbb{C}^{n_1 \times n_1}, \dots, \{F^{a_N}\}_{a_N \in \mathcal{A}_N} \in \mathbb{C}^{n_N \times n_N}$ and the joint probability distribution of the $N$ measurement outcomes $\chi_1 , \dots , \chi_N$ is 
\begin{equation}
\mathbb{P}\left[ \chi_1 = a_1, \chi_2 = a_2, \dots , \chi_N = a_N \right] = \text{Tr}(\rho F_1^{a_1} \otimes \dots \otimes F_N^{a_N}) 
\end{equation}
As we defined earlier a pure state $\rho$ has rank $1$ and in the case that the system $X$ we have consists of subsystems there is a $\vert \psi \rangle \in \mathbb{C}^{n_1}\otimes \dots \otimes \mathbb{C}^{n_N}$ such that $\rho = \vert \psi \rangle \langle \psi \vert$. The state is called \emph{product state} if it is of the form $\vert \psi \rangle = \vert \psi_1 \rangle \vert \psi_2 \rangle \dots \vert \psi_N \rangle$. A state that is not a product state is said to be entangled. A mixed state, i.e. a state with rank greater than 1 is said to be separable if it is a convex combination of pure states. The interesting thing and what makes quantum mechanics interesting is that entangled states can give correlated measurement outcomes. What makes this especially mind-boggling is the fact that the parties can be located anywhere in the universe. This means that the information of a measurement can travel at an instant.\\
In the following example we would like to show that if two players, call them Alice and Bob, share a product state, the result is in fact a product distribution, i.e. the measurement outcome do not correlate. 
So, let $\vert \psi \rangle = \vert \psi_A \rangle \vert \psi_B \rangle$ and let Alice perform a measurement $\{F^a \}_{a \in \mathcal{A}}$ on her $\vert \psi_A \rangle$ and let Bob perform a measurement $\{ G^b \}_{b \in \mathcal{B}}$ on his $\vert \psi_B \rangle$. 
The probability of Alice getting measurement outcome $\chi_A = a$ and Bob getting $\chi_B = b$ is equal to: 
\begin{flalign*}
Tr(\vert \psi \rangle \langle \psi \vert F^a \otimes G^b ) & = \langle \psi \vert F^a \otimes G^b \vert \psi \rangle \\
& = (\langle \psi_A \vert \otimes \langle \psi_B \vert) (F^a \otimes G^b )(\vert \psi_A \rangle  \otimes \vert \psi_B \rangle )\\
& = ((\langle \psi_A \vert F^a)\otimes (\langle \psi_B \vert  G^b))( \vert \psi_A \rangle \otimes \vert \psi_B \rangle) \\
& = \langle \psi_A \vert F^a \vert \psi_A \rangle \otimes  \langle \psi_B \vert G^b \vert \psi_B \rangle \\
&= \langle \psi_A \vert F^a \vert \psi_A \rangle  \langle \psi_B \vert G^b \vert \psi_B \rangle
\end{flalign*}
Where the third and fourth equality follow from the fact that $(A\otimes B) ( C \otimes D) = (AC) \otimes (BD)$ and that the tensor product of two real numbers is equal to their ordinary product. 
The result is just the product of the probability of Alice measuring $a$ and Bob measuring $b$, as desired. 