\subsection{Basic definitions}
In order to make sure everyone is on the same page we will have the following introduction where the necessary groundwork is done. The aim is to call a few basic definitions back to our memory so one can fluently read through this paper. \\
A complex matrix $A \in \mathbb{C}^{n \times n}$ is called Hermitian if $A^*=A$, where $A^*$ denotes the conjugate transpose of $A$. A complex Hermitian matrix $A$ is called positve semidefinite (abbreviated psd.) if one of the following holds: 
\begin{enumerate}
\item[i,] The matrix has only real non-negative eigenvalues 
\item[ii,] There exist complex $n-$dimensional vectors $z_1, \dots, z_n$ s.t. $A_{i,j} = \langle z_i, z_j \rangle = \sum_{k=1}^n \overline{z_{i_k}}z_{j_k}$
\item[iii,] For every $z \in \mathbb{C}^n$ we have $z^*Az\ge 0$
\item[iv,] There exists a complex matrix $B$ s.t. $A=B^*B$
\item[v,] $\trace{(A\cdot B)} \ge 0 $ for all positive semidefinite operators $ B $ defined in the same space. 
\end{enumerate}
It can be shown that $i,$ - $iv,$ are in fact equivalent. The set of positive semidefinite matrices is a cone, meaning for two psd. $n \times n$ matrices $A,B$ and $\alpha, \beta \in \mathbb{R}_+$ we have that $\alpha A + \beta B $ is also positive semidefinite. The set of real $n \times n$ matrices is denoted by $\mathcal{S}_n^+$.\\
\subsection{Tensor products and Dirac notation}
In case someone is not familiar with the \textit{tensor product} a short introduction with an example or two is given here: 
Let $\mathcal{X} = \mathbb{C}^{n_1 \times m_1}$ and $\mathcal{Y}= \mathbb{C}^{n_2 \times m_2}$. Then the tensor product of the vector spaces $\mathcal{X}$ and $\mathcal{Y}$ is defined as $\mathcal{X} \otimes \mathcal{Y} = \mathbb{C}^{n_1n_2 \times m_1m_2}$. The tensor product of complex matrices can be obtained as follows: Index the rows and columns of a matrix by $\mathcal{R}$ and $\mathcal{C}$ and think of the matrix as a map from $\mathcal{R} \times \mathcal{C} \rightarrow \mathbb{C}$. For two complex matrices $A: \mathcal{R_1} \times \mathcal{C_1} \rightarrow \mathbb{C}$ and $B: \mathcal{R_2} \times \mathcal{C_2} \rightarrow \mathbb{C}$ their tensor product is the matrix $A \otimes B : (\mathcal{R_1} \times \mathcal{R_2}) \times (\mathcal{C_1} \times \mathcal{C_2})\rightarrow \mathbb{C}$ defined by $(A \otimes B)( (r_1,r_2),(c_1,c_2))= A(r_1,c_1)B(r_2,c_2)$. Considering a lexicographic order understand tensor products in the following way: $A \otimes B = \begin{pmatrix}
a_{11}B & \dots & a_{1n}B \\
\vdots && \vdots \\
a_{n1}B & \dots & a_{nn}B
\end{pmatrix}$, which is the \textit{Kronecker product} and coherent with the definition. For two complex vectors $v_1,v_2$ when we talk about their tensor products we will mean $v_1 \otimes v_2 = ( v_{1_1}v_2, v_{1_2}v_2, \dots , v_{1_n}v_2)^\top$ from which we can deduce that $\langle x_1 \otimes x_2 , y_1 \otimes y_2 \rangle = \langle x_1, x_2 \rangle \langle y_1 , y_2 \rangle$. Also for any matrices $A,B,C,D$ (assuming fitting dimensions) we have the following identities: 
\begin{enumerate}
\item[i,] $(A \otimes B ) \otimes C = A \otimes (B \otimes C) $
\item[ii,] $A \otimes (B + C) = A \otimes B + A \otimes C $
\item[iii,] $(A \otimes B)(C \otimes D) = (AC) \otimes (BD)$
\end{enumerate}
Also throughout this paper we will stick to the \textit{Dirac notation}, which is the standard notation for describing quantum states. In Dirac notation $\vert \psi \rangle$ refers to a vector in $\mathbb{C}^n$. The conjugate transpose of this vector is written $ \langle \psi \vert$. The non-negative integers, by convention, represent the canonical basis vectors, i.e. 
\begin{equation}
\vert 0 \rangle = \begin{pmatrix}
1 \\
0 \\
0 \\
\vdots \\
0
\end{pmatrix}, \vert 1 \rangle = \begin{pmatrix}
0 \\
1 \\
0 \\
\vdots \\
0 
\end{pmatrix} , \dots , \vert n-1 \rangle = \begin{pmatrix}
0 \\
0 \\
\vdots \\
0 \\
1 
\end{pmatrix}
\end{equation}
Usually the tensor product symbol is omitted when taking the tensor product of two vectors in Dirac notation. This means we write $ \vert \psi \rangle \vert \phi \rangle$ instead of $ \vert \psi \rangle \otimes \vert \phi \rangle$. We also would like to quickly remind ourselves what a Hilbert space is. Let $\mathcal{H}$ be an inner product space. Endow $\mathcal{H}$ with a norm $\vert \vert x \vert \vert = \sqrt{\langle x,x \rangle}$ and a metric $d(x,y) = \vert \vert x-y \vert \vert$. If every Cauchy sequence in $\mathcal{H}$ converges to an element in $\mathcal{H}$, i.e. $\mathcal{H}$ is complete, then $\mathcal{H}$ is a Hilbert space. 
\subsection{States and measurements}
Now we can define a \emph{state}.\\
A state is a complex positive semidefinite matrix $\rho$ that satisfies Tr$(\rho)=1$. The trace of a psd. matrix is equal to the sum of its eigenvalues. The spectral theorem tells us that any $n \times n $ matrix can be decomposed as $\rho = \sum_{i=1}^n \lambda_i \vert \psi_i \rangle \langle \psi_i \vert$ with $\lambda_i$ being its eigenvalues and $\vert \psi_i \rangle$ the corresponding eigenvectors. We call a state \emph{pure} if it has rank $1$, i.e.$ \rho = \vert \psi \rangle \langle \psi \vert$ for some complex unit vector $\vert \psi \rangle$. This means every state is a convex combination of pure states. Note that complex unit vectors are often referred to as states even though states are defined as matrices. What is actually meant is the pure state $\vert \psi \rangle \langle \psi \vert$. The name state for these mathematical objects is chosen because with them the possible configurations of a quantum system can be modeled. A quantum system $X$ is said to be \textit{in} state $\rho$ and is associated with a positive integer $n$, referred to as its dimension and a copy of $\mathbb{C}^n$. The states in $\mathbb{C}^{n \times n}$ give the possible configurations of $X$. A quantum system $X$ may consist subsystems $X_1, \dots , X_N$, where each subsystem $X_i$ is a quantum system for itself. And $X$ is then associated with $\mathbb{C}^{n_1}\otimes \dots \otimes \mathbb{C}^{n_N}$ with $n_i$ being the dimensions of the subsystems. A state $\rho$ in which $X$ is then is a matrix of size $n_1\dots n_N$. \\
What physicists usually do is building some mathematical model that is supposed to describe how the universe behaves and then test this model by performing experiments and comparing the results to what the model predicts. We will now define an experiment, a \emph{measurement} of a quantum state, in a mathematical way. It shall be stated that the measurements we are talking about do not compare to a physical measurement like measuring the temperature, atmospheric pressure or any other continuous physical quantity. The outcome of measuring a temperature in Kelvin may be a real number $T \in [0 , \infty)$ but we are going to assume the measurements we consider only have a finite set of outcomes $\mathcal{A}$. As before we will have an $n$-dimensional quantum system $X$ and a measurement on $X$ in state $\rho$ with outcomes in $\mathcal{A}$ is defined by a set of psd. matrices $\{ F^a \}_{a \in \mathcal{A}} \subseteq \mathbb{C}^{n \times n}$ that sum up to the identity matrix, i.e. $\sum_{a \in \mathcal{A}} F^a = I$. A \emph{projective} measurement is defined by matrices that satisfy $F^aF^b = \delta_{ab}F^a$ for all $a,b \in \mathcal{A}$. The outcome of a measurement is a random variable $\chi$ and its probability distribution is given by $\mathbb{P} \left[ \chi = a \right] = \text{Tr}(\rho F^a)$. In order to be able to define an expected value for an projective measurement it is convenient to define the outcomes in $\mathcal{A}$ as real numbers. In that case we have: 
\begin{equation}
\mathbb{E}\left[ \chi \right] = \sum_{a \in \mathcal{A}} a \text{Tr}(\rho F^a) = \text{Tr} ( \rho ( \sum_{a \in \mathcal{A}} a F^a))
\end{equation}
The sum of the matrices times their outcome value is called an \emph{observable} associated to a projective measurement. A very simple case of this would be a $\{ -1, 1 \}$-valued observable, where $\{ -1, 1 \}$ is the set of outcomes. 
Such an observable is defined as $\sum_{a \in \{ -1, 1 \} } a F^a = (-1)F^- + (1)F^+ = F^+ - F^-$. Since we are considering projective measurements, squaring the difference yields
\begin{equation}
(F^+-F^-)^2 = \underbrace{F^{+^2}}_{= F^+}- \underbrace{F^+F^-}_{\delta_{+-}=0} + \underbrace{F^{-^2}}_{F^-} = F^+ + F^- = I
\end{equation}
i.e.  a $\{ -1, 1 \}$-valued observable is both unitary and Hermitian.\\
Now let us take a quantum system $X$ consisting of subsystems $X_1, \dots, X_N$ an distribute the subsystems among N parties. If $X$ is in state $\rho$ we say that state $\rho$ is shared by the parties. The parties may have an arbitrary distance to each other, i.e. may be located anywhere in the universe. Every party can perform a measurement on their subsystem. This means there are $N$ sets of psd. matrices $\{F^{a_1}\}_{a_1 \in \mathcal{A}_1} \in \mathbb{C}^{n_1 \times n_1}, \dots, \{F^{a_N}\}_{a_N \in \mathcal{A}_N} \in \mathbb{C}^{n_N \times n_N}$ and the joint probability distribution of the $N$ measurement outcomes $\chi_1 , \dots , \chi_N$ is 
\begin{equation}
\mathbb{P}\left[ \chi_1 = a_1, \chi_2 = a_2, \dots , \chi_N = a_N \right] = \text{Tr}(\rho F_1^{a_1} \otimes \dots \otimes F_N^{a_N}) 
\end{equation}
As we defined earlier a pure state $\rho$ has rank $1$ and in the case that the system $X$ we have consists of subsystems there is a $\vert \psi \rangle \in \mathbb{C}^{n_1}\otimes \dots \otimes \mathbb{C}^{n_N}$ such that $\rho = \vert \psi \rangle \langle \psi \vert$. The state is called \emph{product state} if it is of the form $\vert \psi \rangle = \vert \psi_1 \rangle \vert \psi_2 \rangle \dots \vert \psi_N \rangle$. A state that is not a product state is said to be entangled. A mixed state, i.e. a state with rank greater than 1 is said to be separable if it is a convex combination of pure states. The interesting thing and what makes quantum mechanics interesting is that entangled states can give correlated measurement outcomes. What makes this especially mind-boggling is the fact that the parties can be located anywhere in the universe. This means that the information of a measurement can travel at an instant.\\
In the following example we would like to show that if two players, call them Alice and Bob, share a product state, the result is in fact a product distribution, i.e. the measurement outcome do not correlate. 
So, let $\vert \psi \rangle = \vert \psi_A \rangle \vert \psi_B \rangle$ and let Alice perform a measurement $\{F^a \}_{a \in \mathcal{A}}$ on her $\vert \psi_A \rangle$ and let Bob perform a measurement $\{ G^b \}_{b \in \mathcal{B}}$ on his $\vert \psi_B \rangle$. 
The probability of Alice getting measurement outcome $\chi_A = a$ and Bob getting $\chi_B = b$ is equal to: 
\begin{flalign*}
Tr(\vert \psi \rangle \langle \psi \vert F^a \otimes G^b ) & = \langle \psi \vert F^a \otimes G^b \vert \psi \rangle \\
& = (\langle \psi_A \vert \otimes \langle \psi_B \vert) (F^a \otimes G^b )(\vert \psi_A \rangle  \otimes \vert \psi_B \rangle )\\
& = ((\langle \psi_A \vert F^a)\otimes (\langle \psi_B \vert  G^b))( \vert \psi_A \rangle \otimes \vert \psi_B \rangle) \\
& = \langle \psi_A \vert F^a \vert \psi_A \rangle \otimes  \langle \psi_B \vert G^b \vert \psi_B \rangle \\
&= \langle \psi_A \vert F^a \vert \psi_A \rangle  \langle \psi_B \vert G^b \vert \psi_B \rangle
\end{flalign*}
Where the third and fourth equality follow from the fact that $(A\otimes B) ( C \otimes D) = (AC) \otimes (BD)$ and that the tensor product of two real numbers is equal to their ordinary product. 
The result is just the product of the probability of Alice measuring $a$ and Bob measuring $b$, as desired. 

\section{Nonlocal games}
In this section we will introduce nonlocal games, which are a systematic approach of studying quantum mechanics and its properties and comparing it to classical mechanics. They are called nonlocal because the players are assumed to be very far, like light-years, away from each other. But first things first. In the basic case there are three participants, two players Alice and Bob and a referee. The referee sends a piece of information to Alice and Bob. They may or may not receive the same information. Afterwards both Alice and Bob must, without communicating, send an answer to the referee, who then decides whether they both win or both lose and the game ends. Mathematically speaking this means there are four finite sets $\mathcal{A}, \mathcal{B}, \mathcal{S}, \mathcal{T}$, a joint probability distribution $\pi$ over $\mathcal{S} \times \mathcal{T}$, i.e. $\pi : \mathcal{S} \times \mathcal{T} \rightarrow [0,1]$. The referee sends with probability $\pi(s,t)$ $s \in \mathcal{S}$ to Alice and $t \in \mathcal{T}$ to Bob and they both answer with an element $a \in \mathcal{A}$ and $b \in \mathcal{B}$ respectively. Whether they win or lose is determined by a map $V : \mathcal{A} \times \mathcal{B} \times \mathcal{S} \times \mathcal{T} \rightarrow \{ 0 , 1 \}$. They win if $V(a,b,s,t)=1$ and lose otherwise. All players know $\pi$ and $V$ but they they do not know what element the other player received from the referee. They may agree on a strategy beforehand but they must not communicate once the game has started. Obviously, Alice and Bob want to win the game and so they try to maximize their winning probability by choosing a promising strategy.
\subsection{Classical and entangled strategies}
When Alice and Bob use classic deterministic strategies, they both have a deterministic map $a : \mathcal{S} \rightarrow \mathcal{A}$ and $b : \mathcal{T} \rightarrow \mathcal{B}$ respectively. This means beforehand they both agree on what to answer upon what questions received. The winning probability is easily calculated: 
\begin{equation}
\mathbb{E}_{s,t \sim \pi} \left[ V(a(s),b(t),s,t) \right] 
\end{equation}
But, of course, we are dealing with quantum mechanics here so we are interested in entangled strategies and want to study how the availability of these influence the outcome. For an entangled strategy both Alice and Bob have a subsystem $X_A, X_B$ of a quantum system $X$ which is in state $\rho$, i.e. Alice and Bob share state $\rho$. If the state is entangled we know that measurements can give correlated outcomes, which means for the players that they may gain information about the other players outcome by performing a measurement. More technically, there is a positive integer $n$ and a quantum system $X$ consisting of two $n$-dimensional subsystems $X_A, X_B$ in some entangled state $\rho$. Alice and Bob have measurements $\{ F_s^a \}_{a\in \mathcal{A}} , \{ G_t^b\}_{b \in \mathcal{B}} \subseteq \mathbb{C}^{n \times n}$. When the game starts they both get a question $s$ and $t$ and perform their measurement on it. They both send the outcome of their measurement as their answer to the referee. As has be established before, the probability of Alice answering with $a$ and Bob with $b$ is Tr$(\rho F_s^a \otimes G_t^b)$. The winning probability then equals:
\begin{equation}
\mathbb{E}_{s,t \sim \pi} \left[ \sum_{a \in \mathcal{A}} \sum_{b \in \mathcal{B}} \text{Tr}(\rho F_s^a \otimes G_t^b) V(a,b,s,t) \right]
\end{equation}  
The players want to maximize their winning probability. Since the trace function is linear and states are convex combinations of pure states, we only need to consider pure entangled states. 
\subsection{Two player XOR games}
An XOR game is a game where the set of answers $\mathcal{A}$ and $\mathcal{B}$ only consist of $\{ 0,1 \}$ and the predicate $V$ only depends on the exclusive-OR of the answers and the value function $f : \mathcal{S} \times \mathcal{T} \rightarrow \{ 0,1 \}$. In the following let the spare brackets denote the $0/1$ truth value of the statement in between them. Then we have $V(a,b,s,t) = \left[ a \oplus b = f(s,t) \right]$. 
The exclusive OR returns $1$ if and only if one of the inputs is $1$. In a truth table: \\
\begin{center}
\begin{tabular}{l | c r }
$\oplus$ & 0 & 1 \\
\cline{1-3} 

0 & 0 & 1 \\
1 & 1 & 0 
\end{tabular}\\
\end{center}

For a probability distribution $\pi$ and a boolean function $f$, $\mathcal{G}= (\pi, f)$ defines an XOR game.\\ \\
\textbf{Bias and violation ratio.} Alice and Bob can always win an XOR with probability $\frac{1}{2}$ by flipping an unbiased coin. Interesting would be how much this can actually be increased. We define the classical bias of an XOR game to be the difference of the probability of winning and losing for an optimal classical strategy an denote it by $\beta(G)$. The bias of entangled strategies is calculated in the same way and thus is twice the amount by which the maximal winning probability is greater then $\frac{1}{2}$, since $\frac{1}{2}+\gamma - (1 - \frac{1}{2} + \gamma) = 2 \gamma$, $\gamma$ being the amount exceeding $\frac{1}{2}$. We denote the bias of entangled strategies by $\beta^*(G)$. The violation ratio is given by $\frac{\beta^*(G)}{\beta(G)}$.\\ \\

\textbf{Signs and observables.} To make things a little easier regarding calculations we will use the $\{ -1 , 1 \} $-basis rather than the $\{ 0,1 \}$-basis for boolean valued objects. If we have an XOR game $(\pi, f )$ and any two classical strategies $a : \mathcal{S} \rightarrow \{0,1\}$ and $b: \mathcal{B} |righta
 \{ 0,1 \}$ the bias is given by the probability under $\pi$ that $a(s) \oplus b(t) = f(s,t)$ minus the probability under $\pi$ that $a(s) \oplus b(t) \ne f(s,t)$. 
 \begin{flalign*}
 \mathbb{E}_{(s,t) \sim \pi} \left[ (-1)^{[a(s) \oplus b(t) = f(s,t)]} \right] & = \mathbb{E}_{(s,t) \sim \pi} \left[ (-1)^{a(s) \oplus b(t) + f(s,t)} \right] \\
 &= \mathbb{E}_{(s,t) \sim \pi} \left[ (-1)^{a(s)}(-1)^{b(t)}(-1)^{f(s,t)} \right]
  \end{flalign*}
We define the sign matrix $\Sigma_{st} = (-1)^{f(s,t)}$ and functions $\chi(s) = (-1)^{a(s)}$ and $\psi(t) = (-1)^{b(t)}$. Thus the bias is: 
\begin{equation}
\mathbb{E}_{ ( s , t ) \sim \pi} \left[ \chi (s) \psi (t) \Sigma_{st} \right]
\end{equation}
Since in an XOR game the outcomes are $\{ 0, 1 \} $-valued the measurements Alice and Bob have are $\{ F_s^0, F_s^1 \}$ and $\{ G_t^0, G_t^1 \}$. If we consider an entangled strategy with a pure state $\vert \psi \rangle$ and have projective measurements the probability of Alice and Bob answering with $a,b$ upon receiving $s,t$ respectively is $\langle \psi \vert F_s^a \otimes G_t^b \vert \psi \rangle$. We can calculate the expected value: 
\begin{flalign*}
(1)\cdot \mathbb{P}\left[ a = b \right] + (-1) \cdot \mathbb{P} \left[ a \ne b \right] & = \langle \psi \vert F_s^0 \otimes G_t^0 \vert \psi \rangle + \langle \psi \vert F_s^1 \otimes G_t^1 \vert \psi \rangle - \langle \psi \vert F_s^1 \otimes G_t^0 \vert \psi \rangle - \langle \psi \vert F_s^0 \otimes G_t^1 \vert \psi \rangle \\
&= \langle \psi \vert (F_s^0 - F_s^1) \otimes (G_t^0 - G_t^1) \vert \psi \rangle
\end{flalign*}

As in (3) we define the $\{ -1, 1 \}$- observables $F_s = F_s^0 - F_s^1$ and $G_t^0-G_t^1$ with the property that its difference squared is the identity matrix. Using this strategy the bias becomes 
\begin{equation}
\mathbb{E}_{(s,t) \sim \pi} \left[ \langle \psi \vert F_s \otimes G_t \vert \psi \rangle \right]
\end{equation}
So for any XOR game the bias is defined as the difference of the probabilities of winning and loosing which is, if considering the $\{ -1, 1 \}$ basis, the expected value and we are looking to maximize this quantity. Hence, the bias of an XOR games in classical strategies is: 
\begin{equation}
\max \left\lbrace \mathbb{E}_{(s,t) \sim \pi} \left[ \Sigma_{st} \chi (s) \psi (t) \right] : \chi : \mathcal{S} \rightarrow \{ -1, 1 \}, \psi : \mathcal{T} \rightarrow \{-1, 1 \} \right\rbrace
\end{equation} 
For entangled strategies we need to use the $\sup_{n \in \mathbb{N}}$ since the winning probability might increase indefinitely with the dimension of the quantum system. 
The Bias of entangled strategies is 
\begin{equation}
\sup_{n \in \mathbb{N}} \left\lbrace \mathbb{E}_{(s,t) \sim \pi} \left[ \Sigma_{st} \langle \psi \vert F_s \otimes G_t \vert \psi \rangle \right] : \vert \psi \rangle \in \mathbb{C}^{n} \otimes \mathbb{C}^{n} , F_s, G_t \in O(\mathbb{C}^n) \right\rbrace
\end{equation}
where $O(\mathbb{C}^n)$ denotes the set of $\{ -1, 1 \}$-observables in $\mathbb{C}^{n \times n}$. As shown in [KNP17] we can in fact restrict ourselves to projective measurements. More general measurements like POVMs (positive operator valued (probability) measure) are not advantageous. 

\subsection{The CHSH game}
Let us consider a special instance of XOR games which leads to the result that entangled strategies actually can give a remarkable advantage over classical strategies. The game is named after four scientists Clauser, Horne, Shimony and Holt. The question set is $\{0,1\} \times \{0,1\}$ as well as the answer set. The probability distribution over the question set is the uniform distribution and the predicate $V = \left[ a \oplus b = s \land t\right] $. Note that $s \land t$ only evaluates to $1$ if both $s=1$ and $t=1$, which in the case of the uniform distribution happens in $\frac{1}{4}$ of the cases. The best classical strategy then would be either to always answer $a=0, b=1$ or $a=1, b=0$. Since in both cases $a \oplus b = 0$ Alice and Bob win in with probability $\frac{3}{4}$. We will now study the entangled case and check how much the winning probability may be increased. 
Define 
\begin{equation}
X = \begin{bmatrix}
0 & 1 \\
1 & 0
\end{bmatrix} , Y = \begin{bmatrix}
0 & -i \\ 
i & 0 
\end{bmatrix}
\end{equation}
and note that they anti-commute, i.e. $XY + YX = 0$ and square to the identity matrix $X^2 = Y^2 = I$. For Alice define the observable for question $0$ by $F_0 = X$ and for question $1$ by $F_1 = Y$. Bobs observables are going to be $G_0 = (X-Y)/ \sqrt{2}$ for question $0$ and $G_1 = (X+Y)/\sqrt{2}$ for question $1$.  Define the $\vert \text{EPR} \rangle = \frac{\vert 0 \rangle \vert 0 \rangle + \vert 1 \rangle \vert 1 \rangle}{\sqrt{2}} = \frac{1}{\sqrt{2}} \begin{pmatrix} 1 \\ 0 \\ 0 \\ 1 \end{pmatrix}$.
The following auxiliary calculations will be helpful later: 
\begin{flalign*}
\langle \text{EPR} \vert X \otimes X \vert \text{EPR} \rangle & = \frac{1}{2} \begin{pmatrix}
1 & 0 & 0 &1
\end{pmatrix} \begin{pmatrix}
0 & 0 & 0 & 1 \\
0 & 0 & 1 & 0 \\
0& 1 & 0 & 0 \\
1 & 0 & 0 & 0
\end{pmatrix} \begin{pmatrix}
1 \\ 0 \\ 0 \\ 1
\end{pmatrix}\\
& = \frac{1}{2} \begin{pmatrix}
1 &0&0&1
\end{pmatrix} \begin{pmatrix}
1 \\ 0 \\ 0 \\1
\end{pmatrix} = \frac{2}{2} = 1\\
\end{flalign*}

\begin{flalign*}
\langle \text{EPR} \vert Y \otimes Y \vert \text{EPR} \rangle & = \frac{1}{2} \begin{pmatrix}
1 & 0 & 0 &1
\end{pmatrix} \begin{pmatrix}
0 & 0 & 0 & -1 \\
0 & 0 & 1 & 0 \\
0& 1 & 0 & 0 \\
-1 & 0 & 0 & 0
\end{pmatrix} \begin{pmatrix}
1 \\ 0 \\ 0 \\ 1
\end{pmatrix}\\
& = \frac{1}{2} \begin{pmatrix}
-1 &0&0&-1
\end{pmatrix} \begin{pmatrix}
1 \\ 0 \\ 0 \\1
\end{pmatrix} = -1\\
\end{flalign*}

\begin{flalign*}
\langle \text{EPR} \vert X \otimes Y \vert \text{EPR} \rangle & = \frac{1}{2} \begin{pmatrix}
1 & 0 & 0 &1
\end{pmatrix} \begin{pmatrix}
0 & 0 & 0 & -i \\
0 & 0 & i & 0 \\
0& -i & 0 & 0 \\
i & 0 & 0 & 0
\end{pmatrix} \begin{pmatrix}
1 \\ 0 \\ 0 \\ 1
\end{pmatrix}\\
& = \frac{1}{2} \begin{pmatrix}
i &0&0&-i
\end{pmatrix} \begin{pmatrix}
1 \\ 0 \\ 0 \\1
\end{pmatrix} = 0\\
\langle \text{EPR} \vert Y \otimes X \vert \text{EPR} \rangle & = 0
\end{flalign*}

Lets calculate the expected values of the sign $a \oplus b$: 
\begin{flalign*}
 \bullet \text{   } \langle \text{EPR} \vert F_0 \otimes G_0 \vert \text{EPR} \rangle &= \langle \text{EPR} \vert X \otimes \frac{1}{\sqrt{2}}(X-Y) \vert \text{EPR} \rangle  \\
&= \langle \text{EPR} \vert X \otimes \frac{1}{\sqrt{2}}X \vert \text{EPR} \rangle - \langle \text{EPR} \vert X \otimes \frac{1}{\sqrt{2}}Y \vert \text{EPR} \rangle \\
& = \frac{1}{\sqrt{2}} - 0 = \frac{1}{\sqrt{2}}\\
\end{flalign*}

\begin{flalign*}
 \bullet \text{   } \langle \text{EPR} \vert F_1 \otimes G_1 \vert \text{EPR} \rangle &= \langle \text{EPR} \vert Y \otimes \frac{1}{\sqrt{2}}(X+Y) \vert \text{EPR} \rangle \\
&= \langle \text{EPR} \vert Y \otimes \frac{1}{\sqrt{2}}X \vert \text{EPR} \rangle + \langle \text{EPR} \vert Y \otimes \frac{1}{\sqrt{2}}Y \vert \text{EPR} \rangle \\
& = 0 - \frac{1}{\sqrt{2}} = - \frac{1}{\sqrt{2}}\\
\end{flalign*}

\begin{flalign*}
 \bullet \text{   } \langle \text{EPR} \vert F_0 \otimes G_1 \vert \text{EPR} \rangle &= \langle \text{EPR} \vert X \otimes \frac{1}{\sqrt{2}}(X+Y) \vert \text{EPR} \rangle  \\
&= \langle \text{EPR} \vert X \otimes \frac{1}{\sqrt{2}}X \vert \text{EPR} \rangle + \langle \text{EPR} \vert X \otimes \frac{1}{\sqrt{2}}Y \vert \text{EPR} \rangle \\
& = \frac{1}{\sqrt{2}} + 0 = \frac{1}{\sqrt{2}}\\
\end{flalign*}

\begin{flalign*}
 \bullet \text{   } \langle \text{EPR} \vert F_1 \otimes G_0 \vert \text{EPR} \rangle &= \langle \text{EPR} \vert Y \otimes \frac{1}{\sqrt{2}}(X-Y) \vert \text{EPR} \rangle  \\
&= \langle \text{EPR} \vert Y \otimes \frac{1}{\sqrt{2}}X \vert \text{EPR} \rangle - \langle \text{EPR} \vert Y \otimes \frac{1}{\sqrt{2}}Y \vert \text{EPR} \rangle \\
& = 0 - (-\frac{1}{\sqrt{2}}) =  \frac{1}{\sqrt{2}}
\end{flalign*}
Thus, we have
\begin{equation}
\langle \text{EPR} \vert F_s \otimes G_t \vert \text{EPR} \rangle = \begin{cases} \frac{1}{\sqrt{2}} , (0,0), (1,0), (0,1) \\ -\frac{1}{\sqrt{2}} , (1.1) \end{cases}
\end{equation}
which is equivalent to 
\begin{equation}
\langle \text{EPR} \vert F_s \otimes G_t \vert \text{EPR} \rangle = \frac{(-1)^{s \land t}}{\sqrt{2}} , s,t \in \{ 0,1 \}
\end{equation}
The bias of the entangled strategy equals 
\begin{flalign*}
\mathbb{E}_{(s,t) \sim \pi} \left[ \langle \psi \vert F_s \otimes G_t \vert \psi \rangle \right] & = \frac{1}{4} \sum_{s,t = 0}^1 (-1)^{s \land t} \langle \text{EPR} \vert F_s \otimes G_t \vert \text{EPR} \rangle \\
&= \frac{1}{4} \cdot \frac{4}{\sqrt{2}} = \frac{1}{\sqrt{2}}
\end{flalign*}
The bias is $\frac{1}{\sqrt{2}}$ from which follows that the winning probability is by definition: 
\begin{equation}
\frac{1}{2}+ \frac{1}{2}\cdot \frac{1}{\sqrt{2}} = \cos(\pi/8 ) \approx 0.85\dots 
\end{equation}