\subsection{Local Correlation matrices}
So far, we have been dealing with quite specific strategies. The idea is to generalize the concept of strategies into mathematical objects. Suppose the referee sends an element $ s \in \mathcal{S} $ to Alice, respectively $ t \in\mathcal{T} $ to Bob. Suppose Alice and Bob answer according to a deterministic strategy. We will interpret their answers as vectors $ \xi \in \{-1,1\}^{\mathcal{S}}, \eta \in \{-1,1\}^{\mathcal{T}} $.  Their common answer is the product $ \xi_s\eta_t $. So, their strategy can be uniquely described by a  matrix $ \xi\eta^\top $.  Instead of playing a deterministic strategy they could answer in a probabilistic way, meaning that given the input $ s \in \mathcal{S} $, respectively $ t \in \mathcal{T} $ their answers are determined by random variables $ X_s $ and $ Y_t $. Then, their expected common answer is $ \mathbb{E}[X_sY_t] $. In the following definition we define the set of all such matrices encoding the common strategy of Alice and Bob.

\begin{dfn}
	Let $ (X_i)_{1 \le i \le m } $ and $ (Y_j)_{1 \le j \le n} $ be families of random variables on a common probability space such that $ \vert X_i \vert, \vert Y_j \vert \le 1 $ almost surely. Then $ A=(a_{ij}) $ is the corresponding {\itshape classical (or local) correlation matrix} if 
	\begin{align*}
		a_{ij} = \mathbb{E}[X_iY_j]
	\end{align*}
	for all $ 1 \le i \le m, 1 \le j \le n $.
\end{dfn}
As we will see in the sequel, the set of $ m \times n $ correlation matrices is a polytope, denoted by $ \LC_{m,n} $.
\begin{lemma}\label{LemLC}
	An alternative description of $ \LC_{m,n} $ is given by 
	\begin{align}\label{EqLC}
		\LC_{m,n} = \textup{conv} \{  \xi\eta^T \, | \, \xi \in \{-1,1\}^m, \eta \in \{-1,1\}^n     \}.
	\end{align}
\end{lemma}
What does the lemma tell us in addition to a simples description? It states that the matrices encoding all possible strategies are the convex hull of the matrices defined by deterministic ones. We can interpret this as follows: no matter which probabilistic strategy Alice and Bob play there will be a deterministic strategy that is at least as good as theirs. 
\begin{proof}
	Let us denote the right hand side of \ref{EqLC} by $ M $ and let $ \xi\eta^T \in M $ with $ \xi \in \{-1,1\}^m, \eta \in \{-1,1\}^n $. Clearly $ \xi_i, \eta_j \in \{-1,1\} $ define constant $ \mathbb{R} $-valued random variables satisfying $ \vert \xi_i \vert, \, \vert \eta_j \vert \le 1 $. Hence, it suffices to show that $ \LC_{m,n} $ is convex since it contains the vertices of $ M $. Therefore, consider two classical correlation matrices $ a_{ij}^{(k)} = \mathbb{E}[X_i^{(k)}Y_{j}^{(k)}] $ for $ k \in \{0,1\} $ which are defined on a common probability space such that $ \modul{X_i^{(k)}},\modul{Y_j^{(k)}} \le 1 $ We have to show that there exists random variables $ (X_i),(Y_j) $ with $ \modul{X_i},\modul{Y_j} \le 1 $ almost surely such that
	\begin{align*}
		\beta a_{ij}^{(0)}+ (1-\beta)a_{ij}^{(1)} = \mathbb{E}[X_iY_j]
	\end{align*}
	for all $ \beta \in [0,1] $.
	Let $ \alpha $ be a Bernoulli random variable, i.e. $ \mathbb{P}(\alpha = 0) = \beta $, $ \mathbb{P}(\alpha = 1) = 1 - \beta$ and set $ X_i = X_i^{(\alpha)}, Y_j = Y_j^{(\alpha)} $.
	Clearly, it holds $ \modul{X_i}, \modul{Y_j} \le 1 $ almost surely. 
	Then 
	\begin{align*}
		\mathbb{E}[X_iY_j] &= \mathbb{E}[X_i^{(\alpha)}Y_j^{(\alpha)}  \mathds{1}_{ \{\alpha = 0\}}] + \mathbb{E}[X_i^{(\alpha)}Y_j^{(\alpha)}]\mathds{1}_{\{\alpha = 1\}}] \\
		&= \beta \mathbb{E}[X_i^{(0)}Y_j^{(0)} ] + (1-\beta) \mathbb{E}[X_i^{(1)}Y_j^{(1)}],
	\end{align*} 
	which proofs that $ \LC_{m,n} $ is convex.
	
	
	
	For the other inclusion, let $ (a_{ij}) \in \LC_{m,n} $, i.e. $ a_{ij} = \mathbb{E}[X_iY_j] $ for $ \mathbb{R} $-valued random variables $ (X_i),(Y_j) $, defined on a common probability space $ \Omega $ with $ \modul{X_i},\modul{Y_j} \le 1 $ almost surely. 
	We will use the characterization of the $ d-$dimensional cube by its vertices, that is $ [-1,1]^d = \textup{conv} \{\xi \, | \, \xi \in \{-1,1 \}^d \}$ (this can be proved by induction). 
	If we define the random variables $ X= (X_1,\hdots,X_m) $ and $ Y= (Y_1,\hdots,Y_n) $ they satisfy $ X \in [-1,1]^m, \, Y \in [-1,1]^n $ almost surely. Using the characterization of the hypercube we can define random variables $ \lambda_{\xi}^{(X)}: \Omega^m \to [0,1] $ such that 
	\begin{align*}
		X(\omega) = \sum_{\xi \in \{-1,1\}^m}\lambda_{\xi}^{(X)}(\omega)\xi
	\end{align*} 
	almost surely 
	and $ \sum_{\xi \in \{-1,1\}^m}\lambda_{\xi}^{(X)}(\omega) = 1  $ for all $ \omega \in \Omega $. We can deduce that 
	$X_i =  \sum_{\xi \in \{-1,1\}^m}\lambda_{\xi}^{(X)}\xi_i $ almost surely.
	If we proceed analogously for $ Y $ we obtain
	\begin{align*}
		a_{ij} = \mathbb{E}[X_iY_j] &= \mathbb{E} \big [  (\sum_{\xi \in \{-1,1\}^m}\lambda_{\xi}^{(X)}\xi_i ) (\sum_{\eta \in \{-1,1\}^n}\lambda_{\eta}^{(Y)}\eta_j ) \big ]   \\
		&= \sum_{\xi \in \{-1,1\}^m, \eta \in \{-1,1\}^n} \mathbb{E}\big [\lambda_{\xi}^{(X)}\lambda_{\eta}^{(Y)} \big ] \xi_i \eta_j.
	\end{align*}
	If we now observe that $
		\sum_{\xi \in \{-1,1\}^m, \eta \in \{-1,1\}^n}\mathbb{E} [\lambda_{\xi}^{(X)}\lambda_{\eta}^{(Y)}] = 1 $ it follows that $ (a_{ij}) \in  \textup{conv} \{  \xi\eta^T \, | \, \xi \in \{-1,1\}^m, \eta \in \{-1,1\}^n   \}$
 which finishes the proof.
\end{proof}
\noindent Now we are able to count the vertices of $ \LC_{m,n} $. Observing that $ \xi \eta^T = \tilde{\xi} \tilde{\eta}^T $ if and only if $ \xi = \tilde{\xi} $ and $ \eta = \tilde{\eta} $ or $ \xi = -\tilde{\xi} $ and $ \eta = -\tilde{\eta} $ it follows that we have $ 2^{n+m}/2 = 2^{n+m-1} $ different matrices $ \xi \eta $, hence $ \LC_{m,n} $ has $ 2^{n+m-1} $ vertices. To analyze the facial structure of $ \LC_{m,n} $ is rather complicated. 
However, we will do it later on for $ n=m=2 $ and compare it to the set of quantum correlation matrices. 

\subsection{Quantum correlation matrices}
Before we introduce the quantum correlation matrices we will shortly go back to the framework of nonlocal games. As before, Alice and Bob share a state $ \rho $ and get inputs $ s \in \mathcal{S}, t \in \mathcal{T} $ and perform measurements $ \{ F_s^{\xi} \}_{\xi = \pm 1} $, respectively $ \{ G_t^{\eta} \}_{\eta = \pm 1} $.
As we have seen before the probability that their response is $ (\xi,\eta) $ for inputs $ (s,t) $ is given by 
$ a_{st} = \trace{(\rho F_s^\xi \otimes G_t^\eta )} $. Again, we can encode these information in a matrix $ A=(a_{st}) $. This motivates the following (slightly more general) definition. 
\begin{dfn}
	Let $ (X_i)_{1 \le i \le m } $ and $ (Y_j)_{1 \le j \le n} $ be self-adjoint operators on $ \mathbb{C}^{d_1} $, respectively $ \mathbb{C}^{d_2} $ for some positive integers $ d_1,d_2 $, satisfying $ \norm{X_i}_{\infty}, \norm{Y_j}_{\infty} \le 1 $. $ A = (a_{ij}) $ is called {\itshape quantum correlation matrix} if there exists a state $ \rho \in B^{psd}(\mathbb{C}^{d_1} \otimes \mathbb{C}^{d_2})$ such that 
	\begin{align}\label{QCaij}
	a_{ij} = \trace{\rho (X_i \otimes Y_j)}.
	\end{align}
\end{dfn}
\noindent We will write $ \QC_{m,n} $ for the set of all $ m \times n $ quantum correlation matrices.
With regard to quantum information theory it is interesting to analyze the geometry of $ \LC_{m,n} $ and $ \QC_{m,n} $. In the following, we will proof a similar result for $ \QC_{m,n} $ that is: 
\begin{lemma}\label{LemQC}
	\begin{align}\label{EqQC}
		\QC_{m,n} = \{ (\langle x_i,y_j \rangle)_{1 \le i \le m, 1 \le j \le n} \,| \, x_i,y_j \in \mathbb{R}^{ \min \{m,n \} }, \vert x_i  \vert \le 1, \vert y_j \vert \le 1  \},
	\end{align}
	where $ \langle \cdot , \cdot \rangle $ denotes the standard scalar product. 
\end{lemma}
\noindent For the first inclusion we review the definition of an inner product. The basic idea is to define an inner product via the  right hand sight of \ref{QCaij}.
Let $ V $ and $ W $ be two vector spaces and $ k  $ a field. A {\itshape bilinear form} is a map $ \beta: \, V \times W \to k $ that is linear in both variables, that is 
\begin{enumerate}
	\item [i)] $\beta(v_1+v_2,w) = \beta(v_1,w) + \beta(v_2,w)  $
	\item  [ii)]$\beta(\lambda v,w)= \lambda \beta(v,w) $
	\item [iii)]$ \beta(v,w_1+w_2) = \beta(v,w_1)+ \beta(v,w_2) $
	\item [iv)]$ \beta(v,\lambda w) = \lambda\beta(v,w) $
\end{enumerate}
for all $ v,v_1,v_2 \in V, \, w,w_1,w_2 \in W, \, \lambda \in k $. 
If $ V = W $, we call $ \beta $ {\itshape symmetric} if $ \beta(v,w) = \beta(w,v) $, {\itshape positive semidefinite} if 
$ \beta(v,v) \ge 0 $ and {\itshape positive definite} if $ \beta $ is positive semidefinite and $ \beta(v,v)= 0 $ implies that $ v = 0 $. 
If $ \beta: \, V \times V \to k $ is a symmetric positive definite bilinear form it is called an {\itshape inner product}.
Note that if $ H $ is a positive semidefinite operator then $ \beta(v,w) = v^TH\bar{w} $ defines an positive semidefinite bilinear form and an inner product if $ H $ is positive definite. 
Conversely, each positive semidefinite bilinear form $ \beta $ can be written as $ \beta(v,w) = v^TH\bar{w} $ for an Hermitian operator $ H $.
As before, we will denote the right hand side of equation \ref{EqQC} by $ M $.
\begin{proof}[Proof of $ \QC_{m,n} \subset M $]
	Let $ (a_{ij}) \in \QC_{m,n} $. Then there is a sate $ \rho $ on a Hilbert space $ \mathcal{H} = \mathbb{C}^{d_1} \otimes\mathbb{C}^{d_2} $ and Hermitian operators $ \{X_1,\hdots X_m\}\subset  B^H(\mathcal{C}), \, \{Y_1,\hdots Y_n\}\subset B^H(\mathcal{H}) $ satisfying $ \norm{X_i}_{\infty}, \norm{Y_j}_{\infty} \le 1 $ such that 
	$ a_{ij} = \trace{\rho X_i \otimes Y_j} $.
	We define a positive semidefinite symmetric bilinear form on the space of Hermitian operators on $ \mathcal{H} $ by 
	$ \beta: \textup{B}^{H}(\mathcal{H}) \times \textup{B}^{H}(\mathcal{H}) \to \mathbb{R} $ where $ \beta(S,T) =\textup{Re}( \trace{\rho ST}) $.
	We have to check that it indeed satisfies the mentioned properties. 
	Obviously, $ \beta $ is homogeneous in both variables due to the fact that the trace and the real part of a complex number are linear functions and thus homogeneous. We will show additivity for the first variable, the result follows analogously for the second one. It holds
	\begin{align*}
		\beta(S_1+S_2,T) = \textup{Re}(\trace{\rho(S_1+S_2)T}) &= \textup{Re}(\trace{\rho S_1T}) + \textup{Re}(\trace{\rho S_2T}) \\
		&= \beta(S_1,T)+ \beta(S_2+T).
	\end{align*}
	Symmetry follows form 
	\begin{align*}
		\beta(S,T)&= \textup{Re}\trace{(\rho ST)} =\textup{Re}\trace{(\rho ST)^*}  = \textup{Re \trace{(T^*S^* \rho^*})}  \\
		 &=\textup{Re} \trace{(\rho^* T^*S^*)} = \textup{Re} \trace{(\rho T S)} = \beta(T,S).
	\end{align*}
	Moreover, since $ S^*S $ is a positive semidefinite operator for all complex operators $ S $ and $  \rho $ is positive semidefinite we obtain $ \beta(S,S) = \textup{Re}\trace{(\rho SS)} = \textup{Re} \trace{(\rho S^*S)} \ge 0 $, so $ \beta $ is positive semidefinite. 
	Following the steps in appendix \ref{Appendix} we can factorize the kernel and transform $ \beta $ to an inner product on the vector space $ B^H(\mathcal{H})/ \ker{\beta} $. 
	
	\noindent Now, the crucial point is to notice that $ a_{ij} =\trace{(\rho X_i Y_j)}=  \beta(X_i \otimes I,I \otimes Y_j) $. In order to show that these operators coincide with unit vectors with respect to $  \beta  $ we have to show
	$ \beta(X \otimes I, X \otimes I), \beta(I \otimes Y, I \otimes Y) \le 1$ for all $ X \in \{X_1,\hdots,X_m \} $, $ Y \in \{Y_1,\hdots,Y_n \} $. 
	Therefore, let $ d = d_1d_2 $ and let $ \rho $ be a pure state, that is $ \rho = \vert \psi \rangle \langle \psi \vert $ for $ \vert \psi \rangle = \sum_{i}^{d}\lambda_i \xi_i \otimes \eta_i $, where $ \{ \xi_1,\hdots,\xi_d \} \subset \mathbb{C}^{d_1}$ and $ \{  \eta_1,\hdots,\eta_d\} \subset \mathbb{C}^{d_2}$ are orthonormal sets, i.e. $ \braket{\xi_i \vert \xi_j}, \braket{\eta_i \vert \eta_j} = 0 $ for $ i \neq j $, and $ \sum_{i=1}^d \lambda_i^2 = 1 $. This decomposition is called {\itshape Schmidt decomposition} and a consequence of the singular value decomposition. If we can show the desired property for pure sates we can immediately conclude that it holds for general states since each state is a convex combination of pure states. 
	Writing $ \rho $ in this form we get 
	\begin{align*}
	\beta(X \otimes I, X \otimes I) &= \textup{Re} \trace{(\rho X^2 \otimes I)}=  \sum_{i=1}^d \lambda_i^2 \trace{\big ((\ket{\xi_i} \bra{\xi_i} \otimes \ket{\eta_i}\bra{\eta_i})(X^2 \otimes I)\big ) } \\
	&= \sum_{i=1}^{d}\lambda_i^2 \trace{(\ket{\xi_i} \bra{\xi_i}X^2)}\trace{(\ket{\eta_i}\bra{\eta_i})} =  \sum_{i=1}^{d}\lambda_i^2 \trace{(\ket{\xi_i} \bra{\xi_i}X^2)}.
	\end{align*}
	In order to get the desired result we have to show that $ \trace{(\ket{\xi_i} \bra{\xi_i}X^2)} = \trace{(X^2\ket{\xi_i} \bra{\xi_i})} \le 1 $. 
	Note that $1 \ge \norm{X}_{\infty} := \sup_{\modul{y}\le 1} \modul{Xy} $ implies that $ \modul{X^2\ket{\xi_i}} \le 1 $. So the problem can be reduced to $ \vert \trace{uv^*}\vert^2 \le 1 $ for complex vectors $ u,v $ with $ \modul{u},\, \modul{v} \le 1 $. But this holds since due to the Cauchy-Schwarz inequality 
	$ \vert\trace{uv^*}\vert^2 = \vert\sum u_i \bar{v_i}\vert^2 \le \modul{u}^2\modul{v}^2 \le 1 $.
	\noindent If we now identify the operators $ X_i \otimes I $ and $  I \otimes Y_j$ with vectors
	$ (x_i) $ and $ (y_j) $ we have found vectors that almost satisfy the desired properties but they do not have the right dimension and we do not consider the standard scalar product yet. Without loss of generality let $ m \le n $. To obtain the required dimension we will project $ (y_j) $ orthogonally onto $ \textup{span}\{ x_1,\hdots,x_m \} $.
	Let $ \{a_1,\hdots,a_r\} $ be an orthonormal basis of $ \textup{span} \{ x_1,\hdots,x_m \} $ with respect to $ \beta $. 
	The orthogonal projection of $ y_j $ is $ \pi(y_j):= \sum_{i=1}^{r}\beta(a_i,y_j)a_i $ and fulfills
	$ \beta(x_i,y_j) = \beta(x_i,\pi(y_j)) $. 
	Let $ x_i $ and $ \pi(y_j) $ admit the descriptions 
	$ x_i = \sum_{k=1}^{r}\alpha_k^{(i)}a_k$ and $  \pi(y_j) = \sum_{k=1}^r \gamma_k^{(j)} a_k$ for $ \alpha^{(i)}, \gamma^{(j)} \in \mathbb{R}^r $. Then
	\begin{align*}
	a_{ij}= \beta(x,y)= \beta(x,\pi(y)) = \sum_{1 \le k,l \le r} \alpha_k^{(i)} \gamma_l^{(j)} \beta(a_k,a_l) = \sum_{k=1}^{r}\alpha_k^{(i)}\gamma_k^{(j)} = \langle \alpha^{(i)}, \gamma^{(j)} \rangle.
	\end{align*}
	Moreover, since $ \vert\alpha^{(i)} \vert= \modul{x_i} \le 1$ and $\vert \gamma^{(j)}\vert=  \modul{y_j}= \modul{\pi(y_j)} \le 1 $ the vectors $ (\alpha^{(i)}) $ and$ (\gamma^{(j)}) $ have all the properties of the right hand side of \ref{EqQC}.
	Additionally, we also proved that we can demand an even lower dimension for the vectors in \ref{EqQC}, that is  
	$ \min \big \{ \dim( \textup{span} \{ (x_i)_{1 \le i \le m} \}),\, \dim(\textup{span} \{ (y_j)_{1 le j \le n} \} )\big \} $.
\end{proof}

\noindent In order to prove the other inclusion we will use the following proposition. 
\begin{prop} \label{PauliProp}
For all $ n \ge 1 $ there is a subspace of the $ 2^n \times 2^n $ Hermitian matrices where every vector is the multiple of a unitary matrix. 
\end{prop}
\begin{proof}
The proof is based on $ n- $fold tensor products of the Pauli matrices which are the three matrices 
\begin{align*}
X = \begin{pmatrix}
0 & 1 \\ 1 & 0
\end{pmatrix}, \, Y = \begin{pmatrix}
0 & -i \\ i & 0
\end{pmatrix}, \, Z = \begin{pmatrix}
1 & 0 \\ 0 & -1
\end{pmatrix}
\end{align*}
together with the $ 2 \times 2 $ identity matrix .
They are all trace zero unitary Hermitian matrices and anti-commute pair wisely. 
Moreover we define the $ 2^n \times 2^n $ Hermitian matrix 
\begin{align*}
\sigma_A^i = I^{\otimes (i-1)} \otimes A \otimes I^{\otimes (n-i)}
\end{align*}
for $ A \in \{ X,Y,Z \}$ and where $ I $ is the $ 2 \times 2 $ identity matrix. The Hermitian property follows directly from the observation $ (M \otimes N)^* = M^* \otimes N^* $.
Note that $ \sigma_A^i $ and $ \sigma_{A^{\prime}}^j $ anti-commute if $ i = j $ and $ A = A^{\prime} $ and commute otherwise. We use these operators in order to define 
\begin{align*}
U_i = \sigma_X^i \prod_{k = i+1}^{n}\sigma_Y^k, \\
U_{i+n} = \sigma_Z^i \prod_{k = i+1}^n \sigma_Y^k
\end{align*}
for $ i = 1,\hdots,n $. Note that these operators are also traceless Hermitian matrices and anti-commute for $ i \neq j $:    
for $ 1 \le i < j  \le n$ we have 
\begin{align*}
U_iU_j = (\sigma_X^i \prod_{k = i+1}^{n}\sigma_Y^k) \cdot( \sigma_X^j \prod_{k = j+1}^{n}\sigma_Y^k) &= \sigma_X^i \sigma_Y^{i+1} \cdots \sigma_Y^j ( \sigma_X^j \prod_{k = j+1}^{n}\sigma_Y^k) \sigma_Y^{j+1} \cdots \sigma_Y^n \\
&= - \sigma_X^i \sigma_Y^{i+1} \cdots \sigma_Y^{j-1} ( \sigma_X^j \prod_{k = j+1}^{n}\sigma_Y^k) \sigma_Y^{j+1} \cdots \sigma_Y^n  \\
&= - ( \sigma_X^j \prod_{k = j+1}^{n}\sigma_Y^k)(\sigma_X^i \prod_{k = i+1}^{n}\sigma_Y^k)  \\
&= -U_jU_i
\end{align*}
and 
\begin{align*}
U_iU_{n+j} =  (\sigma_X^i \prod_{k = i+1}^{n}\sigma_Y^k) \cdot( \sigma_Z^j \prod_{k = j+1}^{n}\sigma_Y^k) &=
\sigma_X^i \sigma_Y^{i+1} \cdots \sigma_Y^j ( \sigma_Z^j \prod_{k = j+1}^{n}\sigma_Y^k) \sigma_Y^{j+1} \cdots \sigma_Y^n \\
&= - \sigma_X^i \sigma_Y^{i+1} \cdots \sigma_Y^{j-1} ( \sigma_Z^j \prod_{k = j+1}^{n}\sigma_Y^k) \sigma_Y^{j+1} \cdots \sigma_Y^n  \\
&= - ( \sigma_Z^j \prod_{k = j+1}^{n}\sigma_Y^k)(\sigma_X^i \prod_{k = i+1}^{n}\sigma_Y^k)  \\
&= -U_{n+j}U_i.
\end{align*}
Since $ U_iU_i^* = U_iU_i =  I$ they are also unitary. 
Moreover, taking the product of two linear combinations $ X = \sum_{i = 1}^{2n}\xi_i U_i$, $ Y = \sum_{i = 1}^{2n} = \eta_iU_i $ we can calculate 
\begin{align*}
XY = \sum_{i = 1}^{2n} \xi_i \eta_i I + \sum_{1 \le i,j \le  2n}\xi_i\eta_j U_i U_j &= \sum_{i = 1}^{2n} \xi_i \eta_i I + \sum_{1 \le i < j \le  2n}\xi_i\eta_jU_iU_j - \sum_{1 \le i < j \le  2n}U_iU_j \\
&=\sum_{i = 1}^{2n} \xi_i \eta_i I \\
&= \langle \xi, \eta \rangle I.
\end{align*}
So, if we set $ Y = X $ we get the desired result by taking the subspace $ \textup{span} \{ U_i \, | \, i = 1,\hdots,2n \} $

\end{proof}

We will use this in order to proceed with the other inclusion of lemma \ref{LemQC}.

\begin{proof}[Proof of $ \LC_{m,n} \supset M $]
Let $ (x_i)_{1 \le i \le m}, \, (y_j)_{1 \le j \le n} $ be vectors in $\mathbb{R}^{\min \{ m,n \}}$ that satisfy 
$ \modul{x_i},$$ \modul{y_j} \le 1 $. 
Using the notation of the previous proposition's proof we set $ X_i = \sum_{k=1}^{\min \{m,n\}} x_i(k)U_k $ and $ Y_j^{T } = \sum_{k=1}^{\min \{m,n\}}y_j(k)U_k $ where the $ U_k $'s are $ d \times d $ matrices with $  d = 2^{\lceil \min \{m,n\}/2 \rceil} $
Then, by the proof of the previous proposition $ \trace{(X_iY_j^T)} = d\cdot \langle x_i, y_j \rangle  $ and $ \norm{X_i}_{\infty} \le 1 $ since $ X_iX_i^* = \modul{x_i}^2I $ and $ \modul{x_i}^2 \le 1  $. The same holds for $ Y $ since $ Y_j^TY_j^T =   \modul{y_j}^2I = \modul{y_j}I^T = (Y_j^TY_j^T)^T = Y_jY_j  $.
Let $ \ket{\psi} = \frac{1}{\sqrt{d}}\sum_{i= 1}^{d}\ket{ii} $ and $ \rho = \ket{\psi}\bra{\psi} $. Note that we can write $ \rho $ as
\begin{align*}
\rho = \ket{\psi}\bra{\psi} = \frac{1}{d}\sum_{1 \le k,l \le d}\ket{kk}\bra{ll} = \frac{1}{d} \sum_{1 \le k,l \le d}\ket{k}\bra{l} \otimes \ket{k}\bra{l}
\end{align*}
where $  (\ket{k}\bra{l})_{kl} = 1 $ and $ (\ket{k}\bra{l})_{ij} = 0 $ for all $ (i,j)\neq (k,l) $.
Calculating
\begin{align*}
\trace{(\rho X_i \otimes Y_j)} &=\frac{1}{d} \sum_{1 \le k,l \le d} \trace{ \big (\ket{k}\bra{l}X_i \otimes \ket{k}\bra{l}Y_j \big )} = \frac{1}{d} \sum_{1 \le k,l \le d} \trace{\big (\ket{k}\bra{l}X_i \big )} \trace{\big (\ket{k}\bra{l}Y_j \big )} \\
&=  \frac{1}{d} \trace{X_iY_j^T} = \langle x_i,y_j \rangle.
\end{align*}
shows the desired inclusion. 
\end{proof}
\noindent We can easily see that $ \QC_{m,n} $ is convex. Consider $ (a_{ij}), (\bar{a}_{ij}) \in \QC_{m,n} $ with $ a_{ij} = \langle x_i, y_j \rangle $ and $ \bar{a}_{ij} = \langle \bar{x}_i, \bar{y}_j \rangle $ for $ x_{i},y_j, \bar{x}_i,\bar{y}_j \in \mathbb{R}^{\min \{m,n\}} $ such that $ \modul{x_i},\modul{y_j}, \modul{\bar{x}_i}, \modul{\bar{y}_j} \le 1 $.
For $ \lambda \in [0,1] $ we define vectors $\tilde{x}_i:= (\sqrt{\lambda}x_i,\sqrt{1-\lambda}\bar{x_i}), \, \tilde{y}_j:= (\sqrt{\lambda}y_j, \sqrt{1-\lambda}\bar{y}_j) $ and due to  $ \modul{\tilde{x}_i} \le \lambda \modul{(x_i,0)} + (1-\lambda)\modul{(0,\bar{x}_i)} \le 1 $ they are unit vectors. Moreover, $ \langle \tilde{x}_i, \tilde{y}_j \rangle = \lambda \langle x_i,y_j \rangle + (1-\lambda) \langle \tilde{x_i},\tilde{y}_j \rangle$. If we proceed in the same fashion as in the proof of lemma \ref{LemQC} we obtain vectors 
$ \alpha^{(i)},\gamma^{(j)} $ that satisfy $= \langle \alpha^{(i)},\gamma^{(j)} \rangle = \langle \tilde{x} _i, \tilde{y}_j\rangle $ and have dimension smaller or equal to $ \min \{m,n\} $.


\subsection{The relations between quantum correlation and local correlation matrices}

Using the descriptions of lemmata \ref{LemLC} and \ref{LemQC} we can derive some relations between the two sets. 
Let $ \xi\eta^T $ be a vertex of $ \LC_{m,n} $. If we just choose $ x_i = \xi_i (1,0,\hdots,0) \in \mathbb{R}^{\min \{m,n\}}$ and $ y_j = \eta_j(1,0,\hdots 0) \in \mathbb{R}^{\min \{m,n \}}$ we immediately see that $ x_i $ and $ y_j $ are unit vectors and $ \xi_i\eta_j = \langle x_i, y_j \rangle $. Hence, $ \xi\eta^T \in \QC_{m,n} $ and combined with the convexity of $ \QC_{m,n} $ we get $ \LC_{m,n} \subset \QC_{m,n} $.

However, the inclusion is strict in general. Let us consider $ n=m=2 $. 
For $ \LC_{2,2} $ we obtain 
\begin{align*}
\LC_{2,2}= \textup{conv} \{ \pm \begin{pmatrix}
1 & 1 \\
1 & 1
\end{pmatrix} , \, \pm \begin{pmatrix}
-1 & -1 \\
1 & 1
\end{pmatrix} , \, \pm \begin{pmatrix}
-1 & 1 \\
-1 & 1
\end{pmatrix}, \, \pm \begin{pmatrix}
-1 & 1 \\
1 & -1
\end{pmatrix}  \}.
\end{align*}
We can easily see that $ \sigma(\begin{pmatrix}
-1 & 1 \\ 1 & 1 
\end{pmatrix}) \notin \LC_{2,2} $ for $ \sigma \in \Sigma_4 $ where for $ A = (a_{ij}) $ we define $ \sigma (A)  $ by $ (\sigma( A))_{ij}:= a_{\sigma(1)\sigma(j)} $.
We claim that
\begin{equation}\label{facetsLC}
\LC_{2,2} = \{ A \in \mathbb{R}^{2 \times 2} \, | \, -1 \le  \trace{(AM)} \le 1 \textup{ for all } M \in \mathcal{K} \},
\end{equation}
where $ \mathcal{K}  = \{ \frac{1}{2}\sigma(\begin{pmatrix}
-1 & 1 \\
1 & 1
\end{pmatrix}), \sigma(\begin{pmatrix}
1 & 0 \\
0 & 0
\end{pmatrix}) \, | \,  \sigma \in \{ \textup{id}, (1 \, \, 2), (1 \, \, 3), (1 \, \, 4) \} \} $.
The crucial observation is to note that $ \LC_{2,2} $ is affinely isomorphic to the cross polytope scaled  by two, i.e. 
$ \LC_{2,2} \cong 2\textup{CP}_4 := 2\textup{conv} \{  \pm e_i \, | \, i = 1,\hdots,4 \} $, where $ e_i $ are the vectors of the standard basis of $ \mathbb{R}^4 $. For example, this can be seen by interpreting the vertices of $ \LC_{2,2} $ as elements of $ \mathbb{R}^4 $ and then apply the linear transformation given by the matrix 
\begin{align*}
\begin{pmatrix}
-1 & -1 & 1 & 1 \\
-1 & 1 & -1 & 1 \\
-1 & 1 & 1 & -1 \\
1 & 1 & 1 & 1 
\end{pmatrix}.
\end{align*}
Since the polar dual of the cross polytope is the hypercube, i.e. $ (CP_n)^o = [-1,1]^n $,  the face lattice of $ \textup{CP}_4 $ is isomorphic to the opposite lattice of the hypercube's face lattice which implies that the number of facets of $ \textup{CP}_4 $ coincides with the number of vertices of $ [-1,1]^4 $ which is $ 2^4 $. Due to $  \LC_{2,2} \cong 2\textup{CP}_4$ their face lattices of $ \LC_{2,2} $ and $ \textup{CP}_4 $ are isomorphic, so $ \LC_{2,2} $ has $ 2^4 $ facets as well. 
Since all constraints of the right hand side of equation \ref{facetsLC} clearly define non-empty proper faces of $ \LC_{2,2} $, it suffices to show that the characterization is a non-redundant hyperplane description of $ \LC_{2,2} $, implying that all constraints define facets of $ \LC_{2,2} $. But this is indeed true since if we omit for example the constraint 
$ 1/2 \trace{(AM)} \le 1 $ for $ M = \{ \begin{pmatrix}
1 & 0 \\ 0 & 0
\end{pmatrix} \} $, respectively $ M = \begin{pmatrix}
-1 & 1 \\ 1 & 1 
\end{pmatrix} $
the matrices $ \begin{pmatrix}
2 & 0 \\ 0 & 0 
\end{pmatrix} $, respectively $ \begin{pmatrix}
-1 & 1 \\ 1 & 1
\end{pmatrix} $ 
satisfy all other constraints. 
Eventually, we have all information to show that $ \LC_{2,2} $ is a proper subset of $ \QC_{2,2} $. 
Assume we want to maximize in the direction of the facet induced by $M = \begin{pmatrix}
1 & 1 \\ 1 & -1 
\end{pmatrix} $. Note that $ M $ coincides with the matrix $ \Sigma_{s,t} = (-1)^{f(s,t)} $ for the CSHS game in section \ref{CHSH}. Since $ \max \{  \trace{(AM)}\, | \, A \in \LC_{2,2} \} = 2 $ it suffices to show that there is $ A \in \QC_{m,n} $ achieving a better value. 
This is an alternative way to get the same result as in section \ref{CHSH}.

For $ A \in \QC_{2,2} $ we obtain, by lemma \ref{LemQC}, Cauchy-Schwarz and $ \modul{y_i}\le 1 $,
\begin{align*}
\trace{(AM)}&= \langle x_1,y_1 \rangle + \langle x_1,y_2 \rangle + \langle x_2,y_1 \rangle - \langle x_2,y_2 \rangle   \\
&= \langle x_1+x_2,y_1 \rangle + \langle x_1-x_2,y_2 \rangle  \le \modul{x_1+x_2}\modul{y_1} + \modul{x_1-x_2}\modul{y_2}  \\
&\le \modul{x_1+x_2} + \modul{x_1-x_2}.
\end{align*} 
Observing that for $ \modul{x_1},\modul{x_2}\le  1 $
\begin{align*}
&(\modul{x_1+x_2} + \modul{x_1-x_2})^2 \\
&= \langle x_1+x_2,x_1+x_2\rangle  +\sqrt{\langle x_1+x_2,x_1+x_2\rangle  \langle x_1-x_2,x_1-x_2\rangle} +  \langle x_1-x_2,x_1-x_2\rangle  \\
&\le 2\modul{x_1}^2+2\modul{x_2}^2 + \sqrt{\modul{x_1}^4+2\modul{x_1}^2\modul{x_2}^2-4\langle x_1,x_2\rangle^2+\modul{x_2}^4}  \\
&\le  2\modul{x_1}^2+2\modul{x_2}^2 + \sqrt{4(\modul{x_1}^2+\modul{x_2}^2)^2}  \\
&= 4(\modul{x_1}^2+\modul{x_2}^2),
\end{align*}
we can give a precise upper bound for $ \trace{(AM)} $ by 
\begin{align*}
\modul{x_1+x_2} + \modul{x_1-x_2} \le 2\sqrt{\modul{x_1}^2+\modul{x_2}^2} \le 2 \sqrt{2}.
\end{align*}
Thus, we just have to find a matrix that satisfies this bound. A possible choice is induced by the vectors $ x_1 = x_2 = \frac{1}{\sqrt{2}}(1,1) $ and $ y_1 = y_2 =(1,0) $, that is $ A = \frac{1}{\sqrt{2}}\begin{pmatrix}
1 & 1 \\ 1 & 1 
\end{pmatrix} $ yielding the value $ \trace{(AM)} = 2 \sqrt{2} $.
So as we have seen, the inclusion $ \LC_{m,n} \subset \QC_{m,n} $ is strict in general. Elements in $ \QC_{m,n} \setminus \LC_{m,n} $ are called {\itshape non-local}.
Generally, linear functionals $ f$ mapping an $ m \times n $ to a real number that satisfy $ f(A) \le 1 $ for all $ A \in \LC_{m,n} $ are called {\itshape Bell correlation inequalities}. In the case of $ f(A) > 1 $ for some $ A \in \QC_{m,n} $ we talk about {\itshape quantum violations}. 