\section{Bilinear forms and inner products}
\subsection{Basic definitions}

Let $ V $ and $ W $ be two vector spaces and $ k  $ a field. A {\itshape bilinear form} is a map $ \beta: \, V \times W \to k $ which is linear in both variables, that is 
\begin{enumerate}
	\item [i)] $\beta(v_1+v_2,w) = \beta(v_1,w) + \beta(v_2,w)  $
	\item  [ii)]$\beta(\lambda v,w)= \lambda \beta(v,w) $
	\item [iii)]$ \beta(v,w_1+w_2) = \beta(v,w_1)+ \beta(v,w_2) $
	\item [iv)]$ \beta(v,\lambda w) = \lambda\beta(v,w) $
\end{enumerate}
for all $ v,v_1,v_2 \in V, \, w,w_1,w_2 \in W, \, \lambda \in k $. 
If $ V = W $, we call $ \beta $ {\itshape symmetric} if $ \beta(v,w) = \beta(w,v) $, {\itshape positive semidefinite} if 
$ \beta(v,v) \ge 0 $ and {\itshape positive definite} if $ \beta $ is positive semidefinite and $ \beta(v,v)= 0 $ implies that $ v = 0 $. 
If $ \beta: \, V \times V \to k $ is a symmetric positive definite bilinear form it is called an {\itshape inner product}.
Note that if $ H $ is a positive semidefinite operator then $ \beta(v,w) = v^TH\bar{w} $ defines an positive semidefinite bilinear form and an inner product if $ H $ is positive definite. 
Conversely, each positive semidefinite bilinear form $ \beta $ can be written as $ \beta(v,w) = v^TH\bar{w} $ for an Hermitian operator $ H $.

\subsection{How to derive an inner product from a symmetric positive semidefinite bilinear form}\label{App1}
Suppose we have a $ k- $vector space $ V $ equipped with symmetric positive semidefinite bilinear form $ \beta: \, V \times V \to k $. We want to derive a vector space $ U $ that is equipped with an inner product which is induced by $ \beta $. The idea is to to consider the quotient space $ U:=V/\ker\beta $ where $ \ker \beta = \{ v \in V \, | \, \beta(v,w)= 0 \text{ for all } w \in V \} $. Note that the Cauchy-Schwartz inequality $ \beta(v,w)^2 \le \beta(v,v)\beta(w,w) $ implies that 
$ \ker \beta = \{ v \in V \, | \, \beta(v,v)= 0 \} $.
We define $ \tilde{\beta}:  \, U \times U \to k $ by $ \tilde{\beta}([v],[w]) = \beta(v,w) $ where $ [v] = v + \ker\beta, \,$$[w]=w+\ker\beta$.

We have to show that $ \tilde{\beta} $ is well-defined. Therefore, let $ [v]=[v^{\prime}] $, so $v^{\prime}-v \in \ker \beta $. 
For an arbitrary $ [w] \in U $ yields 
\begin{align*}
	\beta([v],[w]) = \beta(v,w) = \beta(v,w)+\beta(v^{\prime}-v,w)= \beta(v^{\prime},w) = \tilde{\beta}([v^{\prime}],[w]).
\end{align*}
The symmetry of $ \beta $ combined with the observation above ensures $ \tilde{\beta}([v],[w]) = \tilde{\beta}([v],[w^{\prime}]) $ for $ [w]= [w^{\prime}] $.

Finally, we get the following equivalence relations: 
\begin{align*}
	\tilde{\beta}([v],[v]) = 0 \Leftrightarrow \beta(v,v) = 0 \Leftrightarrow v \in \ker \beta \Leftrightarrow [v]= \ker \beta,
\end{align*}
which implies that $ \tilde{\beta} $ defines an inner product on $ U $. 

We are also able to analyze the structure of $ U $ without big effort. Let $ \{ v_1,\hdots,v_k, \hdots v_n \} $ be a basis for $ V $ such that $ \ker \beta = \textup{span} \{ v_1,\hdots v_k \}$. So, if we take two elements 
$ v = \sum_{i=1}^{k}a_iv_i + \sum_{i=k+1}^{n}a_iv_i$ and $ w = \sum_{i=1}^{k}b_iv_i + \sum_{i=k+1}^{n}b_iv_i $ then 
\begin{align*}
	[v] = [w] \Leftrightarrow v-w \in \textup{span} \{  v_1,...,v_k \} \Leftrightarrow (a_{k+1},\hdots a_n) = (b_{k+1},\hdots b_n).
\end{align*} 
Hence, we can deduce that $ U \cong \textup{span} \{ v_{k+1},...,v_n \} $. More generally, if $ V = V_1 \oplus V_2 $, then $ V/V_1 \cong V_2 $.