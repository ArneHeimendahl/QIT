Dinge, die definiert werden sollten. 
\begin{enumerate}
	\item norms
	\item Notation, operators of norms 
	\item  perhaps what a state is 
\end{enumerate}


\subsection{How to derive an inner product from a symmetric positive semidefinite bilinear form}
Suppose we have a $ k- $vector space $ V $ equipped with symmetric positive semidefinite bilinear form $ \beta: \, V \times V \to k $. We want to derive a vector space $ U $ that is equipped with an inner product which is induced by $ \beta $. The idea is to to consider the quotient space $ U:=V/\ker\beta $ where $ \ker \beta = \{ v \in V \, | \, \beta(v,w)= 0 \text{ for all } w \in V \} $. Note that the Cauchy-Schwartz inequality $ \beta(v,w)^2 \le \beta(v,v)\beta(w,w) $ implies that 
$ \ker \beta = \{ v \in V \, | \, \beta(v,v)= 0 \} $.
We define $ \tilde{\beta}:  \, U \times U \to k $ by $ \tilde{\beta}([v],[w]) = \beta(v,w) $ where $ [v] = v + \ker\beta, \,$$[w]=w+\ker\beta$.

We have to show that $ \tilde{\beta} $ is well-defined. Therefore, let $ [v]=[v^{\prime}] $, so $v^{‘}-v \in \ker \beta $. 
For an arbitrary $ [w] \in U $ yields 
\begin{align*}
	\beta([v],[w]) = \beta(v,w) = \beta(v,w)+\beta(v^{\prime}-v,w)= \beta(v^{\prime},w) = \tilde{\beta}([v^{\prime}],[w]).
\end{align*}
The symmetry of $ \beta $ combined with the observation above ensures $ \tilde{\beta}([v],[w]) = \tilde{\beta}([v],[w^{\prime}]) $ for $ [w]= [w^{\prime}] $.

Finally, we get the following equivalence relations: 
\begin{align*}
	\tilde{\beta}([v],[v]) = 0 \Leftrightarrow \beta(v,v) = 0 \Leftrightarrow v \in \ker \beta \Leftrightarrow [v]= \ker \beta,
\end{align*}
which implies that $ \tilde{\beta} $ defines an inner product on $ U $. 

We are also able to analyze the structure of $ U $ without big effort. Let $ \{ v_1,\hdots,v_k, \hdots v_n \} $ be a basis for $ V $ such that $ \ker \beta = \textup{span} \{ v_1,\hdots v_k \}$. So, if we take two elements 
$ v = \sum_{i=1}^{k}a_iv_i + \sum_{i=k+1}^{n}a_iv_i$ and $ w = \sum_{i=1}^{k}b_iv_i + \sum_{i=k+1}^{n}b_iv_i $ then 
\begin{align*}
	[v] = [w] \Leftrightarrow v-w \in \textup{span} \{  v_1,...,v_k \} \Leftrightarrow (a_{k+1},\hdots a_n) = (b_{k+1},\hdots b_n).
\end{align*} 
Hence, we can deduce that $ U \cong \textup{span} \{ v_{k+1},...,v_n \} $. More generally, if $ V = V_1 \oplus V_2 $, then $ V/V_1 \cong V_2 $.