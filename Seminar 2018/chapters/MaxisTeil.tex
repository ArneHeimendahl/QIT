In the last section we have seen that $\QC_{m,n}\subset \LC_{m,n}$. In this section we will prove the fact that $\QC_{m,n}\subset K\LC_{m,n}$, which itself isn't surprising, since both sets are convex. However we can find a universal $K$ independent of $m$ and $n$. In particular, we can show, that $K$ is the Grothendieck constant. 

\begin{lemma}[Grothendieck's identity]\label{lem:G_id}
	Let $x,y\in\mathbb{R}^d$ be unit vectors. Let $r\in\mathbb{R}^d$ be a random unit vector chosen from $O(d)$-invariant probability distribution on the unit sphere. Then
	\begin{enumerate}
		\item[i,] $\mathbb{P}[\sgn(\sclr{x}{r})\neq\sgn(\sclr{y}{r})]=\frac{\arccos(\sclr{x}{y})}{\pi}$
		\item[ii,] $\mathbb{E}[\sgn(\sclr{x}{r})\sgn(\sclr{y}{r})]=\frac{2}{\pi}\arcsin(\sclr{x}{y}).$
	\end{enumerate}
\end{lemma}
\begin{proof}
	For the proof of $i,$ assume that $x$ and $y$ are linearly dependent. Since both, $x$ and $y$, are unit vectors, $\arccos(\sclr{x}{y}) = \arccos(1)=0$ if $x=y$ or $\arccos(\sclr{x}{y}) = \arccos(-1) = \pi$ if $x=-y$.
	
	Conversely assume that $x$ and $y$ are linearly independent, i.\,e. $\operatorname{dim}(\spn\{x,y\})=2$. Now project $r$ orthogonally on the plane spanned by $x$ and $y$. This gives us a vector $s\in \spn\{x,y\}$ with $\sclr{x}{r} = \sclr{x}{s}$ and $\sclr{y}{r} = \sclr{y}{s}$. The unit vector $n\coloneqq s/\norm{s}$ is uniformly distributed on the \textcolor{red}{unit circle/disk} that occurs if we consider the intersection of the unit sphere and $\spn\{x,y\}$ by the $O(d)$-invariance of the probability distribution. \\
	
	\textcolor{red}{Noch naeher auf diese Gleichverteilung eingehen?}
	
	\begin{align*}
		\mathbb{P}[\sgn(\sclr{x}{r})\neq\sgn(\sclr{y}{r})]= \mathbb{P}[\sgn(\sclr{x}{n})\neq\sgn(\sclr{y}{n})] 
	\end{align*} 
	
	\noindent\begin{minipage}{\textwidth}	
		If $n$ lies on the segment of the unit circle induced by the green part, the angle between $x$ and $n$ as well as between $y$ and $n$ is smaller than $\pi/2$, hence $\sclr{x}{n}$ and $\sclr{y}{n}$ are positive. 
		\begin{wrapfigure}{r}{0.4\textwidth}
			\vspace{-20pt}
			\begin{center}
				\includegraphics[width=0.38\textwidth]{chapters/fig_unit_circle.pdf}
			\end{center}
			\vspace{-20pt}
		\end{wrapfigure}
		Otherwise, if $n$ lies on the segment of the unit circle induced by the blue part, the angle between both $x$ and $n$ as well as $y$ and $n$ is greater than $3\pi/2$, hence $\sclr{x}{n}$ and $\sclr{y}{n}$ are negative.
		
		\hspace{12pt} Now, if we want to calculate the probability that the signs of the two scalar products disagree, we are interested in the undyed segments of the unit circle. Thus, it is sufficient to calculate the periphery of the undyed segments of the circle. In particular on the unit circle, the angle between two vectors equals the periphery of the segment of the circle between those two vectors. 
			
		\hspace{12pt} Because $\gamma$ and $\delta$ are vertical angles they are both equal. Furthermore, $\alpha$ and $\beta$ have to be equal too, since $\gamma$ and $\delta$ are equal and $\alpha+\delta = \beta+\gamma = \pi/2$. With $\alpha = \arccos(\sclr{x}{y}) - \pi/2$ the first part of Lemma \ref{lem:G_id} follows:
	\end{minipage}
	
	\begin{align*}
		\mathbb{P}[\sgn(\sclr{x}{n})\neq\sgn(\sclr{y}{n})]=2\frac{\frac{\pi}{2}+\alpha}{2\pi} = \frac{\arccos(\sclr{x}{y})}{\pi}.
	\end{align*}
	
	\noindent We conclude with the proof of the second part of Lemma \ref{lem:G_id}: 
	\begin{align*}
		&\mathbb{E}[\sgn(\sclr{x}{r}) \sgn(\sclr{y}{r})] \\
		&\qquad= 1\cdot\mathbb{P}[\sgn(\sclr{x}{r}) = \sgn (\sclr{y}{r} )] - 1\cdot \mathbb{P}[\sgn(\sclr{x}{r}) \neq \sgn(\sclr{y}{r})] \\
		&\qquad= 1 - 2\mathbb{P}[\sgn(\sclr{x}{r}) \neq \sgn(\sclr{y}{r})] \\
		&\qquad= 1 - 2 \frac{\arccos(\sclr{x}{y})}{\pi} \\
		&\qquad= \frac{2}{\pi} \arcsin(\sclr{x}{y}),
	\end{align*}
	because $\arcsin (t) = \arccos(t) = \pi/2$.
\end{proof}

\begin{lemma}[Krivine's trick]\label{lem:krivines_trick}
	Let $x_1,\dots,x_m,y_1,\dots,y_n\in S^{m+n-1}$ be given. Furthermore, let $r\in\mathbb{R}^d$ be a random unit vector chosen form the $O(d)$-invariant probability distribution on the unit sphere. Then there are $x_1^\prime,\dots,x_m^\prime, y_1^\prime,\dots,y_n^\prime\in S^{m+n-1}$ so that
	\begin{equation}
		\mathbb{E}[\sgn(\sclr{x_i^\prime}{r})\sgn(\sclr{y_j^\prime}{r})] = \beta \sclr{x_i}{y_j},
		\label{eq:krivines_trick}
	\end{equation}		
	with $\beta = \frac{2}{\pi} \ln (1+\sqrt{2}).$
\end{lemma}

\noindent For the proof of \ref{lem:krivines_trick} we need to use the $k$-th tensor product of $\mathbb{R}^n$. The $\mathbb{R}^n$ is an $n$-dimensional Euclidean space with inner product \sclr{\cdot}{\cdot} and orthonormal basis $e_1,\dots,e_n$. The \emph{$k$-th tensor product of $\mathbb{R}^n$} is denoted by $(\mathbb{R}^n)^{\tensor k}$ and it is a Euclidean  vector space of dimension $n^k$ with orthonormal basis $e_{i_1}\tensor e_{i_2} \tensor \cdots \tensor e_{i_k}$, $i_j\in\{1,\dots,n\}$. In particular
\begin{align}
	\sclr{e_{i_1}\tensor \cdots \tensor e_{i_k}}{e_{j_1}\tensor \cdots \tensor e_{j_k}}
	&= \prod_{l=1}^k \sclr{e_{i_l}}{e_{j_l}}\nonumber\\
	&=\begin{cases}
		1 & , \text{ if } i_l=j_l \text{ for all } l=1,\dots,n,\\
		0 & , \text{ otherwise},
	\end{cases} \label{eq:orthonormtensor}
\end{align}
and for $v\in\mathbb{R}^n$ with $v=v_1e_1+\cdots +v_ne_n$ we define $v^{\tensor k} \in (\mathbb{R}^n)^{\tensor k}$ by 
\begin{equation}
	v^{\tensor k} = (v_1e_1 + \cdots + v_ne_n) \tensor \cdots \tensor (v_1e_1 + \cdots + v_ne_n) = \sum_{i_1,\dots,i_k} v_{i_1}\cdots v_{i_k} e_{i_1}\tensor\cdots\tensor e_{i_k},
\end{equation}
where the last equation follows by the distributive law (identity $ii,$ of the tensor product). 
Thus, for $v,w\in\mathbb{R}^n$ 
\begin{align}
	\sclr{v^{\tensor k}}{w^{\tensor k}}
%	&\overset{\textcolor{white}{\eqref*{eq:orthonormtensor}}}{=} \sum_{i_1,\dots,i_k} v_{i_1}\cdots v_{i_k}\left(e_{i_1}\tensor\cdots\tensor e_{i_k}\right)^\top \sum_{j_1,\dots,j_k} w_{j_1}\cdots w_{j_k}(e_{j_1}\tensor\cdots\tensor e_{j_k}) \nonumber\\
	&\overset{\textcolor{white}{\eqref*{eq:orthonormtensor}}}{=} \sum_{i_1,\dots,i_k} v_{i_1}\cdots v_{i_k} \sum_{j_1,\dots,j_k} w_{j_1}\cdots w_{j_k} \sclr{e_{i_1}\tensor\cdots\tensor e_{i_k}}{e_{j_1}\tensor\cdots\tensor e_{j_k}} \nonumber\\
	&\overset{\eqref{eq:orthonormtensor}}{=} \sum_{i_1,\dots,i_k} v_{i_1}\cdots v_{i_k}w_{i_1}\cdots w_{i_k} \nonumber\\
	&\overset{\textcolor{white}{\eqref*{eq:orthonormtensor}}}{=}(\sum_{i=1}^n v_iw_i)^k = \sclr{v}{w}^k \label{eq:kth_tensor}
\end{align}
\begin{proof}
	Define the function $E: [-1,+1] \to [-1,+1]$ by $E(t)=\frac{2}{\pi}\arcsin(t)$. Due to Grothendieck's identity (Lemma \ref{lem:G_id}):
	\begin{align*}
		E(\sclr{x_i^\prime}{y_j^\prime} ) &= \mathbb{E}[\sgn(\sclr{x_i^\prime}{r})\sgn(\sclr{y_j^\prime}{r})]\\
		&\overset{!}{=}\beta \sclr{x_i}{y_j}.
	\end{align*}
	
	\noindent Idea: To find $\beta,x_i^\prime,y_j^\prime$ we invert $E$:
	\[
		\sclr{x_i^\prime}{y_j^\prime} = E^{-1} (\beta \sclr{x_i}{y_j})	
	\]
	with 
	\begin{align*}
		E^{-1}(t) &= \sin(\pi/2 \cdot t) \\
		&= \sum_{k=0}^\infty \underbrace{\frac{(-1)^{2k+1}}{(2k+1)!}\left(\frac{\pi}{2}\right)^{2k+1}}_{g_{2k+1}}  t^{2k+1}
	\end{align*}
	which is valid for all $t\in[-1,+1]$.
	
	Define the infinite-dimensional Hilbert space
	\begin{equation}
		H= \bigoplus_{r=0}^\infty (\mathbb{R}^{m+n})^{\tensor 2k+1}.
	\end{equation}
	
	Define $\tilde{x}_i, \tilde{y}_j\in H$, $i=1,\dots,m,j=1,\dots,n$ componentwise:
	\begin{align*}
		(\tilde{x}_i)_k &= \sgn(g_{2k+1}) \sqrt{\modul{g_{2k+1}}\beta^{2k+1}}\, x_i^{\tensor 2k+1} \\
		(\tilde{y}_j)_k &= \sqrt{\modul{g_{2k+1}}\beta^{2k+1}} \,y_j^{\tensor 2k+1}
	\end{align*}
	Then 
	\begin{align*}
		\sclr{\tilde{x}_i}{\tilde{y}_j} &\overset{\textcolor{white}{\eqref*{eq:kth_tensor}}}{=} \sum_{k=0}^\infty g_{2k+1} \beta^{2k+1}\sclr{x_i^{\tensor 2k+1}}{y_j^{\tensor 2k+1}} \\
		&\overset{\eqref{eq:kth_tensor}}{=} \sum_{k=0}^\infty g_{2k+1} \beta^{2k+1} \sclr{x_i}{y_j}^{2k+1} \\
		&\overset{\textcolor{white}{\eqref*{eq:kth_tensor}}}{=} E^{-1}(\beta \sclr{x_i}{y_j}).
	\end{align*}
	
	Hence, $\beta$ is defined by the condition that the vectors $\tilde{x}_i,\dots,\tilde{x}_m,\tilde{y}_1,\dots,\tilde{y}_n$ are unit vectors, that is
	\[
		1 = \sclr{\tilde{x}_i}{\tilde{x}_i} = \sclr{\tilde{y}_j}{\tilde{y}_j}
		%= \sum_{k=0}^\infty \modul{g_{2k+1}} \beta^{2k+1} 
		= \sum_{k=0}^\infty \frac{1}{(2k+1)!}\left(\frac{\pi}{2}\right)^{2k+1}\beta^{2k+1}=\sinh(\frac{\pi}{2}\beta).
	\]
	Consequently
	\[
		\beta = \frac{2}{\pi} \arcsinh(1) = \frac{2}{\pi}\ln(1+\sqrt(2)),	
	\]
	since $\arcsinh (t) = \ln(t+\sqrt{t^2+1})$.

	The only thing that is left to prove, is that the solution of the maximization problem yields vectors $x_1^\prime,\dots,x_m^\prime, y_1^\prime,\dots,y_n^\prime\in S^{m+n-1}$, since our vectors $\tilde{x}_1,\dots,\tilde{x}_m,\tilde{y}_1,\dots,\tilde{y}_n$ are infinite-dimensional. 
	For this reason consider the real matrix $G\in\mathbb{R}^{(m+n)\times(m+n)}$ given by
	\begin{equation}
		G=\begin{pmatrix}
			\sclr{\tilde{x}_1}{\tilde{x}_1} & \cdots & \sclr{\tilde{x}_1}{\tilde{x}_m}& \sclr{\tilde{x}_1}{\tilde{y}_1} & \cdots & \sclr{\tilde{x}_1}{\tilde{y}_n} \\
			 \vdots		& \ddots	& \vdots & \vdots & \ddots & \vdots\\
			 \sclr{\tilde{x}_m}{\tilde{x}_1} & \cdots & \sclr{\tilde{x}_m}{\tilde{x}_m}& \sclr{\tilde{x}_m}{\tilde{y}_1} & \cdots & \sclr{\tilde{x}_m}{\tilde{y}_n} \\
			\sclr{\tilde{y}_1}{\tilde{x}_1} & \cdots & \sclr{\tilde{y}_1}{\tilde{x}_m}& \sclr{\tilde{y}_1}{\tilde{y}_1} & \cdots & \sclr{\tilde{y}_1}{\tilde{y}_n} \\
			 \vdots		& \ddots	& \vdots& \vdots & \ddots & \vdots\\
			 \sclr{\tilde{y}_n}{\tilde{x}_1} & \cdots & \sclr{\tilde{y}_n}{\tilde{x}_m}& \sclr{\tilde{y}_n}{\tilde{y}_1} & \cdots & \sclr{\tilde{y}_n}{\tilde{y}_n} 
		\end{pmatrix}
	\end{equation}
	called \emph{Gram matrix}. Let $z\coloneqq (\tilde{x}_1,\dots,\tilde{x}_m,\tilde{y}_1,\dots,\tilde{y}_n)$. By the linearity of the scalar product
	\begin{align*}
		v^\top G v = \sum_{i,j} v_i G_{ij} v_j 
		= \sum_{i,j} v_i \sclr{z_i}{z_j} v_j 
		= \sclr{\sum_i v_i z_i}{\sum_j v_j z_j} 
		= \norm{\sum_i v_i z_i} > 0
	\end{align*}
	for some $v\in\mathbb{R}^{m+n}$, $v\neq 0$. Hence, $G$ is positive definite and symmetric, thus, $G$ can be diagonalized by an orthogonal matrix. This means that there is a decomposition $G=Q\Lambda Q^\top$ with $Q$ a real orthogonal matrix with columns that are the eigenvectors of $G$ and $\Lambda$ a real and diagonal matrix having the eigenvalues of $G$ on the diagonal. Since the eigenvalues of a positive definite matrix are positive, $\Lambda=\Lambda^{1/2}\Lambda^{1/2}$. Thus,
	\[
		G=(Q\Lambda^{1/2})(Q\Lambda^{1/2})^\top		
	\]
	and due to the symmetry of $G$ likewise
	\[
		G=(Q\Lambda^{1/2})^\top(Q\Lambda^{1/2}).	
	\]
	$A\coloneqq Q\Lambda^{1/2}$ is a real $(m+n)\times (m+n)$ matrix and its columns are the vectors we were looking for.
	\textcolor{red}{Muss man hier dann noch begruenden, dass die Vektoren immer noch Einheitsvektoren sind?}
\end{proof}
\begin{dfn}
	For $M\in\mathbb{R}^{m\times n}$ define the quadratic program \textcolor{red}{wieso quadratisch?}
	\begin{align}
		\norm{M}_{\infty\to 1} &= \max \left\{ \sum_{i=1}^m \sum_{j=1}^n M_{ij} \xi_i \eta_j : \xi_i^2=1, i=1,\dots,m, \eta_j^2=1, j =1,\dots,n \right\} \\
		&=\max\left\{\trace{\xi^\top M \eta}: \xi\in\{-1,1\}^m,\eta\in\{-1,1\}^n\right\}.
	\end{align}
\end{dfn}
\textcolor{red}{computing NP-hard}
\begin{dfn}
	\textcolor{red}{sdp relaxation einfuehren}
	
	SDP relaxation of $\norm{M}_{\infty\to1}$:
	\begin{align*}
		\operatorname{sdp}_{\infty\to 1} (M) = \max 
		&\sum_{i=1}^m\sum_{j=1}^n M_{ij} \sclr{x_i}{y_j}\\
		&x_i,y_j\in\mathbb{R}^{m+n}\\
		&\norm{x_i}=1, i=1,\dots,m\\
		&\norm{y_j}=1, j=1,\dots,n.
	\end{align*}
\end{dfn}
\begin{theo}[Grothendieck's inequality] \label{theo:G_ineq}
	There exists a constant $K$ such that for all $M\in\mathbb{R}^{m\times n}$:
	\begin{equation}
		\norm{M}_{\infty\to 1} \leq \operatorname{sdp}_{\infty\to 1} (M) \leq K \norm{M}_{\infty\to 1}.
	\end{equation}
\end{theo}
\begin{proof}
%	\textcolor{red}{wieso again? wo wurde das vorher schon einmal gemacht?}
%	Again: 
	approximation algorithm with randomized rounding
	
	\begin{algorithm}[H]
		\SetAlgoLined
		\caption{Approximation algorithm with randomized rounding for $\norm{M}_{\infty\to 1}$}
	\end{algorithm}
	\begin{itemize}
		\item[1.] Solve $\operatorname{sdp}_{\infty\to 1} (M)$. Let $x_1,\dots,x_m,y_1,\dots,y_n\in S^{m+n-1}$ be the optimal unit vectors
		\item[2.] Apply Krivine's trick (Lemma \ref{lem:krivines_trick}) and use vectors $x_i,y_j$ to create new unit vectors $x_1^\prime,\dots,x_m^\prime, y_1^\prime,\dots,y_n^\prime\in S^{m+n-1}$.
		\item[3.] Choose $r\in S^{m+n-1}$ randomly
		\item[4.] Round: $u_i = \sgn(\sclr{x_i^\prime}{r})$\\
					\textcolor{white}{Round: }$v_j = \sgn(\sclr{y_j^\prime}{r})$
	\end{itemize}
	
	\noindent Expected quality of the outcome:
	\begin{align*}
		\norm{M}_{\infty\to 1} &\geq \mathbb{E}\left[\sum_{i=1}^m\sum_{j=1}^n M_{ij}u_iv_j\right]\\
		&\overset{\textcolor{white}{\eqref*{eq:krivines_trick}}}{=} \sum_{i=1}^m\sum_{j=1}^n M_{ij} \mathbb{E}[\sgn(\sclr{x_i^\prime}{r}) \sgn(\sclr{y_j^\prime}{r})] \\
		&\overset{\textcolor{white}{\eqref*{eq:krivines_trick}}}{=}\sum_{i=1}^m\sum_{j=1}^n M_{ij}\beta \sclr{x_i}{y_j} \\
		&\overset{\eqref{eq:krivines_trick}}{=}\beta \operatorname{sdp}_{\infty\to 1}(M),
	\end{align*}
	where $\beta = \frac{2\ln(1+\sqrt(2)}{\pi}$, thus $K_G\leq \beta^{-1}$.
\end{proof}

\begin{theo}[Grothendieck-Tsirelson]
	There exists an absolute constant $K\geq 1$ such that, for any positive integers $m,n$, the following three equivalent conditions hold:
	\begin{itemize}
		\item[(1)] We have the inclusion 
			\begin{equation}
				\QC_{m,n} \subset K\LC_{m,n}.
			\end{equation}
		\item[(2)] For any $M\in\mathbb{R}^{m\times n}$ and for any $\rho,X_i,Y_j$ verifying the conditions of Definition \textcolor{red}{4.2.1} we have
			\begin{equation}
				\sum_{i,j} M_{ij} \trace{\rho(X_i\tensor Y_j)}\leq K \max_{\xi\in\{-1,1\}^m,\eta\in\{-1,1\}^n} \trace{\xi^\top M \eta}.
			\end{equation}
			\item[(3)] For any $M\in\mathbb{R}^{m\times n}$ and for any (real) Hilbert space vectors $x_i,y_j$ with $\modul{x_i}\leq 1$, $\modul{y_j}\leq 1$ we have
				\begin{equation}
					\sum_{i,j} M_{i,j}\sclr{x_i}{y_j} \leq K \max_{\xi\in\{-1,1\}^m,\eta\in\{-1,1\}^n} \trace{\xi^\top M \eta}. \label{eq:G_ineq}
				\end{equation}
	\end{itemize}
\end{theo}
\begin{proof}
	Since \eqref{eq:G_ineq} is a direct consequence of Grothendieck's inequality the only thing left to prove is the equivalence between (1)-(3). The equivalence of (3) and (2) (the Tsirelson's bound) is a consequence of \textcolor{red}{the proof of Lemma \textcolor{red}{4.2.2}, where we proved that $\trace{\rho(X_i\tensor Y_j)} = \sclr{x_i}{y_j}$.}
	\textcolor{red}{Dann brauchen wir hier noch Argumente aus Exercise 11.3.}
	Finally, the equivalence between (2) and (1) ...
\end{proof}
\begin{theo}[Tsirelson]
	(Hard direction) For all positive integers $n, r$ and any $x_1,\dots,x_n,y_1,\dots,y_n\in S^r$, there exists a positive integer $d\coloneqq d(r)$, a state $\dr{\psi}\in\mathbb{C}^d\tensor\mathbb{C}^d$ and $\{-1,1\}$-observables $F_1,\dots,F_n,G_1,\dots,G_n\in O(\mathbb{C}^d)$ \textcolor{red}{(sollten das hier $X_i$ und $Y_j$ aus dem Buch sein?)}, such that for every $i,j\in\{1,\dots,n\}$, we have
	\begin{equation}
		\dl{\psi} F_i\tensor G_j \dr{\psi} = \sclr{x_i}{y_j}.
	\end{equation}
	Moreover, $d\leq 2^{\lceil r/2 \rceil}$.
	
	(Easy direction) Conversely, for all positive integers $n,d,$ state $\dr{\psi}\in \mathbb{C}^d\tensor\mathbb{C}^d$ and $\{-1,1\}$-observables $F_1,\dots,F_n,G_1,\dots,G_n\in O(\mathbb{C}^d)$, there exist a positive integer $r\coloneqq r(d)$ and $x_1,\dots,x_n,y_1,\dots,y_n\in S^r$ such that for every $i,j\in\{1,\dots,n\}$, we have
	\begin{equation}
		\sclr{x_i}{y_j} = \dl{\psi}F_i\tensor G_j \dr{\psi}.
	\end{equation}
	Moreover, $r\leq 2d^2$.
\end{theo}
\begin{proof}
	We start by proving the hard direction. 
	
	Reminder: Pauli matrices
	
	Define for each $l=1,\dots,\lceil r/2\rceil$, the \emph{$d$-by-$d$ Clifford matrices},
	\begin{align}
		S_{2l+1} &= Z^{\tensor (l-1)} \tensor X \tensor I^{\tensor(\lceil r/2\rceil - l)},\\
		S_{2l} &= Z^{\tensor (l-1)} \tensor Y \tensor I^{\tensor(\lceil r/2\rceil - l)}.
	\end{align}
	We will prove, that just as the Pauli matrices the Clifford matrices square to the identity matrix (of size $d$-by-$d$) and pair-wise anti-commute. 

	Let $l\in\{1,\dots,\lceil r/2\rceil\}$. Then
	\begin{align*}
		(S_{2l+1})^2 &= (Z^{\tensor (l-1)} \tensor P_l \tensor I^{\tensor(\lceil r/2\rceil - l)}) (Z^{\tensor (l-1)} \tensor P_l \tensor I^{\tensor(\lceil r/2\rceil - l)}) \\
		&= (Z^2)^{\tensor (l-1)} \tensor (P_l^2) \tensor I^{\tensor (\lceil r/2 \rceil -l)} \\
		&= I^{\tensor \lceil r/2 \rceil},
	\end{align*}
	by identity $iii,$ of the tensor product and the properties of the Pauli matrices. The same applies for $S_{2l}$. Thus $d\leq 2^{\lceil r/2 \rceil}$.
	
	Furthermore, let $k,l\in\{1,\dots, \lceil r/2 \rceil\}$, $S_k\in\{S_{2k+1},S_{2k}\}$, $S_l\in\{S_{2l+1},S_{2l}\}$ and corresponding $P_k,P_l\in\{X,Y\}$. W.\,l.\,o.\,g. let $k<l$, then
	\begin{align*}
		S_{k}S_{l} &= (Z^{\tensor (k-1)} \tensor P_k \tensor I^{\tensor(\lceil r/2\rceil - k)})(Z^{\tensor (l-1)} \tensor P_l \tensor I^{\tensor(\lceil r/2\rceil - l)})\\
		&=(Z^2)^{\tensor (k-1)} \tensor (P_k Z) \tensor (I Z)^{\tensor (l-k-1)} \tensor (I P_l) \tensor I^{\tensor \lceil r/2\rceil -l} \\
		&= I^{\tensor (k-1)} \tensor (P_k Z) \tensor Z^{\tensor (l-k-1)} \tensor P_l \tensor I^{\tensor \lceil r/2\rceil -l}\\
		&= I^{\tensor (k-1)} \tensor (- Z P_k) \tensor Z^{\tensor (l-k-1)} \tensor P_l \tensor I^{\tensor \lceil r/2\rceil -l} \\
		&= - S_lS_k,
	\end{align*}
	since $P_k$ is a Pauli matrix.
	
	Additionally, for every $k\neq l$, we have $\trace{S_kS_l}=d$, if $k=l$, and $\trace{S_kS_l}=0$, if $k\neq l$, since $\trace{A\tensor B} = \trace{A}\trace{B}$ and at least $\trace{ZX}=\trace{ZY}=0$.
	
	Define $F_1,\dots,F_n,G_1,\dots,G_n\in\mathbb{C}^{d\times d}$ by
	\begin{align}
		F_i&=\sum_{k=1}^r (x_i)_k S_k,\\
		G_j&=\sum_{k=1}^r (y_j)_k S_k^\top.
	\end{align}
	
	We will start by proving that the matrices $F_1,\dots,F_n,G_1,\dots,G_n$ are $\{-1,1\}$-observables.
	For this reason it is sufficient to show that $F_i^2=G_j^2=I$ for each $i,j\in\{1,\dots,n\}$, as this implies that the matrices have eigenvalues in $\{-1,1\}$. To this end, consider the expansion of $F_i^2$,
	\begin{align*}
		F_i^2 &= \sum_{k,l=1}^r (x_i)_k(x_i)_l S_k S_l \\
		&= \sum_{k=l} (x_i)_k(x_i)_l \underbrace{S_k S_l}_{=I} + \underbrace{\sum_{k<l} (x_i)_k(x_i)_l S_k S_l}_{=\sum_{k>l} (x_i)_l(x_i)_k S_l S_k} + \sum_{k>l} (x_i)_k(x_i)_l S_k S_l \\
		&= \underbrace{\sclr{x_i}{x_i}}_{=1} + \sum_{k>l} (x_i)_k(x_i)_l \underbrace{(S_k S_l+S_l S_k)}_{=0}\\
		&= I, 
	\end{align*}
	since $x_i$ is a unit vector and the Clifford matrices pair-wise anti-commute.
	Of course, the same argument works for $G_j$.
	
	We will continue with proving that for every $i,j\in\{1,\dots,n\}$, we have $\trace{F_iG_j^\top}/d=\sclr{x_i}{y_j}$.
	
	Fix $i,j\in\{1,\dots,n\}$. As before, consider the expansion of the product $F_iG_j^\top$,
	\begin{equation}
		F_iG_j^\top = \sum_{k,l=1}^r (x_i)_k(y_j)_lS_kS_l.
	\end{equation}
	Then,
	\begin{align*}
		\trace{F_iG_j^\top} &= \trace{\sum_{k,l=1}^r (x_i)_k(y_j)_lS_kS_l} 
		= \sum_{k,l=1}^r (x_i)_k(y_j)_l \underbrace{\trace{S_kS_l}}_{=0,\text{ if } k\neq l}
		= \sum_{k} (x_i)_k(y_j)_k \underbrace{\trace{S_k^2}}_{=d}\\
		&= d \sclr{x_i}{y_j} 
	\end{align*}
	
	We now consider the expansion of $\trace{F_iG_j^\top}/d$. Let $\{\dr{1},\dots,\dr{d}\}\subseteq \mathbb{C}^d$ be an orthonormal basis for $\mathbb{C}^d$. Let
	\begin{equation}
		\dr{\psi} = \frac{1}{\sqrt{d}} \sum_{s=1}^d \dr{s} \tensor \dr{s},
	\end{equation}
	be the maximally entangled state. 
	
	We have
	\begin{align*}
		\dl{\psi} F_i\tensor G_j \dr{\psi} &= \frac{1}{d} \sum_{s,t=1}^d \dl{s}\tensor\dl{s} F_i\tensor G_j \dr{t}\tensor \dr{t} \\
		&= \frac{1}{d} \sum_{s,t=1}^d \dl{s} F_i \dr{t} \dl{s} G_j \dr{t} \\
		&= \frac{1}{d} \trace{F_iG_j^\top}\\
		&= \sclr{x_i}{y_j}.
	\end{align*}
	
%	(Easy direction) Note that since $\dr{\psi}$ hat norm 1 and the observables $F_i$ and $G_j$ are unitary operators, $F_i\tensor I \dr{\psi}$ and $I\tensor G_j\dr{\psi}$ are unit vectors in $\mathbb{C}^{d^2}$. Additionally, note that since $F_i$ and $G_j$ are Hermitian, we have that the inner product
%	\[
%			
%	\]
\end{proof}
\newpage