\documentclass[11pt]{amsart}

\usepackage{amssymb}
\usepackage{dutchcal}        %nice Hilbertspace symbols
\usepackage{braket}          %for braket notation
\usepackage[utf8]{inputenc}
\usepackage{bm}              %bold writing
\usepackage{tikz}            %for drawing some nice graphics
\usepackage{tikzsymbols}
\usepackage[version=3]{mhchem}%chemical formulas nicely written
\usepackage{graphicx}        %some graphics stuff, like subfigures
\usepackage{subcaption}
\usepackage{wrapfig}         %figures in text float
\usepackage{algorithm2e}
\usepackage{cite}            %to cite
\usepackage{dsfont}           %for unity matrix
\usepackage{nicefrac}
\usepackage{xcolor}
\usepackage{xspace}
\usepackage{setspace}
\usepackage{tikz}
\usepackage{mathtools}
\usepackage{enumitem}
\usepackage[top=2.5cm, bottom=2.5cm, left=3.5cm, right=3.5cm]{geometry}
\usepackage{lmodern}		% nice schriftart
\usepackage{etoolbox}        %reformat sections etc.
\usepackage{float}									% forcieren von H bei Grafiken
\usepackage[unicode=true,							% Hyperlinks
bookmarks=true,bookmarksnumbered=false,bookmarksopen=false,
breaklinks=false,pdfborder={0 0 0},backref=false,colorlinks=false]
{hyperref}
\DeclareMathAlphabet{\mathpzc}{OT1}{pzc}{m}{it}

\makeatletter
\let\@afterindenttrue\@afterindentfalse
\makeatother
\pagestyle{plain}   %no title at every page
\graphicspath{ {images/} }





%--------------------------only for tikz---------------------------------------------------------
\newcommand{\xdownarrow}[1]{%
	{\left\downarrow\vbox to #1{}\right.\kern-\nulldelimiterspace}
}
\newcommand{\xuparrow}[1]{%
	{\left\uparrow\vbox to #1{}\right.\kern-\nulldelimiterspace}
}

\newcommand{\xswarrow}[1]{%
	{\left\swarrow\vbox to #1{}\right.\kern-\nulldelimiterspace}
}

\usetikzlibrary{shapes.callouts,calc,arrows,decorations.pathmorphing,intersections,
	decorations.pathreplacing
}
\tikzset{
	fatlevel/.style   = { ultra thick, black },
	level/.style   = {thick, black },
	vertex/.style = {very thick, black,opacity=1, shorten <= 0.15cm, shorten >= 0.15cm},
	virtual/.style={thick,densely dashed},
	connect/.style = { dashed, black },
	redtransition/.style = {thick,->,>=stealth',shorten >=1pt, red},
	transition/.style = {thick, black,->,>=stealth',shorten >=1pt},
	trans/.style={thick,<->,>=stealth},
	notice/.style  = { draw, rectangle callout, callout relative pointer={#1} },
	label/.style   = { text width=3cm },
	llabel/.style   = { text width=9.5cm },
	interface/.style={
		% The border decoration is a path replacing decorator. 
		% For the interface style we want to draw the original path.
		% The postaction option is therefore used to ensure that the
		% border decoration is drawn *after* the original path.
		postaction={draw,decorate,decoration={border,angle=135,
				amplitude=0.3cm,segment length=2mm}}},
}
%-----------------------------------------------------------

\definecolor{colororange}{HTML}{E65100} % orange
\definecolor{colordgray}{HTML}{795548} % dark gray for note
\definecolor{colorhgray}{HTML}{212121} % heavy dark gray for normal text
\definecolor{colorgreen}{HTML}{009688} % green
\definecolor{colorlgray}{HTML}{FAFAFA} % background light gray
\definecolor{colorblue}{HTML}{0277BB} % blue

%-----------------------------------------------------------








%------------------------- Theorem enviroments -----------------------------------------
\newtheoremstyle{mytheorem}% ⟨name⟩
{12pt}% ⟨Space above⟩
{12pt}% ⟨Space below⟩
{\itshape}% ⟨Body font⟩
{}% ⟨Indent amount⟩
{\bf}% ⟨Theorem head font⟩
{.}% ⟨Punctuation after theorem head⟩
{.5em}% ⟨Space after theorem head⟩
{}% ⟨Theorem head spec (can be left empty, meaning ‘normal’)⟩
\theoremstyle{mytheorem}
\newtheorem{theo}{Theorem}

\newtheoremstyle{mylemma}% ⟨name⟩
{12pt}% ⟨Space above⟩
{12pt}% ⟨Space below⟩
{\itshape}% ⟨Body font⟩
{}% ⟨Indent amount⟩
{\bf}% ⟨Theorem head font⟩
{.}% ⟨Punctuation after theorem head⟩
{.5em}% ⟨Space after theorem head⟩
{}% ⟨Theorem head spec (can be left empty, meaning ‘normal’)⟩
\theoremstyle{mytheorem}
\newtheorem{lemma}[subsubsection]{Lemma}

\newtheoremstyle{kor}% ⟨name⟩
{12pt}% ⟨Space above⟩
{12pt}% ⟨Space below⟩
{\itshape}% ⟨Body font⟩
{}% ⟨Indent amount⟩
{\bf}% ⟨Theorem head font⟩
{.}% ⟨Punctuation after theorem head⟩
{.5em}% ⟨Space after theorem head⟩
{}% ⟨Theorem head spec (can be left empty, meaning ‘normal’)⟩
\theoremstyle{mytheorem}
\newtheorem{kor}{Corollary}

\newtheoremstyle{prop}% ⟨name⟩
{12pt}% ⟨Space above⟩
{12pt}% ⟨Space below⟩
{\itshape}% ⟨Body font⟩
{}% ⟨Indent amount⟩
{\bf}% ⟨Theorem head font⟩
{.}% ⟨Punctuation after theorem head⟩
{.5em}% ⟨Space after theorem head⟩
{}% ⟨Theorem head spec (can be left empty, meaning ‘normal’)⟩
\theoremstyle{mytheorem}
\newtheorem{prop}{Proposition}

\newtheoremstyle{conj}% ⟨name⟩
{12pt}% ⟨Space above⟩
{12pt}% ⟨Space below⟩
{\itshape}% ⟨Body font⟩
{}% ⟨Indent amount⟩
{\bf}% ⟨Theorem head font⟩
{.}% ⟨Punctuation after theorem head⟩
{.5em}% ⟨Space after theorem head⟩
{}% ⟨Theorem head spec (can be left empty, meaning ‘normal’)⟩
\theoremstyle{mytheorem}
\newtheorem{conj}{Conjecture}

\newtheoremstyle{mydef}% ⟨name⟩
{12pt}% ⟨Space above⟩
{16pt}% ⟨Space below⟩
{}% ⟨Body font⟩
{}% ⟨Indent amount⟩
{\bf}% ⟨Theorem head font⟩
{.}% ⟨Punctuation after theorem head⟩
{.5em}% ⟨Space after theorem head⟩
{}% ⟨Theorem head spec (can be left empty, meaning ‘normal’)⟩
\theoremstyle{mydef}
\newtheorem{dfn}[subsubsection]{Definition}


\newtheoremstyle{myremark}% ⟨name⟩
{12pt}% ⟨Space above⟩
{16pt}% ⟨Space below⟩
{\itshape}% ⟨Body font⟩
{}% ⟨Indent amount⟩
{\bf}% ⟨Theorem head font⟩
{}% ⟨Punctuation after theorem head⟩
{.5em}% ⟨Space after theorem head⟩
{}% ⟨Theorem head spec (can be left empty, meaning ‘normal’)⟩
\theoremstyle{myremark}
\newtheorem{rmk}{Remark}
\newtheorem*{rmk*}{Remark}
\newtheorem*{qst*}{Question}


\newtheoremstyle{myex}% ⟨name⟩
{0pt}% ⟨Space above⟩
{16pt}% ⟨Space below⟩
{\itshape}% ⟨Body font⟩
{}% ⟨Indent amount⟩
{\bf}% ⟨Theorem head font⟩
{}% ⟨Punctuation after theorem head⟩
{.5em}% ⟨Space after theorem head⟩
{}% ⟨Theorem head spec (can be left empty, meaning ‘normal’)⟩
\theoremstyle{myex}
\newtheorem{ex}{Example}
\newtheorem*{ex*}{Example}





%-------------------------- reformat sections and subsections
\patchcmd{\subsection}{\bfseries}{\itshape\bfseries\centering}{}{}
\patchcmd{\subsection}{-.5em}{.5em}{}{}
\patchcmd{\section}{\normalfont}{\normalfont\bfseries\Large}{}{}
\patchcmd{\section}{-.5em}{.5em}{}{}
\patchcmd{\subsubsection}{\normalfont}{\itshape\bfseries\centering}{}{}
\patchcmd{\subsubsection}{-.5em}{.5em}{}{}


%---------------- space between page numbering
	\addtolength{\textheight}{-\baselineskip}
	\addtolength{\footskip}{\baselineskip}


%---------------- section spacing and table of contents formatting
	\setcounter{tocdepth}{3}% to get subsubsections in toc
	\let\oldtocsection=\tocsection
	\let\oldtocsubsection=\tocsubsection
	\let\oldtocsubsubsection=\tocsubsubsection
	\renewcommand{\tocsection}[2]{\hspace{0em}\oldtocsection{#1}{#2}}
	\renewcommand{\tocsubsection}[2]{\hspace{2em}\oldtocsubsection{#1}{#2}}
	\renewcommand{\tocsubsubsection}[2]{\hspace{4em}\oldtocsubsubsection{#1}{#2}}





%---------------------------- definitions
\newcommand{\el}{ {\scriptstyle{\in}}\; }
\newcommand{\sumel}{ {\scriptscriptstyle{\in}} }
\newcommand{\Hilb}{ {\mathcal{H}} }
\newcommand{\dualHilb}{ {\mathcal{H}^*} }
\newcommand{\Haml}{ {\hat{H}} }
\newcommand{\modul}[1]{{\left|{#1}\right|}}
\newcommand{\norm}[1]{\ensuremath{ \Vert {#1} \Vert}}
\newcommand{\trace}[1]{\ensuremath{\text{Tr} \, {#1} }}
\newcommand{\QC}{\ensuremath{\textup{QC}}}
\newcommand{\LC}{\ensuremath{\textup{LC}}}
\newcommand{\sclr}[2]{\ensuremath{\langle{#1},{#2}\rangle}}
\newcommand{\tensor}{\ensuremath{\otimes}}
\newcommand{\dl}[1]{\ensuremath{\langle {#1} \vert}}
\newcommand{\dr}[1]{\ensuremath{\vert {#1} \rangle}}


\makeatletter
\def\eqref{\@ifstar\@eqref\@@eqref}
\def\@eqref#1{\textup{\tagform@{\ref*{#1}}}}
\def\@@eqref#1{\textup{\tagform@{\ref{#1}}}}
\makeatother

\DeclareMathOperator*{\esssup}{ess\,sup} % essentiellen Supremums
\DeclareMathOperator{\spn}{span} % Span
\DeclareMathOperator{\supp}{supp} % Träger
\DeclareMathOperator{\proj}{proj} % Träger
\DeclareMathOperator{\tr}{Tr} 
\DeclareMathOperator{\sgn}{sign}
\DeclareMathOperator{\Real}{Re}
\DeclareMathOperator{\Imag}{Im}
\DeclareMathOperator{\arcsinh}{arcsinh}
