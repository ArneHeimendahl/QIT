\documentclass[10pt]{beamer}

%Language package english
\usepackage[english]{babel}
\usepackage[T1]{fontenc}
\usepackage[utf8]{inputenc}

\usepackage{dutchcal}        %nice Hilbertspace symbols
\usepackage{braket}          %for braket notation

%\setbeamercovered{transparent}  %in order to see shape of the following bullet points 

%for striking out text
\usepackage[normalem]{ulem}




%fuer Identitaet 
\usepackage{dsfont}


%Aussehen der Präsentation
\usetheme{Boadilla}
\usecolortheme{sidebartab}

%\usepackage{lmodern}		% nice schriftart
%\usepackage{etoolbox}        %reformat sections etc.
%---------------------------- definitions
%\newcommand{\el}{ {\scriptstyle{\in}}\; }


\newcommand{\Hilb}{ {\mathcal{H}} }
\newcommand{\dualHilb}{ {\mathcal{H}^*} }
\newcommand{\Haml}{ {\hat{H}} }
\newcommand{\modul}[1]{{\left|{#1}\right|}}
\newcommand{\norm}[1]{\ensuremath{ \Vert {#1} \Vert}}
\newcommand{\trace}[1]{\ensuremath{\text{Tr} \, {#1} }}
\newcommand{\QC}{\ensuremath{\textup{QC}}}
\newcommand{\LC}{\ensuremath{\textup{LC}}}
\newcommand{\sclr}[2]{\ensuremath{\langle{#1},{#2}\rangle}}
\newcommand{\tensor}{\ensuremath{\otimes}}


\DeclareMathOperator*{\esssup}{ess\,sup} % essentiellen Supremums
\DeclareMathOperator{\spn}{span} % Span
\DeclareMathOperator{\supp}{supp} % Träger
\DeclareMathOperator{\proj}{proj} % Träger
\DeclareMathOperator{\tr}{Tr} 
\DeclareMathOperator{\sgn}{sign}
